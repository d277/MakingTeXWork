\chapter[\protect\TeX\ on the Macintosh]{\protect\TeX\ on the\\Macintosh}
\RCSID$Id: ch15.tex,v 1.1 2002/08/23 14:58:46 nwalsh Exp $
\label{chap:mac}

\ifincludechapter\else\endinput\fi

\bgroup % <<<<<<<<<<<<<<<<<<<<<<<<<<<<<<<<<<<<<<<<<<<<<<<<<<<<<<<<<<<<<<<<
%
\def\floatpagefraction{.2}

\index{Macintosh!TeX implementations@\TeX\ implementations}For the 
most part, using \TeX\ on the Macintosh is like using \TeX\ on 
any other system.  Certainly, the \TeX\ documents that you edit are
the same, and the output (on paper) is the same as the output from
any other version of \TeX.

However, because they are immersed in a
consistent graphical environment, Macintosh tools have a substantially different
appearance from their non-graphical counterparts.\footnote{I'm not going
to argue about the relative merits of graphical and non-graphical
environments or particular implementations of graphical environments.
When all is said and done, the Mac {\em is\/} different.  At least today.}

There are four implementations of \TeX\ available for the Macintosh.
Of these four, one is commercial, two are shareware, and one is (mostly)
free.  The following sections present an overview of each implementation,
in alphabetical order.  

\section{CMac\protect\TeX}

The \ixx{\cmactex}{cmactex@\cmactex} package includes the most recent 
versions of \TeX\ and
\MF\ and all of the standard tools.  A port of \dvips\index{dvips!for Macintosh} is also included,
as well as a \ext{DVI} previewer and a \program{\idx{PrintPS}} tool for
printing PostScript files directly to a LaserWriter printer over AppleTalk.

Each utility is a straightforward port of its \Unix\ counterpart.
Command-line options have been replaced by standard
Mac dialog boxes and menus where appropriate.  \TeX\ has been extended to include a
built-in editor, although it is not necessary to use that editor if
you have another favorite.

By design, \cmactex\ is a very modular package.  This makes it easy to
substitute different tools, or different ports of the same tools, where
it is advantageous to do so.  For example, you can use 
\program{\idx{MacGS}}\index{Ghostscript!for Macintosh} (a 
Macintosh version of \program{Ghostscript}) as a
previewer if you like.

Small versions of \TeX, ini\TeX, \MF, and \program{iniMF} are provided
in the free distribution of \cmactex.  Big versions are 
available only in a commercial distribution purchased directly from the
author.  At the time of this writing, 
the commercial distribution is
available on diskettes and via email.
The commercial version also provides
fully automated font generation (see ``\nameref{sec:autofont}'' in
Chapter~\ref{chap:fonts}, {\it \nameref{chap:fonts}}) and faster versions of
\TeX\ and \MF.

The \ixx{configuration files}{configuration files!for Macintosh} used 
by \cmactex\ resemble the environment
variables used by implementations of \TeX\ on other systems.  You can
set up multiple search folders for input files and fonts, for example,
by providing a list of folder names separated by colons.

Table~\ref{tab:mac:cmacdist} summarizes the \cmactex\ version 2.1
distribution available on the CTAN archives (in
\ctandir{systems/mac/cmactex}) as of July, 1993.  The top-level folders and
their contents are presented, not a list of the archive files that
form the distribution.

\begin{xtable}{l|l}
  \caption{Summary of the \protect\cmactex\ Distribution at 
    CTAN\label{tab:mac:cmacdist}}\\
  \bf Folder  & \bf Description \\[2pt]
  \hline
  \tstrut
  \it TeX         & Small versions of \TeX \\
  \it Metafont    & Small versions of \MF \\
  \it view\&print & \ext{DVI} previewer and \program{PrintPS} \\
  \it dvips5516   & Complete distribution of \dvips, version 5.516. \\
  \it Utilities   & All of the standard \texware\ and \mfware\ utilities \\
  \it ams         & Versions of \AMSTeX\ and \AMSLaTeX \\
  \it ams-pk      & \AmS-fonts in \ext{PK} format \\
  \it ams-tfm     & \ext{TFM} files for the \AmS-fonts \\
  \it amsfonts    & \MF\ sources for version 2.0 of the \AmS-fonts \\
  \it tfms        & \ext{TFM} files for the CMR and \LaTeX\ fonts \\
  \it cmpk300     & The CMR fonts in \ext{PK} format\\
  \it lpk300      & The \LaTeX\ extensions to CMR in \ext{PK} format \\[2pt]
  \hline
\end{xtable}
  
The installation instructions for \cmactex\ are easy to understand, but
you will have to configure \cmactex\ before you try to use it.  
Unfortunately, the default configuration files do not reflect the layout
of folders that results directly from unpacking the archive files.

\cmactex\ includes a prebuilt format file for Plain \TeX, but if you 
want to use \LaTeX, you will have to build the format file with ini\TeX\
first.

\section{\protect\directex}

\ixx{\directex}{directex@\directex}%
\index{Macintosh!TeX MPW implementation@\TeX\ MPW implementation} 
is a \ixx{Macintosh Programmer's Workshop (MPW)}{Macintosh!Programmer's Workshop (MPW)} based \TeX\ package.  It is distributed in archive files containing
eight disk images.  You will have to copy each disk image onto a diskette
(using a tool provided) before you can install \directex.

Because I don't have access to a Mac with MPW installed, there is very 
little that I can say about \directex\ at this point.

\section{\protect\oztex}

\ixx{\oztex}{oztex@\oztex} is a complete \TeX\ package that includes an integrated \ext{DVI}
previewer.  \oztex\ can print \TeX\ \ext{DVI} files directly to any printer 
selected by the \program{\idx{Chooser}}.  Because \oztex\ does not include \MF,
you may need to get from some other source \ext{PK} files at a resolution 
appropiate for your
printer.  The standard \oztex\ distribution includes
a complete set of
\ixx{\ext{PK} files}{PK files@\ext{PK} files} for 300dpi and 360dpi printers.

A default configuration file and a selection of specialized
configuration files for different printers and environments are
provided with \oztex.  The distinction between big and small
implementations of \TeX\ has been replaced by configurable memory
limits\index{configurable memory limits!for oztex@for \oztex}.  With enough 
RAM and appropriate configuration, you
should be able to get \oztex\ to process any \TeX\ file you
give it.

\oztex\ includes a simple text editor called \program{\idx{$\Sigma$Edit}},
but you can replace it with any editor you choose.
Table~\ref{tab:mac:oztexdist} summarizes the \oztex\ version 1.5
distribution available on the CTAN archives (in
\ctandir{systems/mac/oztex}) as of July, 1993.  The top-level folders and
their contents are given, not a list of the archive files that
form the distribution.

\begin{xtable}{l|l}
  \caption{Summary of the \protect\oztex\ Distribution at 
    CTAN\label{tab:mac:oztexdist}}\\
  \bf Folder  & \bf Contents \\[2pt]
  \hline
  \tstrut
  \it Configs     & Configuration files \\
  \it TeX-formats & Format files for Plain \TeX\ and \LaTeX\\
  \it TeX-fonts   & \ext{TFM} files for CMR, \LaTeX, and PostScript fonts\\
  \it Help-files  & Online help files\\ 
  \it PS-files    & PostScript sources for \oztex's PostScript built-in driver\\
  \it TeX-docs    & Example \TeX\ files\\
  \it LaTeX-docs  & \LaTeX\ sources for a 26 page User's Guide to \oztex\\
  \it $\Sigma$Edit& A simple text editor desk accessory\\
  \it TeX-inputs  & Input files for Plain \TeX\ and \LaTeX\\
  \it PK-files    & A set of \ext{PK} files for 300dpi and 360dpi printers\\[2pt]
  \hline
\end{xtable}

The \oztex\ \ext{DVI} printer recognizes \verb|\special| commands for
inserting PICT\index{PICT file format}, 
PNTG\index{PNTG file format} (\program{MacPaint}), and 
EPSF\index{EPS (Encapsulated PostScript) file format} images
into your documents.  Provision is also made for including raw PostScript
code if the selected printer is a \idx{PostScript} printer.

\oztex\ is a shareware program.  If you continue to use it after a 
reasonable trial period, you are expected to purchase it.

\section{Textures}

\program{\idx{Textures}} is a commercial implementation of \TeX\ 
from \idx{Blue Sky Research}. It has a number of features that make it unique in the
\TeX\ market. It is supported by a complete 
user's guide and access to telephone and email product support.

\Textures\ supports an interactive preview mode called 
        {\it Lightning} \Textures\index{Lightning Textures}\index{previewing!for Macintosh Textures}.  It is 
        this feature that really sets 
        \Textures\ apart
        from other implementations.  In this mode, changes to your document
        are reflected immediately in the preview window.  In an environment
        with sufficient resources (memory and processing speed), the result
        is striking.  Constructing complex items like tables and mathematical
        formulae is much easier, especially for the \TeX\
        novice, than using the conventional edit, \TeX, preview, debug
        cycle.  Because the log file is also visible, it's easy to see
        when you've written erroneous \TeX\ code.

        Note that {\it Lightning Textures\/} is not really a WYSIWYG
        environment (like \program{Scientific Word}, for example) because
        you still enter regular \TeX\ commands in a purely textual
        fashion.  You get immediate feedback in a different window.

\Textures\ is fast.  Extensive instrumentation and hand-tuning
        of the program has produced an executable that is
        several (maybe many) times faster than other \TeX\ executables
        on similar hardware.

A complete set of Computer Modern Roman fonts is provided 
        in Adobe Type~1 format.  This means that any font can
        be rendered at any size without loss of quality.
PostScript versions of the \AmS-fonts are also available.  The
        fonts can be purchased separately in either Macintosh or
        Adobe \ext{PFB} formats.

\Textures\ is a very integrated environment.  This can hardly be
labelled a disadvantage considering how well it works, but it does
mean that some extra effort is required if you want access to your
document in a less integrated fashion.  Using another editor to
compose your document is possible, but it prevents you from using
{\it Lighting} \Textures.  Starting with version 1.6, the
\Textures\ editor includes a macro programming language, so you can
customize it with features that you find useful.  Blue Sky Research
provides the tools you need to incorporate other fonts into
\TeX\ or extract \TeX\ files (like \ext{DVI} files) that are normally
hidden from view by \Textures.

Figure~\ref{fig:mac:textures} shows an example of a \Textures\ session.  The
preview quality in this image is less than optimal because the Macintosh
that \Textures\ is running on does not have Adobe Type Manager.

\epsimage{fig:mac:textures}{Editing, previewing, and typesetting 
  in Textures}

Because the Computer Modern Roman fonts are included in Adobe Type~1
format, \MF\ is not provided in the \Textures\ package.  Recently, BSR
made their version of \MF\ freely available.
BSR's \MF\ 
was designed to work with \Textures, and creates output files that are
suitable for \Textures, but not necessarily the standard \ext{GF} and
\ext{TFM} files you might expect.  Those files can be obtained elsewhere, of course.

The \Textures\ package includes implementations of many auxiliary \TeX\
programs including \ixx{\BibTeX}{bibtex@\BibTeX}, \program{\idx{MakeIndex}}, 
the \program{\idx{Excalibur}}
spellchecker (see the ``\nameref{sec:Excalibur}'' section later in
this chapter), a \program{\idx{DVITool}}
for importing and exporting \ext{DVI} files, and font tools for importing
and exporting fonts. (Because \Textures\ doesn't use \ext{PK} files directly,
the standard \mfware\ tools are not provided.)  The \Textures\ font tools
are freely available from Blue Sky Research; see the ``\nameref{sec:bsrtools}''
section later in this chapter.  \Textures\ also supports virtual fonts.

\Textures\ includes the \idx{Eplain} and \idx{Midnight} macro
packages in addition to Plain \TeX\ and \LaTeX.  ini\TeX\ is built into
\Textures, so you can make additional format files as described in
Chapter~\ref{chap:macpack}, \textit{\nameref{chap:macpack}}.  Making format files
with \Textures\ requires a Macintosh Plus or other system with at least
1Mb of memory.

The \Textures\ previewer and printing operations understand bitmap or
scalable (EPSF) pictures inserted into your document with
\verb|\special| commands.  Although not visible on the previewer, raw
PostScript can also be inserted for documents printed on PostScript
devices.

\section{Other Tools}

The following sections describe Macintosh versions of other common
tools.  Some of these programs are unique to the Mac, while others
are ports of tools from other systems.

\subsection{Alpha}

\program{\idx{Alpha}} is a sophisticated shareware editor for text files.
\program{Alpha} uses \program{\idx{Tcl}}, an interpreted C-like
language, as a macro programming language for extending the editor.
\LaTeX\ support, which is written in \program{Tcl}, is very complete.  
An example of the \program{Alpha} editor is shown in 
Figure~\ref{fig:mac:alpha}.

\epsimage{fig:mac:alpha}{\protect\program{Alpha} editing the fonts chapter 
  from this book}

\subsection{BBEdit}

\program{\idx{BBEdit}} is another shareware editor.  Like \program{Alpha},
it has a wide range of features including a \LaTeX-aware editing
mode.
An example of the \program{BBEdit} editor is shown in 
Figure~\ref{fig:mac:bbedit}.

\epsimage{fig:mac:bbedit}{\protect\program{BBEdit} editing the fonts chapter 
  from this book}

\newpage
\subsection{BSR Font Tools}
\label{sec:bsrtools}

The \ixx{Macintosh Programmer's Workshop (MPW)}{Macintosh!Programmer's Workshop (MPW)} is
required to use the Blue Sky Research Font Tools.  
These tools were written for \Textures\ users so that they could make
other fonts, like those created with \MF, for example, usable in \Textures.  In
practice however, they create standard Macintosh font resources, so
the resulting fonts can be used by any Mac application.
Table~\ref{tab:mac:bsrtools} describes the tools included in the
\program{\idx{BSR Font Tools}} package.

\begin{xtable}{l|l}
  \caption{Font Tools in the BSR package\label{tab:mac:bsrtools}}\\
  \bf Tool  & \bf Description \\[2pt]
  \hline
  \tstrut
  \it GFtoPK    & Converts \MF\ \ext{GF} files into standard \ext{PK} format \\
  \it PKtoFOND  & Creates Mac FOND resource from \ext{PK} and \ext{TFM} files\\
  \it TFMtoSuit & Creates a font metrics suitcase \\
  \it NFNTcon   & Finds NFNT resource numbering conflicts\\
  \it TFtoPL    & Converts \ext{TFM} files into \ext{PL} files \\
  \it PLtoTF    & Translates \ext{PL} files (back) into \ext{TFM} files\\[2pt]
  \hline
\end{xtable}

\subsection{Excalibur}
\label{sec:Excalibur}

\program{\idx{Excalibur}} is a spellchecker designed to work
with \LaTeX\ documents.
An example of \program{Excalibur} is shown in 
Figure~\ref{fig:mac:Excalibur}.

\epsimage{fig:mac:Excalibur}{\protect\program{Excalibur} spellchecking 
  the fonts chapter from this book}

\subsection{Hyper\protect\BibTeX}

\index{HyperBibTeX@Hyper\BibTeX}This is a hypercard stack 
for maintaining bibliographic databases%
\index{bibliographies!hypercard stack}%
\index{bibliographies!for Macintosh} suitable 
for 
use with \ixx{\BibTeX}{bibtex@\BibTeX}.  \BibTeX\ 
is described in Chapter~\ref{chap:bibtex}, {\it\nameref{chap:bibtex}}.
An example of Hyper\BibTeX\ is shown in 
Figure~\ref{fig:mac:hyperbib}.

\epsimage{fig:mac:hyperbib}{The \protect\program{Hyper\BibTeX} view of 
  a bibliographic database}

\subsection{MacGS}

\program{\idx{MacGS}} is a Macintosh port 
of GNU \program{Ghostscript}\index{Ghostscript!for Macintosh}, described in
Chapter~\ref{chap:preview}, {\it\nameref{chap:preview}}.

\subsection{dvidvi}

\program{\idx{dvidvi}} is a Macintosh port of the \program{dvidvi} utility for 
rearranging pages in a \ext{DVI} file.  

\subsection{MacDVIcopy}

\program{\idx{MacDVIcopy}} is a port of the \program{\idx{DVICOPY}} utility,
which transforms virtual font references in a \ext{DVI}
file into the appropriate non-virtual fonts or commands.  This allows
\dvidriver{}s and previewers that lack support for virtual fonts to
preview documents that use them.

\subsection{Mac\protect\BibTeX}

\idx{Mac\BibTeX} is a port of the standard \BibTeX\ utility for 
accessing bibliographic databases in a document.  \BibTeX\ is described
in Chapter~\ref{chap:bibtex}.

\subsection{MacMakeIndex}

\program{\idx{MacMakeIndex}} is a port of the standard \program{MakeIndex}
for creating sorted, multilevel indexes in a document.
\program{MakeIndex} is described in Chapter~\ref{chap:bibtex}.

\egroup % <<<<<<<<<<<<<<<<<<<<<<<<<<<<<<<<<<<<<<<<<<<<<<<<<<<<<<<<<<<<

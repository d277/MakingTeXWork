\chapter[Non-commercial Environments]{Non-commercial\\Environments}
\RCSID$Id: ch13.tex,v 1.1 2002/08/23 14:58:46 nwalsh Exp $
\label{chap:freetex}

\ifincludechapter\else\endinput\fi

\index{tex@\TeX!non-commercial implementations}This chapter 
offers an overview of several common free and shareware
\TeX\ systems.  If you notice any conflicts between the information in
this chapter and
the documentation that comes with the software, please consider the
documentation to be more accurate and up-to-date.  
I've attempted to provide hints and practical suggestions, but
software installation is largely dependent on your system configuration and
the way you want to use \TeX.

\section{Web2C}
\label{free:web}

\index{Web2C}\index{Web@\Web} The official 
sources for all of the standard \TeX\ programs are distributed
in a format known as \web.\footnote{Hence the spider on the cover, by the
way.}  \web\ is an implementation of a programming style known as ``literate
programming.''  One central tenet of literate programming is that source
code and documentation should be written in parallel.  Literate programming
enforces this tenet by combining the two in a single file.  \web\ source
files contain a mixture of source code and documentation.  

In \web\ files, the documentation is written in \TeX, and the source code
is written in Pascal. (Other literate programming environments use different
documentation and programming languages.)  To print the documentation,
a special preprocessor called \ixx{\program{weave}}{weave} transforms 
the \web\ source into
a \TeX\ document that can be formatted and printed.  Another preprocessor,
called \ixx{\program{tangle}}{tangle}, transforms the \web\ source into a Pascal program
that can be compiled and executed.

In practice, it is far more likely that you have access to a C compiler than
to a Pascal compiler.  This is where \program{Web2C} comes in.  \program{Web2C}
is a special-purpose Pascal-to-C translator that makes it possible to compile
and build the \TeX\ sources in most \Unix\ environments (and probably many
other environments as well).

Karl Berry\index{Berry, Karl} maintains the 
\program{Web2C} distribution.  You can retrieve it
from the CTAN archives in the directory \ctandir{systems/unix/web2c}.

Preparing to build \TeX\ with the \program{Web2C} distribution is very
straightforward.  Unpack the archive files, read the file named \filename{README},
and follow the instructions in the file named \filename{INSTALL}.   Many people
have built \TeX\ using these sources.  If you have difficulty, readers of the 
\ixx{\texttt{Info-TeX} mailing list}{Info-TeX mailing list@\texttt{Info-TeX} mailing list} or the 
\ixx{\texttt{comp.text.tex} newsgroup}{comp.text.tex newsgroup@\texttt{comp.text.tex} newsgroup} will almost certainly be able to help.

In my experience, there are only two parts of the installation that are 
likely to cause any difficulty: compiling \filename{tangleboot.c} and building
the X Windows support in \MF.

Because \program{tangle} is written in \web, there is an obvious
bootstrapping problem (how do you \program{tangle} \filename{tangle.web}
in order to compile \program{tangle}?).  The \program{Web2C}
distribution includes a small C program called \ixx{\filename{tangleboot.c}}{tangleboot.c},
which overcomes this hurdle.  It's possible that getting
\filename{tangleboot.c} to compile on your machine may require some
tinkering.  I had this problem on one machine with one release of
\program{Web2C}.  Subsequent releases of \program{Web2C} seem to have
corrected this problem (in my case, at least).

Compiling X Window support for \MF\ has regularly been a thornier problem.
If you are not comfortable programming in C, I cannot suggest a simple
way to correct these difficulties in the general case.  (However, asking for 
help in the 
appropriate newsgroups is very likely to produce a solution for any 
particular problem.)

The most common errors that I've encountered in building the X Windows support
for \MF\ are conflicting prototypes and type definitions.  My solution to 
these problems, inelegant though it may be, has generally been to ``correct''
or remove the offending declarations from the \MF\ sources. (Note: the X
Window support is written directly in C, not \web, so it is considerably
easier to edit.)  In my experience, this has always been successful, but your
mileage may vary.

\section{\protect\emTeX}

\index{emTeX@\emTeX}\index{OS/2!tex implementations@\TeX\ implementations}
The \emTeX\ distribution 
is a very complete, free distribution of \TeX\
for MS-DOS and OS/2 systems.  \emTeX\ will run on any 80x86 
processor, although some of the larger, faster executables require at 
least a 386.
There is no single right way to
install \TeX, so what follows is only one possible installation.  This
installation uses the default directories and assumes a single-user
environment.  Installing \emTeX\ in a network environment is mentioned
briefly in ``\textit{\nameref{sec:netemtex}},'' later in this chapter.
Incorporating the more recent, beta test versions of \emTeX\ is described
in the section called
\linebreak
\newpage
``\nameref{sec:betaemtex}.''
\emTeX\ is available on CTAN in the directory 
\ctandir{systems/msdos/emtex}.

On the whole, the installation instructions for \emTeX\ are provided in the
distribution.  Begin by reading the English or German versions of the
documentation provided.  Additionally, you should read the \dvidriver\
documentation at least once. (It's quite long, and you may not remember all of
it after a single reading, but at least you'll have a feel for what can be
done.)

Table~\ref{tab:emtexfilesum} summarizes the files in the distribution.
The standard \emTeX\ distribution occupies six high-density floppy disks.
A complete installation requires more than 12 megabytes of storage 
without fonts.  You must also get a set of fonts (available in
font library format for \emTeX) or set up automatic font generation.

{\catcode`\_=\active
\def_{\ttunder}
\begin{xtable}{l|l|r|r}
  \caption{Summary of the \protect\emTeX\ 
    Distribution\label{tab:emtexfilesum}}\\
    &             &        & \multicolumn{1}{l}{\bf Expanded} \\
    \textnormal{\bf File} & \bf Abbreviated Contents & \multicolumn{1}{l}{\bf Size} & \multicolumn{1}{|l}{\bf Size}     \\[2pt]
    \hline
\endfirsthead
  \caption[]{Summary of the \protect\emTeX\ (continued)}\\
    &             &        & \multicolumn{1}{|l}{\bf Expanded} \\
    \textnormal{\bf File} & \bf Abbreviated Contents & \multicolumn{1}{|l}{\bf Size} & \multicolumn{1}{|l}{\bf Size}     \\[2pt]
    \hline
\endhead
    \tstrut
    \it *.eng        & English documentation                  & 125Kb & 125Kb \\
    \it {}*.ger      & German documentation                   & 140Kb & 140Kb \\
    \it {}*.exe      & Delete, remove, pkunzip                & 227Kb & 227Kb \\
    \it tex1.zip     & \TeX\ executables, \ext{TFM}s, inputs  & 368Kb & 645Kb \\
    \it tex2.zip     & Plain formats, more \ext{TFM}s         & 262Kb & 403Kb \\
    \it blatex.zip   & Big \TeX\ \LaTeX\ format               & 227Kb & 494Kb \\
    \it latex1.zip   & Documentation and formats              & 243Kb & 393Kb \\
    \it latex2.zip   & \LaTeX\ styles and \ext{TFM}s          & 233Kb & 705Kb \\
    \it latexdoc.zip & \LaTeX\ style documentation            & 110Kb & 315Kb \\
    \it makeindx.zip & \program{MakeIndex}                    & 52Kb  & 152Kb \\
    \it pictex.zip   & \PiCTeX\ macros                        & 43Kb  & 144Kb \\
    \it texware.zip  & \program{texchk}, \program{texconv}, \program{maketcp}, &  265Kb & 433Kb \\
     & \TeX{}ware                             &       &       \\
    \it dvidrv1.zip  & Driver configuration, batch,           & 119Kb & 370Kb \\
    \it & and setup files                        &       &       \\
    \it dvidrv2.zip  & Driver executables                     & 345Kb & 551Kb \\
    \it dvidrv3.zip  & Preview executables                    & 293Kb & 506Kb \\
    \it texcad.zip   & \TeX{}Cad                              & 118Kb & 236Kb \\
    \it bmf1.zip     & Big \MF                                & 261Kb & 472Kb \\
    \it mf1.zip      & \MF\ executables                       & 245Kb & 398Kb \\
    \it mf2.zip      & \MF\ sources,                          & 337Kb & 613Kb \\
    \it & OS/2 executable                        &       &       \\
    \it mf3.zip      & \MF\ sources                           & 273Kb & 768Kb \\
    \it mfware1.zip  & \ext{GF} tools, \ext{PK} tools         & 320Kb & 504Kb \\
    \it bibtex.zip   & \BibTeX                                & 117Kb & 319Kb \\
    \it bmf2.zip     & OS/2 executable,                       & 266Kb & 564Kb \\
    & Big \MF\ bases                         &       &       \\
    \it btex1.zip    & Big \TeX\ executables                  & 259Kb & 447Kb \\
    \it btex2.zip    & Big \TeX\ Plain format                 & 269Kb & 533Kb \\
    \it emsy.zip     & Special fonts from \emTeX              & 8Kb   & 11Kb \\
    \it mfware2.zip  & \program{MFjob}, some more inputs      & 138Kb & 230Kb \\
    \it misc_mf.zip  & Miscellaneous inputs                   & 36Kb  & 130Kb \\
    \it pkedit.zip   & \program{PKEdit}                       & 51Kb  & 97Kb\\
    \it dvidrvma.zip & \dvidriver\ manual                     & 178Kb  & 494Kb \\
    \it web.zip      & \web\ tools                            & 128Kb  & 252Kb \\
    \it g*.zip       & German versions of \TeX                & 800Kb  & 1614Kb \\[2pt]
    \hline
\end{xtable}
}

For simplicity, the discussion that follows assumes that you have enough
disk space to install the entire 12Mb distribution.  If you can't get that
much space (even temporarily), you can perform the steps discussed in the
section ``\nameref{sec:emtex:smaller}'' as you install.

\subsection{Where to Start}

Begin by reading the English (\filename{*.eng}) or German (\filename{*.ger})
documentation files on the first disk.  Where there is disagreement
between this book and those files, believe the files---especially
if a new version of \emTeX\ has been released.

\subsection{Unpacking the Archives}

The \emTeX\index{emTeX@\emTeX!unpacking archives} distribution 
is a collection of \ext{ZIP} files spread over
six disks.  You must install these files into your hard drive.  If you have
more than one hard disk, you may install \emTeX\ on any drive you choose.
Select one with more than 12Mb
of free space, at first.  You can copy the
files onto another drive after you've deleted things that you don't need.

All of the \emTeX\ files are installed in a subdirectory tree rooted
at \filename{emtex}.  If you absolutely must use a different
subdirectory tree, unpack each archive file into a temporary directory
and then copy the files into the subtree you want to use.  You will have
to do much more extensive customization of the batch files and
configuration files before \emTeX\ will work.  That sort of 
customization isn't described here in detail.

Make the root directory of the hard disk that you wish to install onto
the current directory, and then use the \ixx{\program{pkunzip}}{pkunzip} %
% eberhard says this isn't true
%\footnote{The
%version of \program{pkunzip} distributed with \emTeX\ is quite old.  It
%will not, for example, unpack the beta-test archives which were created
%with a newer version of \program{pkzip}.  If you have access to a more
%recent version of an unzipping program, use it.} 
program
distributed on the first disk, or any later version of
\program{pkunzip}, to unpack each of the archive files.  You can also
use the \ixx{\program{Info-Zip}}{Info-Zip} version of \program{unzip} (version 5.0 or
later) to unpack the archives.  
If you do not want the German versions of the \TeX\ macro files, do
not bother to unpack the \filename{lkurz.zip} or \filename{g*.zip}
files on disk six.

After you have unpacked the files, you will have a directory tree rooted
at \filename{\bs EMTEX} containing about 12Mb of files and the following additional
directories: \filename{bibinput}, \filename{bmfbases}, \filename{btexfmts}, \filename{doc},
\filename{mfbases}, \filename{mfinput}, \filename{MFjob}, \filename{remove}, \filename{texfmts},
\filename{texinput}, and \filename{tfm}.

\newpage
\subsection{Setting Up the Environment}
\index{emTeX@\emTeX!setting environment}

A lot of information about where and how things are installed for \emTeX\ is
stored in the \glo{environment}.  If you are unfamiliar with the DOS or OS/2
environment, consult your operating system reference for more details.  In
brief, the environment\index{environment!for emTeX@for \emTeX} is a collection of named
strings.  For example, your \envvar{PATH} is in the environment.

The \emTeX\ distribution includes a batch file called
\filename{set-tex}\index{set-tex batch file}, which 
sets all of the environment variables that
\emTeX\ uses.  Edit this file and
change all references to C: to the drive where you
installed \emTeX.  If you installed \emTeX\ on drive C:, you needn't
change anything.  The section ``\nameref{sec:emtex:basiccust}''
later in this chapter
describes the variables used by \emTeX\ in greater detail.

If you are using MS-DOS and haven't already made the environment
larger than its default size, you may need to do so in order to use
\emTeX.  Insert the following command, or something like it, in your
\filename{CONFIG.SYS} file:

\begin{shortexample}
SHELL=C:\COMMAND.COM /P /E:1024
\end{shortexample}

This is very system-dependent. You will have to use different
settings if you keep \filename{COMMAND.COM} in a subdirectory or if you
use a different command processor.

\subsection{Testing the Installation}
\label{sec:emtex:testinstall}

\index{emTeX@\emTeX!testing installation}After you 
have set up the environment, make the \filename{emtex} directory the
current directory and test \TeX\ by typing:

\begin{exindent}
\begin{alltt}
C:\ttbackslash{}EMTEX>{\bf tex}
This is emTeX, Version 3.0 [3a] (no format preloaded)
**{\bf \ttbackslash{}relax}

*{\bf This is a test}

*{\bf \ttbackslash{}bye}
[1]
Output written on texput.dvi (1 page, 224 bytes).
Transcript written on texput.log.
\end{alltt}
\end{exindent}

This will produce a small \ext{DVI} file 
called \filename{texput.dvi}\index{texput.dvi file}.  Keep
this file; you can use it in a few minutes to test the previewer.

To test \LaTeX, enter the command:

\begin{ttindent}
\$ \textbf{tex &lplain testpage}
\end{ttindent}

See the section ``\nameref{sec:runtex:cmdcautions}'' 
in Chapter~\ref{chap:running}, {\it \nameref{chap:running}\/}, 
for an explanation of some potential problems with
using the ampersand (\&) character on the command line.

\newpage
\subsection{Installing Fonts}

\index{emTeX@\emTeX!installing fonts}\index{fonts!installing for emTeX@installing for \emTeX}There 
are several sets of fonts available for \emTeX.  You should select
the set that is appropriate for your printer.   They are summarized in
Table~\ref{tab:emtex:fontlibs}.  The CTAN directory
\ctandir{systems/msdos/emtex}
contains the libraries or disks of libraries.

{\catcode`\_=\active
\def_{\ttunder}
\begin{xtable}{l|l|p{2.9in}}
  \caption{Fonts Libraries Available for \protect\emTeX\ on 
    CTAN\label{tab:emtex:fontlibs}}\\
  \bf CTAN Directory & \bf Resolution & \bf Description \\[2pt]
  \hline
  \tstrut
  \it fx_\,fonts      & 240x216 & Epson FX series printers, Tandy DMP-130 printer\\
  \it fx_med_\,fonts  & 240x144 & Epson FX series (for Windows drivers) \\
  \it ito_\,fonts     & 160x144 & C. ITOH 8510A printers \\
  \it lj_\,fonts      & 300x300 & 300dpi Laser printers \\
  \it p6h_\,fonts     & 360x360 & Epson LQ series, NEC P6 \& P7, IBM Proprinter,
                              and Panasonic KX-P1124 printers \\
  \it p6m_\,fonts     & 360x180 & Epson LQ series, NEC P6 \& P7, IBM Proprinter,
                              and Panasonic KX-P1124 printers \\
  \it p6l_\,fonts     & 180x180 & Epson LQ series, NEC P6 \& P7, IBM Proprinter
                              and Panasonic KX-P1124 printers \\
  \it psfonts       & 300x300 & Standard 35 PostScript fonts \\[2pt]
  \hline
\end{xtable}
}

Not every font set is appropriate for every model of the series mentioned.
If your printer isn't mentioned, use the \ixx{\program{dvidot}}{dvidot} program 
included with
\emTeX\ to generate an appropriate configuration file.  Select
the font set with the same resolution as your printer.

The PostScript fonts included in the \ixx{\filename{psfonts}}{psfonts} libraries 
are suited
primarily for previewing.  The quality of many of the fonts is substandard.
(They were generated from unhinted outlines with \ixx{\program{Ghostscript}}{Ghostscript}.)
A font substitution file is provided for 300dpi, 180dpi, and 360dpi printers.

Once you have retrieved the font libraries that you need, unpack them
(if they are in \ext{ZIP} archives) and place the \ext{FLI} files in
the \filename{\bs TEXFONTS} directory of the drive where you installed
\emTeX.  The amount of disk space required depends on the font set
that you install.  It can be more than 5.5 megabytes.  The section called
``\nameref{sec:emtex:autofont},'' later in this chapter, provides an overview 
of automatic
font generation under \emTeX.  This can save a considerable amount of
disk space.

\subsubsection{Testing font installation}

After you have installed a font set, you should be able to preview and
print \TeX\ documents.  If you followed the instructions in
``\nameref{sec:emtex:testinstall},'' earlier in this chapter, you have
two \ext{DVI} files in your \filename{\bs EMTEX} directory.

First, figure out which configuration file is appropriate
for your installation.  The table near the top of \filename{dvidrv.doc}
in the \emTeX\ documentation directory should help you determine which
one is appropriate.

The default configuration file for previewing is the LaserJet configuration.
If you installed the LaserJet fonts (\filename{lj\ttunder*.fli}), you do not have to
change the previewer.  Otherwise, edit the preview batch file (\filename{v.bat}
for MS-DOS systems and \filename{v.cmd} for OS/2 systems) and change the 
configuration file.  The preview batch file looks like this:

\begin{shortexample}
@echo off
dviscr @lj.cnf /ocr=1 /fl=-1 %1 %2 %3 %4 %5 %6 %7 %8 %9
\end{shortexample}

The configuration file name is \filename{lj.cnf}.  If, for example, you
are using the Epson FX series printer fonts, you would change this to
\filename{fx.cnf}:

\begin{shortexample}
@echo off
dviscr @fx.cnf /ocr=1 /fl=-1 %1 %2 %3 %4 %5 %6 %7 %8 %9
\end{shortexample}

Now type \textbf{v texput} to preview the test file.  Under OS/2, you 
should do this from a full-screen session, not a windowed session.  You
can type \textbf{v testpage} to preview the \LaTeX\ test page, if you
created it.

\subsection{Making \protect\emTeX\ Smaller}
\label{sec:emtex:smaller}

\index{emTeX@\emTeX!making smaller}The complete 
distribution of \emTeX\ is much larger than the
practical requirements.  This section discusses the pros and cons of
deleting (or at least archiving) parts of the \emTeX\ distribution.

\subsubsection{Deleting extra documentation}

The \filename{\bs emtex\bs doc} directory contains two complete sets of 
documentation, one in English and the other in German.  Delete the set that
you do not need.  This will save about 500Kb.

\subsubsection{Deleting extra executables}

There are six versions of the \TeX\ and \MF\ executables in
the \emTeX\ distribution.  You can probably delete at least four from
each.  Follow these guidelines:

\begin{itemize}
  \item If you have a 80286 or larger processor, delete the non-286
        versions: \filename{btex.exe}, \filename{tex.exe}, \filename{bmf.exe},
        and \filename{mf.exe}.
  \item If you do not have OS/2, delete the OS/2 versions: 
        \filename{btexp.exe}, \filename{texp.exe}, \filename{bmfp.exe},
        and \filename{mfp.exe}.  Conversely, if you have OS/2 and you are
        satisfied that you will never want to run \TeX\ or \MF\ under
        DOS or Windows, delete the non-OS/2 versions:
        \filename{btex286.exe}, \filename{tex286.exe}, \filename{bmf286.exe},
        \filename{mf286.exe}, and the non-286 versions listed above.  
        If you need both, you'll have to keep
        both around, of course.
  \item Finally, if disk space is at a premium and/or you have a fast 
        computer, you may choose to delete the small versions of \TeX\
        and \MF.  The small versions are faster than the big versions,
        but large or complex documents and fonts may require the big
        versions, so don't delete them.  If you want to delete the small
        versions, delete \filename{tex.exe}, \filename{tex286.exe},
        \filename{mf.exe}, \filename{mf286.exe}, \filename{texp.exe}, and 
        \filename{mfp.exe}.  In this case you must keep at least one of 
        \filename{btex.exe}, \filename{btex286.exe}, and \filename{btexp.exe}
        depending on your environment.  Similarly, you must keep at least one
        of the big \MF{}s.  If you've decided
        to delete the small versions, you can also delete two directories:
        \filename{emtex{\bs}mfbases} and \filename{emtex{\bs}texfmts}.  These
        contain the format files and base files for the small versions
        of \TeX\ and \MF.
\end{itemize}

Suppose that you have a 80386 computer running MS-DOS (not OS/2) and 
you are willing to sacrifice the speed of the small versions of \TeX\
and \MF.  You would keep the \filename{btex286.exe} and
\filename{bmf286.exe} executables and delete the rest plus the
formats and bases that you don't need, saving 2704Kb.  You can save
an equivalent amount of space by deleting the big 80286 versions and
keeping the big non-286 versions, or by deleting all of the non-OS/2
versions, depending on your environment.

Table~\ref{tab:emtexotherex} summarizes the other executables in the
\emTeX\ distribution.  This will help you decide which programs you
need to keep.  Programs that you don't feel you need should probably
be stored in archive files, possibly on diskettes, in case you change
your mind.  

\begin{xtable}{l|r|l}
  \caption{Other \protect\emTeX\ Executables\label{tab:emtexotherex}}\\
  \bf Program & \multicolumn{1}{|l|}{\bf Size} & \bf Description\\[2pt]
  \hline
\endfirsthead
  \caption[]{Other \protect\emTeX\ Executables (continued)}\\
  \bf Program & \multicolumn{1}{|l|}{\bf Size} & \bf Description\\[2pt]
  \hline
\endhead
  \tstrut
  \it ask.exe      & 6Kb   & Prompts user in batch file, only used by gh.bat\\
  \it bibtex.exe   & 94Kb  & \BibTeX \\
  \it chtopx.exe   & 39Kb  & Obsolete: converts \ext{CHR} files to \ext{PXL} 
                             format \\
  \it pxtoch.exe   & 40Kb  & Obsolete: converts \ext{PXL} files to \ext{CHR}
                             format \\
  \it pxtopk.exe   & 49Kb  & Obsolete: converts \ext{PXL} files to \ext{PK}
                             format \\
  \it dvidot.exe   & 129Kb & Dot matrix printer driver \\
  \it dvihplj.exe  & 134Kb & HP LaserJet printer driver \\
  \it makedot.exe  & 54Kb  & Updates dot matrix printer parameters \\
  \it dvimsp.exe   & 128Kb & Converts \ext{DVI} files into graphic images \\
  \it fontlib.exe  & 67Kb  & Font library management \\
  \it maketcp.exe  & 28Kb  & Change the \TeX\ code page \\
  \it MFjob.exe    & 51Kb  & \MF\ driver\\
  \it pcltomsp.exe & 56Kb  & Converts HP LaserJet graphics into \ext{MSP} or
                             \ext{PCX} format \\
  \it pkedit.exe   & 94Kb  & Edits \ext{PK} files \\
  \it pktopx.exe   & 55Kb  & Converts \ext{PK} files to \ext{PXL} format\\
  \it texcad.exe   & 79Kb  & Interactive \LaTeX\ drawing tool \\
  \it texchk.exe   & 49Kb  & Syntax checker for \LaTeX\\
  \it texconv.exe  & 27Kb  & Converts input files (see ``code pages'') \\
  \it dviscr.exe   & 215Kb & Previewer \\
  \it dviscrs.exe  & 164Kb & Previewer for small-memory configurations \\
  \it dvivik.exe   & 142Kb & Previewer for Viking I displays \\
  \it dvitype.exe  & 72Kb  & Texware: shows \ext{DVI} file\\
  \it pltotf.exe   & 74Kb  & Texware: converts \ext{PL} files to \ext{TFM} 
                             format \\
  \it tftopl.exe   & 48Kb  & Texware: converts \ext{TFM} files to \ext{PL}
                             format \\
  \it gftodvi.exe  & 69Kb  & Mfware: makes proof sheets for a font \\
  \it gftopk.exe   & 53Kb  & Mfware: converts \ext{GF} files to \ext{PK} 
                             format\\
  \it gftopxl.exe  & 59Kb  & Mfware: converts \ext{GF} files to \ext{PXL} 
                             format\\
  \it gftype.exe   & 61Kb  & Mfware: shows \ext{GF} file \\
  \it mft.exe      & 55Kb  & Mfware: pretty-prints \ext{MF} files\\
  \it pktogf.exe   & 47Kb  & Mfware: converts \ext{PK} files to \ext{GF} 
                             format\\
  \it pktype.exe   & 45Kb  & Mfware: shows \ext{PK} file \\
  \it makeindx.exe & 102Kb & Builds indices for documents \\
  \it pooltype.exe & 31Kb  & Web: shows contents of pool files \\
  \it tangle.exe   & 52Kb  & Web: creates compilable file \\
  \it weave.exe    & 72Kb  & Web: creates documentation file \\
  \it vftovp.exe   & 69Kb  & Virtual fonts: converts \ext{VF} files to 
                             \ext{VPL} format \\
  \it vptovf.exe   & 73Kb  & Virtual fonts: converts \ext{VPL} files to
                             \ext{VF} format \\[2pt]
  \hline
\end{xtable}

Deciding which programs to remove from the \emTeX\ directory depends, to a
large extent, on how you use \TeX.  If you haven't been using \TeX\ for
very long, try to keep as many things as you can until you get a better
feel for what you'll be using.  Save copies of all the programs you delete
on floppy disks so that they are easy to get back.  Here are some
suggestions:

\begin{itemize}
  \item Delete the previewer and printer drivers that you don't need.
  \item Delete obsolete programs unless you know you need them.
  \item Delete \web\ programs if you'll never be compiling \TeX\ yourself
        or reading \TeX\ program source code.
  \item Delete \program{Mfware} programs if you won't be doing a lot of work with
        fonts.  Note: keep \program{gftopk.exe} if you want to use
        automatic font generation.
  \item Delete \program{Texware} and \program{Virtual} font programs if you won't be working
        with virtual fonts. (You can probably delete \program{dvitype.exe},
        in any case).
  \item Delete \program{bibtex.exe} if you won't be working with 
        bibliographic databases.
  \item Delete \program{makeindx.exe} if you won't be working with indexes.
  \item Delete \program{ask.exe} if you won't be using \program{gh.bat}.
\end{itemize}

Next, figure out what batch files you'll be using and delete (or archive)
all the others.  If you aren't using OS/2, this includes all of the
\filename{*.cmd} files.

\subsubsection{Deleting extra printer drivers}

Read about the \ixx{\program{dvidot}}{dvidot} program for printing to a dot matrix
printer, if you have one.  When you figure out which of the \ext{DOT}
parameter files you need, delete (or archive) the rest.  Follow these
guidelines:

\begin{itemize}
\item If you don't have a Viking I display, delete \program{dvivik.exe}.

\item If you don't have a laser printer, delete \program{dvihplj.exe}.

\item Delete \program{dviscrs.exe} if you have enough memory to use 
      \program{dviscr}
      instead.  If you are always going to use a Windows or OS/2 previewer,
      you can delete \program{dviscr.exe} as well.
\end{itemize}

\subsection{Basic Customization}
\label{sec:emtex:basiccust}

\index{emTeX@\emTeX!customizing}There are 
two ways to customize \emTeX.  One way is to customize the
environment variables that \emTeX\ uses; the other is to customize the
configuration files that \emTeX\ uses for previewing and printing
(configuration files have no effect on \TeX\ itself).

\subsubsection{Environment variables}

There are thirteen environment variables\index{environment variables!for emTeX@for \emTeX}\index{variables!environment!for emTeX@for \emTeX} that 
\emTeX\ uses.  For
simplicity in the following discussion, assume that \emTeX\ is installed
on drive C:.  If it is not, change {\em every\/} reference to C:
to the drive actually in use.

\begin{iplist}[\textnormal]{.25in}
  \ipitem[\ixx{\envvar{EMTEXDRV}}{EMTEXDRV (environment variable)@\envvar{EMTEXDRV} (environment variable)} (Not set by default)]

  Special note: this variable will become obsolete with the next
  release of \emTeX.

  This variable tells \emTeX\ what drive to use when searching for
  default paths.  Although I don't recommend it, if you use entirely
  standard paths for your files, you can rely on the compiled-in
  defaults in the \emTeX\ executables.\footnote{One of the reasons I
  don't recommend it is that it isn't easy to determine precisely what
  the compiled-in defaults are, and they may change without notice.}
  The \envvar{EMTEXDRV} environment variable should be set to the single
  letter of the drive on which these paths reside.  For example, if
  you installed \emTeX\ on drive C: (in completely standard directories),
  you could set \texttt{\envvar{EMTEXDRV}=C}.

  \ipitem[\ixx{\envvar{EMTEXDIR}}{EMTEXDIR (environment variable)@\envvar{EMTEXDIR} (environment variable)} (Not set by default)]

  Special note: this is a new variable; it will become available with
  the next release of \emTeX.  \envvar{EMTEXDIR} replaces \envvar{EMTEXDRV}.

  This variable tells \emTeX\ what path to use when searching for
  default paths.
  The \envvar{EMTEXDIR} environment variable should be set to the root
  path where \emTeX\ is installed (for example, \verb|C:\EMTEX|).

  \ipitem[\ixx{\envvar{TEXINPUT}}{TEXINPUT (environment variable)@\envvar{TEXINPUT} (environment variable)}]

  Defines the directories that \emTeX\ uses when searching for \TeX\
  input files.  You would typically leave \filename{C:{\bs}EMTEX{\bs}TEXINPUT}
  in this path and extend it with directories of your own in place
  of \filename{C:{\bs}MYTEX}.

  \ipitem[\ixx{\envvar{TEXFMT}}{TEXFMT (environment variable)@\envvar{TEXFMT} (environment variable)}]

  Specifies where small versions of \TeX\ look for format files.  Typically
  you would leave this set as it is and add new format files to that directory.
  You do not have to set this variable if you are not using the small
  versions of \TeX.

\newpage
  \ipitem[\ixx{\envvar{BTEXFMT}}{BTEXFMT (environment variable)@\envvar{BTEXFMT} (environment variable)}]

  Specifies where large versions of \TeX\ look for format files.  You will
  typically leave this set as it is and add new format files to the
  directory specified.  If you do not use large versions of \TeX, you need
  not set this variable.  

  \ipitem[\ixx{\envvar{TEXTFM}}{TEXTFM (environment variable)@\envvar{TEXTFM} (environment variable)}]

  \TeX\ uses the directories stored in this variable when searching for
  \ext{TFM} files for fonts used in your documents.  If you install new
  fonts, you should put the \ext{TFM} files for these fonts in their
  own directory and add the name of that directory to the \envvar{TEXTFM}
  path.

  \ipitem[\ixx{\envvar{MFINPUT}}{MFINPUT (environment variable)@\envvar{MFINPUT} (environment variable)}]

  The \MF\ program looks for input files in this path.  It is analogous
  to the \envvar{TEXINPUT} variable.  If you add new \MF\ fonts to your
  system, for example, the \AmS\ fonts, place them in a different
  directory and add that directory to this path.  If you do not use \MF,
  you do not need this variable.

  \ipitem[\ixx{\envvar{MFBAS}}{MFBAS (environment variable)@\envvar{MFBAS} (environment variable)}]

  Analogous to \envvar{TEXFMT}.  Specifies where small versions of
  \MF\ look for base files.  If you do not use \MF, you do not need
  this variable.

  \ipitem[\ixx{\envvar{BMFBAS}}{BMFBAS (environment variable)@\envvar{BMFBAS} (environment variable)}]

  Analogous to \envvar{BTEXFMT}. Specifies where large versions of
  \MF\ look for base files.  If you do not use \MF, you do not need
  this variable.

  \ipitem[\ixx{\envvar{MFJOB}}{MFJOB (environment variable)@\envvar{MFJOB} (environment variable)}]

  Specifies where \program{MFjob} looks for input files.  If you are not
  using \MF\ or \program{MFjob} for font generation, you do not need this
  variable.

  \ipitem[\ixx{\envvar{BIBINPUT}}{BIBINPUT (environment variable)@\envvar{BIBINPUT} (environment variable)}]

  Specifies where \program{bibtex} searches for bibliography database
  files (\ext{BIB} files).  Clearly, if you're not using \BibTeX, you
  don't need \envvar{BIBINPUT}.

  \ipitem[\ixx{\envvar{DVIDRVINPUT}}{DVIDRVINPUT (environment variable)@\envvar{DVIDRVINPUT} (environment variable)}]

  This tells the \emTeX\ \dvidriver{}s where to search for \ext{DVI}
  files.  In practice, I simply set this to ``\verb|.|'' so that only
  the current directory is used.  If you find it convenient to search
  for \ext{DVI} files, by all means set it to an appropriate path.
  The default path includes \filename{C:{\bs}EMTEX{\bs}DOC} so that it is easy
  to preview or print the local guide.

  \ipitem[\ixx{\envvar{DVIDRVFONTS}}{DVIDRVFONTS (environment variable)@\envvar{DVIDRVFONTS} (environment variable)}]

  This tells the \emTeX\ \dvidriver{}s where to search for fonts.  The
  \filename{C:{\bs}TEXFONTS} directory is the default location for font
  libraries.  If other fonts are installed, it's natural to include them
  in this path.  For example, preview versions of the PostScript fonts
  are now available from CTAN in font library format.  If you install
  those libraries in \filename{C:{\bs}PSFONTS}, you should change the
  \envvar{DVIDRVFONTS} environment variable to include both paths.

  \ipitem[\ixx{\envvar{DVIDRVGRAPH}}{DVIDRVGRAPH (environment variable)@\envvar{DVIDRVGRAPH} (environment variable)}]

  The \emTeX\ \dvidriver{}s can incorporate \ext{PCX} and \ext{MSP} graphics
  with a \verb|\special| command.  The \envvar{DVIDRVGRAPH} environment
  variable tells the drivers where to look for the graphic files.  The
  \verb|$r| portion of the path is replaced by the resolution of the
  image required.  For example, I set this to \filename{C:{\bs}TEXGRAPH{\bs}GR\/\$r}
  and put 300dpi graphic images in \filename{C:{\bs}TEXGRAPH{\bs}GR300}, 420dpi
  images in \filename{C:{\bs}TEXGRAPH{\bs}GR420}, etc.  The default value of
  this variable is set so that the local guide will print properly.
\end{iplist}

\subsubsection{Setting up configuration files}

\index{configuration files!emTeX@\emTeX}Configuration file 
options are described completely in the \dvidriver\
documentation that accompanies \emTeX.  Read the English or German
versions of \filename{dvidrv.doc} carefully.  The ``out of the box''
configuration is sufficient for most uses, but specifying a new configuration
file is a convenient way to alter the default size of the page in preview
mode or set up \emTeX\ to print pages in ``two-up'' or ``four-up'' variations
(for printing booklets, for example).

\subsection{Installing the Beta Test Versions of \protect\emTeX}
\label{sec:betaemtex}

It has been several years since the standard distribution of \emTeX\ was
assembled.  In that time, several other releases have been made.  These
are beta test releases, but they are as rock-solid as the standard
distribution.  Personally, I encourage you to install the beta test versions,
especially because they offer features not found in the standard release.

The beta test versions are stored in a series of \ext{ZIP} files at CTAN
in the directory \ctandir{systems/msdos/emtex/betatest}.
Table~\ref{tab:emtex:betasummary} summarizes the files in the beta test
distribution.\footnote{As this chapter is being 
written, Eberhard Mattes\index{Mattes, Eberhard} is 
working on a new version of \emTeX.  This new version will, at some 
point, replace the existing beta test files, making this section somewhat
out of date.  Use the {\it READ.ME} files in the actual distribution
as the definitive source of information.}

{\catcode`\_=\active
\def_{\ttunder}
\begin{xtable}{l|l}
  \caption{The Beta Test Files for \protect\emTeX\label{tab:emtex:betasummary}}\\
  \bf File     & \bf Contents \\[2pt]
  \hline
  \tstrut
  \it btexb8.zip   & Big \TeX\ executables \\
  \it dvidrv_1.zip & New drivers and support files \\
  \it dvispell.zip & \program{dvispell} \\
  \it maketcp.zip  & \program{maketcp} \\
  \it mfb1.zip     & \MF\ executables \\
  \it MFjob11l.zip & \program{MFjob} \\
  \it mfpm.zip     & OS/2 support for \MF\ preview\\
  \it pkeditpm.zip & OS/2 PM version of \program{PKEdit}\\
  \it tex386b8.zip & 386 version of \TeX\\
  \it texb5.zip    & Small \TeX\ executables \\[2pt]
  \hline
\end{xtable}
}

To install the beta test versions, simply unpack the archives into your
\filename{\bs emtex} directory, replacing any existing files with the new versions.
After you have installed the beta test versions, you can repeat the 
space-saving steps suggested earlier in this chapter.

What are the advantages of the beta test versions?

\begin{itemize}
  \item Automatic font generation is supported.  This means that you do 
        not need large font library files.
  \item The beta test versions are \TeX\ version 3.141; the standard 
        distribution is the slightly older version 3.0.
  \item The beta test distribution includes a 386-specific version of \TeX,
        which is much faster than other versions.
  \item For OS/2 users, a Presentation Manager \ext{DVI} previewer is
        provided.
\end{itemize}

\subsection{Running \protect\emTeX\ 386 in Windows}

\index{emTeX@\emTeX!386 processor}\index{emTeX@\emTeX!for Windows}The 
386-specific version of \TeX\ provided by the beta test distributions
of \emTeX\ has some compelling advantages over the other \TeX\ executables:
it's a big \TeX\ and it's fast.  Unfortunately, it does not
run in Windows (or some other MS-DOS environments).

The problem is memory management.  In order to be large and fast, the
\ixx{\filename{tex386}}{tex386} executable relies on some advanced features of the
386 (and higher) processors. These features are provided under MS-DOS
by a ``DOS extender.''  The DOS extender\index{MS-DOS!extender for emTeX@extender for \emTeX} allows 
MS-DOS applications to
use ``protected mode,'' where they can address more than 640Kb of
memory.  There are several competing standards for protected mode
memory management.  \ixx{\filename{emx}}{emx}, the extender implemented in
\emTeX, is not compatible with the DPMI standard used by Microsoft
Windows and some other MS-DOS extenders (for example, MS-DOS
sessions under OS/2).

Luckily, there is a freely available MS-DOS extender, which is
compatible with both DPMI and \emTeX, the RSX extender.  Starting with
version beta-11 of \emTeX, the\index{RSX extender} is very easy to
use.\footnote{Earlier versions of \filename{tex386.exe} had the
\filename{emx} extender ``bound'' into the executable, which made the
process more complicated.  The \filename{emx} extender had to be
unbound before the new one could be used.}

There are two versions of the extender: \filename{rsx}\index{RSX extender}, for use in
MS-DOS with \idx{DPMI extenders}, and \filename{rsxwin}\index{rsxwin extenders}, for use in
Windows.  Both versions are available on the CTAN archives in the
directory \filename{support/msdos/dpmi/rsx}.

To use the RSX extender in MS-DOS with \filename{tex386}
version beta-11 or later, simply put \filename{rsx.exe} somewhere on
your \texttt{PATH}.  When \emTeX\ discovers a DPMI extender, it will
use \filename{rsx} instead of \filename{emx}.

\newpage
To use the RSX extender in Windows, you must modify the command line
that is executed to run \filename{tex386}.  Where you currently use:

\begin{ttindent}
\textit{d:{\bs}path1\bs}tex.exe \textit{options}
\end{ttindent}

you must now use:

\begin{ttindent}
\textit{d:{\bs}path2\bs}rsxwin.exe -e \textit{d:\bs path1\bs}tex386.exe \textit{options}
\end{ttindent}

The \filename{rsxwin} extender will set up an extender for \emTeX\ and then
run \TeX.

\subsection{Automatic Font Generation with \protect\emTeX}
\label{sec:emtex:autofont}

\begin{note}{NOTE}
Automatic font generation\index{emTeX@\emTeX!automatic  font generation}\index{fonts!automatic generation!with emTeX@with \emTeX} is 
not supported by the standard distribution
of \emTeX.  You must install the beta test versions to use
automatic font generation.
\end{note}

The new versions of the \emTeX\ \dvidriver{}s include a program called
\ixx{\program{dvidrv}}{dvidrv}, which handles automatic font generation.  The
\program{dvidrv} program runs the driver and if the driver indicates
that fonts are missing, gives you the opportunity to use font substitution
or build new fonts.   The \program{MFjob} program is used to build the
fonts.  \program{MFjob} runs the \MF\ and \program{GFtoPK} to produce
and install the necessary fonts.  \program{MFjob} installs the \ext{PK} files
into the first directory on the font path.

The \program{dvidrv}
program can be replaced by a batch file that can select an
appropriate font-rendering program (something other than \MF, for example)
to build the necessary fonts.  This feature is only available in the OS/2 versions of \emTeX\
releases 1.4t and later.
The batch file in Example~\ref{ex:dvidxx} in Appendix~\ref{app:examples},
\textit{\nameref{app:examples}}, is a replacement for \program{dvidrv}.
It uses \ixx{\program{ps2pk}}{ps2pk} to make \ext{PK} versions of PostScript
Type~1 fonts if an appropriate \ext{PFB} file can be found.  Otherwise, it
calls \program{MFjob} in the same way that \program{dvidrv} does.

\subsection{Installing \protect\emTeX\ on a Network}
\label{sec:netemtex}

Ordinarily, \emTeX\ does not require write access to any of its data
files, so installation on a network\index{emTeX@\emTeX!installing  on a network} is not a problem.  The location
of temporary files that must be writable
is controlled with environment variables.  If automatic font generation is 
being used, there are two applicable cautions.

First, if you use the batch file approach to automatic font generation
in order to support different rendering software, be aware that your batch
file must handle the fact that multiple files may be written to the same
directory from different users.  Don't let filenames collide.

Second, it is ideal if you can provide a single world-writable location 
for automatically generated fonts.  If multiple users are working
with similar documents, each will not need private
copies of the fonts they require.  If your networking software tracks
the time of last file access, you can determine which files to delete
from this global font area on a regular basis.

\section{texas}

\ixx{\program{texas}}{texas} is a 32-bit 
MS-DOS version of \TeX.  It is a big \TeX.
Unlike \program{tex386}, which comes with \emTeX, the \program{texas}
executable uses a royalty-free, commercial DOS extender.  The
advantage of this extender is that it can run under Windows and other
protected-mode environments such as DESQview.

Before you install \program{texas}, you should get a complete \TeX\ system
from some other location (\emTeX\ or \gTeX, for example).  The
\program{texas} distribution does not include anything other than
the \TeX\ executable and the MS-DOS extender---no format files, no input files,
nothing.

The \textit{-i} switch 
runs \program{texas} in \ixx{ini\TeX}{initex@ini\TeX} mode.
Using the instructions from Chapter~\ref{chap:macpack},
{\it \nameref{chap:macpack}}, you should be able to build new format
files for use with \program{texas} if you have another \TeX\
distribution to work from.

\program{texas} uses the following environment 
\index{environment variables!for texas}\index{variables!environment!for texas} variables: 

\begin{iplist}[\textnormal]{.25in}
  \ipitem[\envvar{TEXINPUTS}]

  \index{TEXINPUT (environment variable)@\envvar{TEXINPUT} (environment variable)!for emTeX@for \emTeX}Defines the directories 
  that \program{texas} uses when searching for \TeX\
  input files.  You would typically leave the defaults in this path
  and extend it with directories of your own in place of \filename{C:{\bs}MYTEX}.

  \ipitem[\ixx{\envvar{TEXFORMATS}}{TEXFORMATS (environment variable)@\envvar{TEXFORMATS} (environment variable)}]

  Specifies where \program{texas} looks for format files.  Typically
  you would leave this set as is and add new format files to that directory.

  \ipitem[\ixx{\envvar{TEXFONTS}}{TEXFONTS (environment variable)@\envvar{TEXFONTS} (environment variable)}]

  \TeX\ uses the directories stored in this variable when searching for
  \ext{TFM} files for fonts used in your documents.  If you install new
  fonts, you should put the \ext{TFM} files in their
  own directory and add the name of that directory to the \envvar{TEXFONTS}
  path.

  \ipitem[\ixx{\envvar{TEXPOOL}}{TEXPOOL (environment variable)@\envvar{TEXPOOL} (environment variable)}]

  Specifies where \program{texas} looks for its pool file.  Unless you have
  installed \program{texas} in an unusual way, you will not have to change
  this setting.

  \ipitem[\ixx{\envvar{DOS4GVM}}{DOS4GVM (environment variable)@\envvar{DOS4GVM} (environment variable)}]

  \program{texas} uses a DOS extender\index{MS-DOS!extender for texas}
  to overcome memory limitations under
  MS-DOS.  If the extender runs out of memory, it writes parts of the
  memory-image that it is maintaining out to disk.  This environment
  variable specifies where that information should be written.  You do not
  usually need to change this setting.
\end{iplist}

\section{\protect\sbTeX}

\ixx{\sbTeX}{sbTeX@\sbTeX} is another MS-DOS implementation of \TeX.  The
\ixx{\filename{sb38tex}}{sb38tex} distribution includes \TeX\ and ini\TeX\ executables,
\ext{TFM} files for the Computer Modern Roman fonts, and \TeX\ sources for the
Plain \TeX\ format.  This is \TeX\ version 3.141.  You will have to get
\dvidriver{}s and other \TeX\ tools from a different package.

The \ixx{\filename{sb32xet}}{sb32xet} distribution is \XeT\ version 3.1.\footnote{\XeT\
is a variant of \TeX\ that can typeset in both left-to-right and right-to-left
modes.}  The source files for the \XeT\ \filename{nailp} format are included,
as well as the supporting files for Hebrew.

The \ixx{\filename{sbmf13}}{sbmf13} distribution is \MF\ version 2.71.  Only the
sources for the Plain \MF\ base file are provided.  Like the \sbTeX\
distribution, this is a very minimal set of files.  Only the
\ixx{\program{GFtoPK}}{GFtoPK} utility is included; all of the other \MF\
programs you need will have to be obtained from another package.

The default directories for \sbTeX\ are \filename{\bs TEX\bs INPUTS} for input
files, \filename{\bs TEX\bs FORMATS} for format files, and \filename{\bs
TEX\bs FONTTFMS} for \ext{TFM} files.  The default drive is C:, but that can
be changed with the \program{sb38set} program.

You can also modify these paths by setting the \envvar{TEXINPUTS} environment
variable\index{environment variables!for sbTeX@for \sbTeX}\index{variables!environment!for sbTeX@for \sbTeX} for 
input files, the \envvar{FMTSB} environment variable for format
files, and the \envvar{FONTTFMS} environment variable for \ext{TFM} files.
Additionally, \sbTeX\ uses the \envvar{SBFTMP} environment variable to 
determine where temporary files should be located if font caching is
performed. Here is a more complete description of sb\TeX's environment
variables:

\begin{iplist}[\textnormal]{.25in}
  \ipitem[\envvar{TEXINPUTS}]

  \index{TEXINPUT (environment variable)@\envvar{TEXINPUT} (environment variable)!for sbTeX@for \sbTeX}Defines the 
  directories that \sbTeX\ uses when searching for \TeX\
  input files.  You would typically leave the defaults in this path
  and extend it with directories of your own in place of \filename{C:{\bs}MYTEX}.

  \ipitem[\ixx{\envvar{FMTSB}}{FMTSB (environment variable)@\envvar{FMTSB} (environment variable)}]

  Specifies where \sbTeX\ looks for format files.  Typically
  you would leave this set as it is and add new format files to that 
  directory.

  \ipitem[\ixx{\envvar{FONTTFMS}}{FONTTFMS (environment variable)@\envvar{FONTTFMS} (environment variable)}]

  \TeX\ uses the directories stored in this variable when searching for
  \ext{TFM} files for fonts used in your documents.  If you install new
  fonts, you should put the \ext{TFM} files for these fonts in their
  own directory and add the name of that directory to the \envvar{FONTTFMS}
  path.

  \ipitem[\ixx{\envvar{SBFTMP}}{SBFTMP (environment variable)@\envvar{SBFTMP} (environment variable)}]

  When font-caching is enabled, \sbTeX\ writes cached information to disk when
  it runs out of memory.  This environment variable specifies where that
  information should be written.  You do not usually need to change this
  setting.
\end{iplist}

\section{\protect\gTeX}

The \ixx{\gTeX}{gTeX@g\TeX} package is distributed in two ways.  One way is intended to
supplement (actually, replace part of) the \emTeX\ distribution.  The
advantage of \gTeX\ executables over \emTeX\ is that they will run
under Windows.  The 386 versions of \emTeX\ executables will not run
under Windows.  On the other hand, although \gTeX\ will
run in an MS-DOS session under OS/2, only the \emTeX\
executables will actually run in a native OS/2 session.
The other distribution is a complete set of \ext{TFM} files, source, and
input files for Plain \TeX, \LaTeX, and \AMSTeX.
In either distribution, you must get \dvidriver{}s and some ancillary
programs from another complete package (like \emTeX).

The following programs are included in the \gTeX\
distribution:

\begin{exindent}
\begin{tabular}{lll}
\BibTeX\ & \program{GFtoPK} & \program{VFtoVP} \\
\program{DVItype} & \program{MFT} & \program{VPtoVF} \\
\program{DVIcopy} & \program{PKtoGF}\ \ \  & \program{MakeIndx} \\
\program{GFtoDVI}\ \ \  & \program{PKtype} & \MF \\[2pt]
\multicolumn{3}{l}{\program{amSpell}  (a third party spell-checker, not traditionally part of \TeX)}\\[2pt]
\multicolumn{3}{l}{\program{MEwin}  (an Emacs-like editor for Windows, also not traditionally part of \TeX)}\\
\end{tabular}
\end{exindent}

The \gTeX\ executables use the same environment 
variables\index{environment variables!for gTeX@for \gTeX}\index{variables!environment!for gTeX@for \gTeX} as \emTeX.


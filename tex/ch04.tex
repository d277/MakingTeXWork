\extrachapspace=.8pc
\chapter{Macro Packages} 
\RCSID$Id: ch04.tex,v 1.1 2002/08/23 14:58:46 nwalsh Exp $
\label{chap:macpack}

\ifincludechapter\else\endinput\fi

Everyone who uses \TeX\ uses a \ixx{macro  package}{macro packages}\index{format files!for  macro packages} (also called a ``format'').
A macro package extends \TeX\ to provide functionality that is
suited to a particular task or set of tasks.

This chapter provides a summary of \TeX\ macro packages.  General-purpose
packages designed to typeset a wide range of documents---articles, books,
letters, and reports---are examined first.  The general-purpose packages
described are Plain \TeX, Extended Plain \TeX, \LaTeXe, \LaTeX, \AMSTeX,
\AMSLaTeX, Lollipop, and \TeXinfo.  After surveying the general-purpose
packages, several special-purpose packages designed to handle
specific tasks---typesetting transparencies, music, chemistry or physics
diagrams, etc.---are described.  The special-purpose packages surveyed
are \SliTeX, \FoilTeX, Seminar, \MusicTeX, ChemStruct, and Chem\TeX.

There are a lot of overlapping features and similar commands in the
general-purpose packages.  To understand why this is the case,
consider how a new macro package comes into existence.  An ambitious person,
who is very familiar with \TeX, decides that there are some things she would like to
express in her documents that are difficult to express with existing formats.
Perhaps, for example, no existing format produces documents that match the
precise specifications required for publishing in her field, or perhaps she
has in mind a whole new document structuring paradigm.  A more mundane
possibility is simply that she has been customizing an existing format for
some time and now feels it has enough unique features to be useful to
others.

In any event, a new format is born.  Now, if this format is designed
for a very specific task, writing multiple-choice mathematics exams, for
example, it might not have very many general-purpose writing features.
On the other hand, if it is designed for writing longer, more general
documents (e.g., history textbooks or papers to appear in a particular journal)
then there are a number of features that it is likely to include; provisions
for numbered lists, cross references, tables of contents, indexes,
and quotations are all examples of features common to many documents.

To support these common features,
many macro packages have similar control sequences.  This stems from the 
fact that they are all built on top of a common set 
of \ixx{primitives}{primitives (tex)@primitives (\TeX)}  and
that macro package authors tend to copy some features of
other packages into their own.

You may find that you'd like to use the features of several
different packages in the same document.  Unfortunately, there is no
provision for using multiple formats to process a single document.
The features required to process most documents are shared by
all of the general-purpose formats, however.  You are more likely to need
multiple macro packages if you want to use a special format to
construct a diagram or figure and incorporate it
into a document.  Chapter~\ref{chap:pictures},
{\it \nameref{chap:pictures}\/}, describes several ways to take
``electronic snippets'' of one document and insert them into another,
which is one possible solution to this problem.

If you're beginning to feel a little lost,
%``in a maze of twisty macro packages, all alike'', 
have no fear.  Most general-purpose formats are
sufficient for most documents.  And there's no reason why every
document you write has to be done with the same format.  Many people
find \LaTeX\ and Plain \TeX\ sufficient, but if you're writing an
article for a particular journal and someone has written a format
specifically for that journal's documents, by all
means use it.  It is more likely, however, that someone has written a
\ixx{style file}{style files} which 
tailors \LaTeX\ to the requirements of the
journal.

In addition to describing some common macro packages in this chapter, I'll
describe how to build format files for them.
If the packages that you want to use have already been
installed at your site, you can ignore the installation sections.

The packages that you find most convenient will
depend on the tasks you perform and how well each package suits your
work style.  The list of packages in this chapter is not meant to be
all-inclusive, nor is it my intention to suggest which packages are best.
Use the ones {\em you\/} like, for whatever reasons.

I can hear some of you already, ``I don't really need a macro package,'' you
say, ``I can roll my own with just \TeX.''  

And you are absolutely correct.

I don't recommend it, however.  It's akin to using your compiler without
any of the built-in functions.  Most \TeX\ primitives offer little
support by themselves for writing documents.

\begin{sidebar}[A New Perspective]
\ixx{\LaTeXe}{latexe@\LaTeXe} (and its successor, which will be \ixx{\LaTeX3}{latex3@\LaTeX3}) seek to address
many of the problems mentioned above by defining a core \LaTeX\
format with extension packages to provide custom features.

The \LaTeXe\ system provides a single 
format \index{format files!latexe core@\LaTeXe\ core} file that supports
\ixx{\LaTeX}{latex@\LaTeX}, \ixx{\AMSTeX}{AMSTeX@\AMSTeX}, and \ixx{\SliTeX}{SliTeX@\SliTeX}.  For 
the time being, \LaTeXe\ is described
separately in this chapter along with the other formats.  
Be aware, however, that \LaTeXe\ is now the standard \LaTeX\footnotemark
and you should migrate to it as soon as possible.
\end{sidebar}
\footnotetext{At the time of this writing, it's actually still in test-release, but it 
may be available as a standard release by the time you read this.}
\section{Installation: Making Format Files}

A format file\index{format files!making}\index{macro packages!making format files}, as 
described in Chapter~\ref{chap:running},
{\it \nameref{chap:running}}, is a special ``compiled'' version of a
macro package.  The \ixx{ini\TeX}{initex@ini\TeX} program interprets all of the control
sequences in a macro package and writes the corresponding memory image
into a file.
Loading a format file is much faster than loading individual macro
packages in your document because \TeX\ does not have to interpret any 
of the control sequences while it is also processing text.

In general, all format files should be stored in the same
directory.\stepcounter{footnote}\ff[2]{On Macintosh systems and other environments that don't
have directories, format files are typically stored in their own folders (or
other metaphorically appropriate place ;-).}  If you install \TeX\ in a
directory called \filename{tex}, then formats typically go in a directory
called \filename{formats}\index{formats directory} in 
the \ixx{\filename{tex} directory}{tex directory}.  This is not
universally the case because you need separate format directories for big and
small \TeX{}s.\ff[3]{The distinction between big and small \TeX\ is
described in the section called ``\nameref{sec:bigandsmall}'' in
Chapter~\ref{chap:running}.}

Usually, an \ixx{environment variable}{environment  variables}\index{variables!environment} indicates 
where the format files are
located.  Environment variables are a common way of customizing your
interaction with programs.  They are usually set in 
your \ixx{\filename{AUTOEXEC.BAT} file}{AUTOEXECBAT files@AUTOEXEC.BAT files} for 
MS-DOS, \ixx{\filename{CONFIG.SYS}}{CONFIGSYS files@CONFIG.SYS files} for 
OS/2, or the \textit{rc}-file
for a shell (i.e., \ixx{\filename{.cshrc}}{cshrc file@.cshrc file}, 
\ixx{\filename{.kshrc}}{kshrc files@.kshrc files}) on \Unix\ systems.  
Any good reference book for your operating system
or shell will describe how to use environment variables.

A common name for the environment variable that indicates where \TeX\ formats
are located is \envvar{TEXFORMATS}.  
Implementations that
provide big and small \TeX{}s need another variable to
indicate the directory that contains formats for big \TeX.

Ini\TeX\ is not always a separate program.  Some implementations of \TeX\
combine the functionality of \TeX\ and ini\TeX\ into a single program and
use a special switch at runtime to determine which function to perform.
In this chapter, all of the examples use the program name \filename{initex}
to identify ini\TeX.  If you use an implementation that doesn't
provide a separate ini\TeX\ program, you should use the \TeX\ program
with the \ixx{ini\TeX\ switch}{initex@ini\TeX!switch} instead.  For example, for
em\TeX, use \filename{tex /i} instead of \filename{initex} when you
build a format.

Like \TeX, ini\TeX\ needs to be able to find input files\index{initex@ini\TeX!using environment variables}.  Usually, this is accomplished by searching the
directories listed in the \envvar{TEXINPUTS} environment variable.  Place the
input files that ini\TeX\ needs in a directory on the \envvar{TEXINPUTS} path
before running ini\TeX\ unless otherwise directed.  The \envvar{TEXINPUTS}
environment variable is discussed in the ``\nameref{sec:userfiles}'' section
of Chapter~\ref{chap:running}.

\subsection{Hyphenation Patterns}

In order for \TeX\ to correctly hyphenate words, every format file must
contain a set of \ixx{hyphenation patterns}{hyphenation!patterns}.  The 
patterns are part
of an algorithmic solution to the problem of breaking a word into 
syllables for hyphenation.

The details of the hyphenation algorithm (given in Appendix H of
\textit{The \TeX{}book}~\cite{kn:texbook}) are too complex to describe
here, but two aspects of this solution deserve particular emphasis.
First, using patterns means that a dictionary of hyphenated words is
not necessary.\footnote{A small set of exceptions is maintained because
the algorithm isn't perfect.}  This saves a lot of space and time.  Second, 
by loading different sets of patterns, \TeX\ can achieve equal success at
hyphenating any language---even English ;-).  There are actually 
at least two sets of hyphenation patterns for English, one for British 
English and one for American English.
Chapter~\ref{chap:foreign}, \textit{\nameref{chap:foreign}}, describes
how to load multiple sets of hyphenation patterns for typesetting
multilingual documents\index{hyphenation!language-specific}.

\section{General-purpose Formats}

This section describes several macro 
packages \index{macro packages!general-purpose} that are designed for
formatting standard documents like articles or books.
In order to provide some form of comparative measure, each macro
package is used to create the same document, a one-page report that looks like
Figure~\ref{fig:sampresults}.  I constructed this example to
demonstrate a few common elements in a document: several sizes of
headings, a paragraph of text, inline and displayed mathematics, and a
few fonts.  There are lots of other things that aren't shown (tables,
figures, footnotes, etc.), and these elements vary as much as any other
in the different macro packages.

\epspage{perf-ltx.eps}{2.4in}{.9,.7}{Sample page}{fig:sampresults}

%Included in each section is a sample of the commands required to
%produce this report.
%
%Some macro packages, like \AMSTeX, are designed to produce documents
%that match particular specifications.  If the result is substantially
%different from Figure~\ref{fig:sampresults}, the document produced is
%also shown.
%
Observant readers will notice that the examples are shown in the Computer
Modern fonts while the rest of this book uses different fonts.\footnote{Really
observant readers may have noticed that it's a version of Garamond ;-)}
There are a number of reasons why the example is shown in Computer Modern.
For one thing, all of the formats discussed here use the Computer Modern fonts
by default.  Using a different set of fonts would only add more
complexity to each example.  A more subtle problem is that I do not have
appropriate mathematics fonts for Garamond.  There are a number of complex
issues involving the use of fonts in \TeX.  They are discussed in
Chapter~\ref{chap:fonts}, {\it \nameref{chap:fonts}}.

\subsection{Plain \protect\TeX}

\ixx{Plain \TeX}{Plain tex@Plain \TeX} is the format \index{macro packages!Plain tex@Plain \TeX} 
written by \ixx{Donald Knuth}{Knuth, Donald} during the development
of \TeX.  It is described fully in {\it The {\TeX}book}~\cite{kn:texbook}.

Plain \TeX\ ties together 
the \TeX\ primitives \index{primitives (tex)@primitives (\TeX)} in a way that makes
it practical to work in \TeX.  If you do not have a computer programming
background, you may find Plain \TeX\ a little bit intimidating.  It
is definitely a ``roll your own'' environment.  Although it demonstrably
contains all of the functionality required to write everything from
letters to books, there is very little ``user-friendly'' packaging
around the internals of \TeX.

Aside from user interface considerations, which are highly subjective,
Plain \TeX\ lacks some functionality when compared to other formats.
There is no provision in Plain \TeX\ for automatically numbered
sections, labelled figures, tables of contents, indexes, or bibliographies.
Any of these functions can be constructed in Plain \TeX\ if you are
willing to invest the time and energy required to write your own
macros, but they are not built into Plain \TeX.

If you enjoy writing your own macros or plan to produce
novel types of documents, a firm grasp of Plain \TeX\ will allow you
to write anything in \TeX.  A firm grasp of Plain \TeX\ also makes it easier
to understand and modify other formats (like \LaTeX) that are built
on top of Plain \TeX.

In addition, Plain \TeX\ is the only format that is always distributed with
\TeX.  The other formats discussed in this chapter are freely
available but do not come with \TeX.

The Plain \TeX\ input that produces the report in
Figure~\ref{fig:sampresults} is shown in Example~\ref{ex:sampplaintex}.

\exampleinput{perf-ptx}{ex:sampplaintex}{Plain \protect\TeX\ Input}

%\epspage{perf-ptx.eps}{2.5in}{.85,.375}{Plain \protect\TeX\ results}{fig:plainresults}

\subsubsection{Building the Plain \protect\TeX\ format}

To build Plain \TeX\index{Plain tex@Plain \TeX!building}, you 
need only two files: \filename{plain.tex}\index{plaintex files@plain.tex files!for Plain tex@for Plain \TeX} and
\filename{hyphen.tex}\index{hyphentex files@hyphen.tex files!for Plain tex@for Plain \TeX}.  These 
files are distributed with \TeX\ so they
should be available as soon as you have installed \TeX.  
The \filename{hyphen.tex} file is language-dependent.  Readers who frequently
work with non-English text should read Chapter~\ref{chap:foreign}
for more information about obtaining
non-English hyphenation patterns.

The command:

\begin{ttindent}
\$ \textbf{initex plain \bs{}dump}
\end{ttindent}

will create the Plain \TeX\ format.
Move the resulting 
files, \filename{plain.fmt}\index{plainfmt files@plain.fmt files} and
\filename{plain.log}\index{plainlog files@plain.log files}, into 
your \TeX\ formats directory.

\subsection{Extended Plain \protect\TeX}

\ixx{Extended Plain \TeX}{Extended Plain tex@Extended Plain \TeX}\index{macro packages!Extended Plain tex@Extended Plain \TeX} extends Plain \TeX\ in a number of
useful ways without forcing you to use any particular ``style'' of
output.
The argument is this: although Plain \TeX\ really doesn't provide all of the
features that you need (tables of contents, cross references, citations,
enumerated lists, convenient access to verbatim input, etc), many of these
features don't have any direct impact on the appearance of your document.
Unfortunately, other general-purpose macro packages like \LaTeX, which do
provide these features, tend to force you to accept their notion of what the
typeset page should look like.\footnote{Of course, that's not strictly true.
You can change the page layout of \LaTeX\ (and most other packages) to be
almost anything, but it does require learning a lot about the macro package.
If you are already familiar with Plain \TeX\ (or some other Plain \TeX-derived
package), you probably have a set of macros that produce documents in the
style you like.  Why reinvent the wheel?}

Extended Plain \TeX\ is an attempt to solve that problem.
It provides many behind-the-scenes features without providing any general
page layout commands (like \cs{chapter} or \cs{section}), which means
that these features can be used inside Plain \TeX\ without much difficulty 
and without
changing the layout of typeset pages.

\subsubsection{Building the Extended Plain \TeX\ format}

To build the Extended Plain \TeX\ format\index{Extended Plain tex@Extended Plain \TeX!building}, you need the \filename{plain.tex}\index{plaintex files@plain.tex files!for Extended Plain tex@for Extended Plain \TeX}
and \filename{hyphen.tex} files \index{hyphentxt files@hyphen.txt files!for Extended Plain tex@for Extended Plain \TeX}  required to build the Plain format as well as
the \filename{eplain.tex} file \index{eplaintex files@eplain.tex files!for Extended Plain tex@for Extended Plain \TeX}  distributed with Extended Plain
\TeX.\footnote{Extended Plain \TeX\ is available in the
\filename{macros/eplain} directory in the CTAN archives.}  Make
and install the Plain \TeX\ format first, then change to the directory that
contains the Extended Plain \TeX\ distribution.

The command:

\begin{ttindent}
\$ \textbf{initex \&plain eplain \bs{}dump}
\end{ttindent}

will create the Extended Plain \TeX\ format.  Move the resulting
files, \filename{eplain.fmt} \index{eplainfmt files@eplain.fmt files!for Extended Plain tex@for Extended Plain \TeX}  and \filename{eplain.log}\index{eplainlog files@eplain.log files!for Extended Plain tex@for Extended Plain \TeX}, into your \TeX\ formats
directory.

\subsection{\protect\LaTeXe\ Versus \protect\LaTeX}
\label{macpack:sec:latexvslatexe}

The tremendous popularity of \ixx{\LaTeX}{latex@\LaTeX}\index{macro packages!latex@\LaTeX} in the \TeX\ community has had an
unfortunate side effect: because it is a very familiar and flexible
format, many people have used it as the basis for extensions of one sort
or another.  This has resulted in a wide range of (slightly) incompatible
formats and a lot of frustration.  

This situation is being rectified by a new release of \LaTeX, currently called
\ixx{\LaTeXe}{latexe@\LaTeXe}\index{macro packages!latexe@\LaTeXe}.  The new 
release replaces the existing dialects of \LaTeX\ (\LaTeX\
with and without NFSS, \SliTeX, \AMSLaTeX, etc.) with a single core system
and a set of extension packages.  \LaTeXe\ includes a compatibility mode
which will allow it to continue to format existing documents without change
(provided that they do not rely on local modifications to the \LaTeX\
format, of course).  Local modifications can also be incorporated into
the \LaTeXe\ system as extension packages, making \LaTeXe\ a complete
replacement for all existing versions of \LaTeX\ and packages closely
derived from \LaTeX.

The most significant and least compatible difference between \LaTeX\
and \LaTeXe\index{latex@\LaTeX!differences from latexe@differences from \LaTeXe}\index{latexe@\LaTeXe!differences from latex@differences from \LaTeX} is the font selection scheme.
%Should this occur here or just in section five? Section starting now.
There are many control sequences for selecting fonts in \LaTeX.  Some
control the typeface (\cs{rm}, \cs{tt}, \cs{sc}, etc.);
some the size (\cs{small}, \cs{normalsize}, \cs{large}, etc.);
and some the appearance (\cs{it}, \cs{bf}, \cs{em}, etc.).
  
Under the \idx{Old Font Selection Scheme (OFSS)}, the control sequences for
selecting a font completely override any font selection already in place.
Consider, for example, the control sequences \cs{it} and \cs{bf},
which switch to italic and boldface.  Using \cs{bf}\cs{it} produces italic
text, and \cs{it}\cs{bf} produces boldface text, and {\em neither} produces
boldfaced-italic text (which is probably what you wanted).

Under the \idx{New Font Selection Scheme}\index{fonts!NFSS}, typeface 
(called \textit{family}
in NFSS parlance), appearance (called \textit{series} and \textit{shape}), and
size are viewed as orthogonal components in font selection.  Because these
parameters are independent, selecting an italic appearance with the \cs{it}
control sequence switches to italic in the current typeface and size.  Under
the NFSS, \cs{bf}\cs{it} \textit{does} select boldface-italic in both the current
typeface and size (if it is available).
% Queried section ends here.

\LaTeXe\ supports only the NFSS, 
version 2 (called NFSS2)\index{New Font Selection Scheme!version 1 (NFSS)}\index{fonts!NFSS2}.  For 
more than a year, the NFSS (initially
version 1, and more recently version 2) has been available as an
extension for \LaTeX\ 2.09.  However, in light of stable test releases of
\LaTeXe, the NFSS2 package for \LaTeX\ 2.09 has been withdrawn.  The
discussion of NFSS in this book applies equally well to \LaTeX\ 2.09 with
NFSS2, but it is described in terms of \LaTeXe\ in an effort to be more
applicable in the future.
The NFSS is discussed in more detail in Chapter~\ref{chap:fonts}.

\subsection{\protect\LaTeXe}
\label{sec:latexe}

\ixx{Leslie Lamport}{Lamport, Leslie}'s \LaTeX\ format 
is probably the most commonly used \TeX\
format.  It is described in {\it \LaTeX: A Document Preparation
System}~\cite{ll:latexbook} and many other \TeX\ books.  \ixx{\LaTeXe}{latexe@\LaTeXe} is 
the new
standard \LaTeX.  It is described in \textit{The \LaTeX\
Companion}~\cite{latexcompanion}.  The next edition of {\it \LaTeX: A Document
Preparation System}~\cite{ll:latexbook}
will also describe \LaTeXe.

\begin{sidebar}
At the time of this writing, \LaTeXe\ is available only in a test release, but
by the time you read this, it is likely to be available as the new standard
\LaTeX.  It is described first in this chapter to emphasize that you
should begin using \LaTeXe\ as soon as possible.  Once \LaTeXe\ is out of
testing, it will become \LaTeX, and support for older versions will not be
provided (at least not by the \LaTeX\ developers).  \LaTeXe\ includes a
\ixx{compatibility mode}{latex@\LaTeX!compatibility mode} for 
old \LaTeX\ documents so the transition should be
relatively painless.
\end{sidebar}

The central theme of \LaTeX\ is ``structured document preparation.''  An
ideal \LaTeX\ document is described entirely in terms of its structure:
chapters, sections, paragraphs, numbered lists, bulleted items, tables,
figures, and all the other elements of a document are
identified descriptively.  For example, you enclose figures in a figure
\textit{environment} identified by the control sequences 
\verb|\begin{figure}| and \verb|\end{figure}|.

When you are ready to print your document, select an appropriate document
class\index{class document in latex@class document in \LaTeX}, and \LaTeX\ formats 
your document according to the rules of the
selected style.  In the case of the ideal document, it might first be printed
in a magazine or newsletter using the article class.  Later, when it is
incorporated into a book, selecting the book class is {\em all\/} that is
required to produce appropriate output; the document itself is unchanged.

\LaTeX\ is written on top of Plain \TeX.  This means that almost any control
sequence \index{control sequences!in latex@in \LaTeX} or 
macro that you learn about in Plain \TeX\ can also be used in
\LaTeX.  Of course, \LaTeX\ insulates you from many Plain \TeX\ commands by
wrapping a much more user-friendly interface around them.

The \LaTeXe\ input that produces the sample page in
Figure~\ref{fig:sampresults} is shown in Example~\ref{fig:samplatex2e}.
The only difference between this document and an old \LaTeX\ document is
the use of the \cs{documentclass} declaration instead of the
\cs{documentstyle} declaration.\footnote{\LaTeXe\ will process
documents that use \texttt{\ttbackslash documentstyle} in \LaTeX\ 2.09
compatibility mode.}  For more complex documents, other minor
changes may also be necessary.

\exampleinput{perf-l2e}{fig:samplatex2e}{\protect\LaTeXe\ Input}

\subsubsection{Building the \protect\LaTeXe\ format}
\label{sec:buildlatexe}

The \LaTeXe\ distribution is available from the directory
\filename{macros/latex2e/core} in the CTAN 
archives\index{latex@\LaTeX!building}.

The following steps will build the \LaTeXe\ format. For more complete
installation instructions, read the file \filename{install.l2e} in the 
\LaTeXe\ distribution.

\begin{enumerate}
  \item Place the \LaTeXe\ distribution in a temporary directory and make that
    directory the current directory.  After the installation is complete, you
    will need to move only selected files into the standard places.

\item Restrict to only the current directory the directories that \TeX\
  searches for input files

  This can usually be accomplished by setting the environment variable
  \texttt{TEXINPUTS}\footnote{Under \emTeX, this variable is called
  \texttt{TEXINPUT}.  On the Macintosh, file searching is frequently
  controlled by a configuration file or dialog box.}  to a single period or
  the absolute path name of the current directory.

\item Copy the \filename{hyphen.tex} file \index{hyphen.tex files!for latexe@for \LaTeXe} from the Plain \TeX\ distribution
  into the current directory.

\item Issue the command:

\begin{ttindent}
\$ \textbf{initex unpack2e.ins}
\end{ttindent}

  This will unpack all of the distribution files.

  \item Build the format file by issuing the command:

\begin{ttindent}
\$ \textbf{initex latex2e.ltx}
\end{ttindent}

   Move the resulting 
   files \filename{latex2e.fmt} \index{latex2e.fmt files} and \filename{latex2e.log}\index{latex2elog files@latex2e.log files} into 
   the \TeX\ formats directory.

  \item In addition to the files needed to build the format, unpacking
    the \LaTeXe\ distribution creates many files that are needed for 
    formatting documents.
    These files must be placed in a location where \TeX\ will find them.
    However, in order to maintain a functioning \LaTeX\ 2.09 system,
    you must not place the new files in the same input directory as the
    existing files.\footnote{Strictly speaking, this is only true for files
    that have the extension \filename{.sty} because the old version of 
    \LaTeX\ will not attempt to use the other files.}

    Create a new directory (or folder) for the new files.  On the
    \Unix\ system that I use, where existing input files are stored in a
    directory called \filename{/usr/local/lib/tex/inputs}, I created
    \filename{/usr/local/lib/tex/latex2e} to store the new files.  You will
    have to add the new directory to the \emph{front} of the list of
    directories that \TeX\ searches for input files whenever you format a
    document with \LaTeXe.

    Move the files that the installation script produces into the new
    directory.  Move the files \filename{docstrip.tex},
    \filename{latexbug.tex}, \filename{sfontdef.ltx}, \filename{slides.ltx},
    \filename{testpage.tex}, and all of the files that end in \filename{.cfg},
    \filename{.cls}, \filename{.clo}, \filename{.def}, \filename{.fd}, and
    \filename{.sty}.  You should also move the files \filename{gglo.ist} and
    \filename{gind.ist} someplace where \program{MakeIndex} can find them.
    (\program{MakeIndex} is described in Chapter~\ref{chap:bibtex},
    \textit{\nameref{chap:bibtex}}.)

    One of the aspects of the test releases that continues to change
    is the exact list of files that must be moved.  Consult the 
    \filename{install.l2e} file in the distribution for the exact
    list.  The list above is from the test version of January 28, 1994.
\end{enumerate}

\subsection{\protect\LaTeX}
\label{sec:latex}
\label{macpack:sec:latex}

This section briefly covers \LaTeX\ version 2.09.  This version of
\ixx{\LaTeX}{latex@\LaTeX}\index{macro packages!latex@\LaTeX} is still very 
widely used but it is being phased out.

The \LaTeX\ input to produce the sample page in Figure~\ref{fig:sampresults}
is shown in Example~\ref{fig:samplatex}.  

\newpage
\exampleinput{perf-ltx}{fig:samplatex}{\protect\LaTeX\ Input File}

Support for the NFSS in \LaTeX\ 2.09 has been withdrawn.  If you need
to build a format with support for NFSS, consult the ``\nameref{sec:latexe}''
section of this chapter.

\subsubsection{Building the \protect\LaTeX\ format with the OFSS}
\label{sec:buildlatexofss}

The \LaTeX\ \index{latex@\LaTeX!building}\index{Old Font Selection Scheme (OFSS)}distribution\footnote{\LaTeX\ is available in the
\filename{macros/latex/distribs/latex} directory in the CTAN archives.}
includes three subdirectories, \filename{sty}, \filename{doc}, and
\filename{general}.  All of the \LaTeX\ files required to
build the format file 
%(\filename{lplain.tex}, \filename{lhyphen.tex}, \filename{lfonts.tex}, 
%and \filename{latex.tex}) 
are in the \filename{general} subdirectory.  You will also need the
\filename{hyphen.tex} file \index{hyphen.tex files!for latex@for \LaTeX} required 
to build the Plain format.

In the \filename{general} subdirectory, the command:

\begin{ttindent}
\$ \textbf{initex lplain}
\end{ttindent}

will create the \LaTeX\ format.  Move the resulting files,
\filename{lplain.fmt} \index{lplainfmt files@lplain.fmt files!for latex@for \LaTeX} and 
\filename{lplain.log}\index{lplainlog files@lplain.log files!for latex@for \LaTeX}, to the \TeX\ formats
directory.  In order to complete the installation, copy the files from the
\filename{sty} directory in the \LaTeX\ distribution into a directory where
\TeX\ searches for input files.

\subsection{\protect\AMSTeX}

When the \idx{American Mathematical Society} selected \TeX\ as a document 
preparation system, they
decided to extend it in a number of ways to make the creation of papers
and journals easier.  They had two goals: to make it easier for authors
to write mathematical papers in \TeX\ and to make the resulting papers
conform to a particular set of style specifications.
\ixx{\AMSTeX}{AMSTeX@\AMSTeX} is described completely 
in {\it The Joy of \TeX}~\cite{ms:joyoftex}.

\AMSTeX\ provides many commands that resemble \LaTeX\ environments.  These
have the form \texttt{\bs \textit{environment}} $\ldots$ 
\texttt{\bs end\textit{environment}}.  
In addition, \AMSTeX\ provides the notion of a document
style to control style-related formatting issues.

Another important contribution made by the \ams\ when creating \AMSTeX\
was the construction of a large number of new 
fonts\index{AMSTeX@\AMSTeX!fonts}\index{fonts!AMSTeX@\AMSTeX}.  
The \ams\ provides
fonts with many more mathematical symbols than the fonts that come with
\TeX.  These fonts are available as a separate package and can be used
with any \TeX\ macro package, not just \AMSTeX.

The \AMSTeX\ input required to produce the document in
Figure~\ref{fig:amsresults} is shown in Example~\ref{ex:sampamstex}.  The
result of formatting this document does not appear exactly like
Figure~\ref{fig:sampresults} because \AMSTeX\ uses the style conventions of
the \ams.

\exampleinput{perf-amt}{ex:sampamstex}{\protect\AMSTeX\ Input File}

\epspage{perf-amt.eps}{2.5in}{.85,.75}{AMS sample page}{fig:amsresults}

\subsubsection{Building the \protect\AMSTeX\ format}

In order to build the \AMSTeX\ format, you need 
the \filename{plain.tex} \index{plaintex files@plain.tex files!for AMSTeX@for \AMSTeX} and 
\filename{hyphen.tex} files \index{hyphentex files@hyphen.tex files!for AMSTeX@for \AMSTeX} from 
Plain \TeX\ and the \filename{amstex.ini} \index{amstexini files@amstex.ini files} and
\filename{amstex.tex} \index{amstextex files@amstex.tex files} files 
from the \AMSTeX\ distribution\index{AMSTeX@\AMSTeX!building}.

The command:

\begin{ttindent}
\$ \textbf{initex amstex.ini}
\end{ttindent}

will create the \AMSTeX\ format.  Move the resulting files,
\filename{amstex.fmt} \index{amstexfmt files@amstex.fmt files} and 
\filename{amstex.log}\index{amstexlog files@amstex.log files}, into 
your \TeX\ formats
directory.

\subsection{\protect\AMSLaTeX}

\ixx{\AMSLaTeX}{AMSLaTeX@\AMSLaTeX}\index{macro packages!AMSTeX@\AMSTeX}, like \AMSTeX, 
provides many features to make typesetting 
mathematics convenient while meeting the standards of the 
\idx{American Mathematical Society} for 
publication.  However, \AMSTeX\ lacks many of 
the features that are present in \LaTeX, like automatically numbered 
sections and tools for creating tables of contents and indexes.

When \LaTeX\ gained popularity, many authors requested permission to
submit articles to the \ams\ in \LaTeX.
In 1987, the \ams\ began a project to combine the features of \AMSTeX\
with the features of \LaTeX.  The result is \AMSLaTeX.

\AMSLaTeX\ provides all of the functionality of \LaTeX\ because it is an
extension of \LaTeX. It also provides the functionality of \AMSTeX\ in
\LaTeX\ syntax and access to additional mathematical constructs and math 
symbols not present in \LaTeX.  .

The input required to produce Figure~\ref{fig:sampresults} is not shown
because they do not differ significantly from the \LaTeX\ sample.
Because the sample document doesn't use any of \AMSLaTeX's additional
features, it is exactly the same as the \LaTeX\ document.

\subsubsection{Building the \protect\AMSLaTeX\ format}

The \AMSLaTeX\ macros are embodied entirely in style files for \LaTeX.  It is
not necessary to build a special format file\index{AMSLaTeX@\AMSLaTeX!building}.
However, the \AMSLaTeX\ macros require the New Font Selection
Scheme (NFSS).  Consult the section on \LaTeX, above, for instructions on
building the \LaTeX\ format with NFSS.

\subsection{Lollipop}

It can be argued that \LaTeX\ has the following deficiency: although there
are many different style options available, it is not easy for a novice
user to change a style option.  Changing the internals of most \LaTeX\ style
options requires a deep understanding of \TeX.

The \idx{Lollipop}\index{macro packages!Lollipop} format 
is very different from the other formats.
The central thrust of Lollipop is that it should be easy to change
and customize document styles.
All Lollipop documents are built from five different generic
constructs: headings, lists, text blocks, page grids, and external
items.

The Lollipop input required to produce a document like the sample page
in Figure~\ref{fig:sampresults} is shown in Example~\ref{ex:samplollipop}.

\exampleinput{perf-lol}{ex:samplollipop}{Lollipop Input File}

%\epspage{perf-lol.eps}{2.5in}{.85,.75}{Lollipop sample page}{fig:lollipopresults}

\subsubsection{Building the Lollipop format}

To make \index{Lollipop!building}the Lollipop 
format, you need the Lollipop distribution and the file
\filename{hyphen.tex} from the Plain \TeX\ distribution.

This command will create the Lollipop format:

\begin{ttindent}
\$ \textbf{initex lollipop \bs{}dump}
\end{ttindent}

It should be performed
in the directory where you installed the Lollipop distribution so
that all of the Lollipop files can be located.  
Move the resulting 
files \filename{lollipop.fmt}\index{lollipopfmt files@lollipop.fmt files} and
\filename{lollipop.log}\index{lollipoplog files@lollipop.log files} into 
your \TeX\ formats directory.

\subsection{\protect\TeXinfo}

The \TeXinfo\ format is a general-purpose format, but it was designed to
support a particular application: to produce both online documentation and
professional quality typeset documentation from the same source file.  It is
discussed in more detail in Chapter~\ref{chap:online},
{\it \nameref{chap:online}}.

The input file shown in Example~\ref{ex:infinp} produces the typeset output
shown in Figure~\ref{fig:infex}.  The input file for this example is
complicated by the fact that it contains mathematics.  None of \TeX's
sophisticated mechanisms for handling mathematics are applicable to plain
ASCII online documentation.  The online documentation produced by the
example in Example~\ref{ex:infinp} is shown in Figure~\ref{fig:infinf}.

The \ixx{\TeXinfo}{texinfo@\TeXinfo}\index{macro packages!texinfo@\TeXinfo} format 
is the official documentation format of the \idx{Free Software Foundation (FSF)}.  Although less commonly used, a \LaTeX\ variant
called \ixx{\LaTeX{}info}{latex{}info@\LaTeX{}info} is also available.

\exampleinput{perf-inf}{ex:infinp}{\protect\TeX{}info Input}

\epspage{perf-inf.eps}{3.25in}{.49,3.075}{\protect\TeX{}info sample page}{fig:infex}

\begin{figure}
  \begin{center}
    \FramedVerbInput{perf-inf.inf}
    \caption{Online documentation produced by MakeInfo}
    \label{fig:infinf}
  \end{center}
\end{figure}

\subsection{Other Formats}

There are a number of other macro packages available for \TeX.  Some of them
are summarized below.  The fact that they are not discussed more fully here
(or listed below, for that matter) is not intended to reflect on the quality
of the format.  The formats discussed above are examples of the ways in which
\TeX\ can be extended.  All of the formats below extend \TeX\ in a way similar
to one of the formats already mentioned.  For any particular application, one
of these macro packages might be a better choice than the formats discussed 
above.

\begin{iplist}{.25in}
  \ipitem [EDMAC]

    Provides support for typesetting critical editions of texts
    in a format similar to the Oxford Classical Texts with marginal line
    numbers and multiple series of footnotes and endnotes keyed by line
    number.  \idx{EDMAC}\index{macro packages!EDMAC} is 
    available from the CTAN archives in the directory
    \ctandir{macros/plain/contrib/edmac}.

  \ipitem [\INRSTeX]

    Provides support for multilingual documents in French\index{French} and 
    English.  \ixx{\INRSTeX}{INRSTeX@\INRSTeX}\index{macro packages! INRSTeX} is 
    available from the CTAN archives in the directory
    \ctandir{macros/inrstex}.

\newpage
  \ipitem [\LamSTeX]

    Extends \AMSTeX\ with \LaTeX-like features and improved
    support for commutative diagrams.  \ixx{\LamSTeX}{LamSTeX@\LamSTeX}\index{macro packages!LamSTeX@\LamSTeX} is available from the CTAN
    archives in \ctandir{macros/lamstex}.

  \ipitem [\REVTeX]

    Extends \LaTeX\ to provide support for typesetting articles
    for journals of the \idx{American Physical Society}, 
    the \idx{Optical Society of America}, and the \idx{American Institute for Physics}.  \ixx{\REVTeX}{REVTeX@\REVTeX}\index{macro packages!REVTeX@\REVTeX} is available in
    the directory \ctandir{macros/latex/contrib/revtex} of the CTAN archives. 

  \ipitem [\TeX{}sis]

    Provides facilities for typesetting articles, papers, and
    theses.  It is particularly tuned for physics 
    papers.  \ixx{\TeX{}sis}{tex{}sis@\TeX{}sis}\index{macro packages!tex{}sis@\TeX{}sis} also
    provides support for other kinds of documents, such as letters and memos.
    It is based upon Plain \TeX.  \TeX{}sis is available from the CTAN
    archives in the directory \ctandir{macros/texsis}.

    In addition to \REVTeX\ and \TeX{}sis, there are several other packages in
    the \filename{macros} directory on CTAN that were designed for
    typesetting documents about 
    physics: \idx{PHYSE}\index{macro packages!PHYSE}, 
    \idx{PHYZZX}\index{macro packages!PHYZZX}, 
    and \idx{PSIZZLE}\index{PSIZZLE}.

  \ipitem [\TeX/Mathematica]

    Supports interactive use of \idx{Mathematica} on \Unix\
    workstations running GNU emacs.  Mathematica explorations can be annotated
    with \TeX/\LaTeX, and Mathematica graphics can be incorporated
    into documents.  \ixx{\TeX/Mathematica}{tex/Mathematica@\TeX/Mathematica}\index{macro packages!tex/Mathematica@\TeX/Mathematica} is available from the CTAN archives in
    the directory \ctandir{macros/mathematica}.

  \ipitem [Script\TeX]

    Supports typesetting screenplays in \TeX.
    \ixx{Script\TeX}{Scripttex@Script\TeX}\index{macro packages!Scripttex@Script\TeX} is 
    available from the CTAN archives in the directory
    \ctandir{macros/scripttex}.

  \ipitem [Ver\TeX]

    Supports typesetting articles for economic journals.
    \ixx{Ver\TeX}{Vertex@Ver\TeX}\index{macro packages!Vertex@Ver\TeX} is available 
    from the CTAN archives in the directory
    \ctandir{macros/plain/contrib/vertex}.
\end{iplist}

%\subsection{\protect\INRSTeX}
%
%The \INRSTeX\ style is an extension of Plain \TeX\ by Michael Ferguson.
%It is described completely in \cite{mf:inrstexref}.
%
%The \INRSTeX\ reference is a good, free overview of \TeX.
%
%It was designed to typeset English and French.
%
%It has some apparent dependencies on ML\TeX\ and DVIALW (for graphics,
%at least).
%

\section{Special-purpose Formats}

In addition to the general-purpose packages \index{macro packages!special-purpose} discussed above,
there are dozens, if not hundreds, of extensions to \TeX\ that are
designed for very specific tasks.  Many of the extensions are
\LaTeX\ style files; they provide styles for many academic journals,
university theses, resum\'es, diagrams of various sorts, PostScript
interfaces, linguistics, multinational language support, \Unix\  
``man'' pages, program listings, and countless other tasks.

To give you a feel for the range of tasks that \TeX\ can perform, I've
selected a few packages to highlight the latitude of customization that is
possible.  Figure~\ref{fig:caffeine} shows the chemical structure of caffeine
(a molecule dear to my heart) rendered with 
the \ixx{Chem\TeX}{Chemtex@Chem\TeX}\index{macro packages!Chemtex@Chem\TeX} package.
The source is shown in Example~\ref{fig:caffeine-source}. Another chemistry
package, \idx{ChemStruct}\index{macro packages!ChemStruct}, was 
used to draw Figure~\ref{fig:lithium}.
Its source is shown in Example~\ref{fig:lithium-source}.  Taking \TeX\ in
another direction, the \ixx{Music\TeX}{Musictex@Music\TeX}\index{Musictex@Music\TeX} package 
was used to typeset the first two
bars of Mozart's K545 sonata in C-major in Figure~\ref{fig:mozart}.  The
Music\TeX\ input is shown in Example~\ref{fig:mozart-source}.  
Several more examples are presented in Chapter~\ref{chap:foreign} where
formats for typesetting non-English languages are described.

%Finally, a short
%sample of arabic, typeset with Arab\TeX, is shown in Figure~\ref{fig:arab}
%(the source is in Example~\ref{ex:arabsrc}).

%%%%%%%%%%%%%%%%%%%%%%%%%%%%%%%%%%%%%%%%%%%%%%%%%%%%%%%%%%%%%%%%%%%%%%

\epsbox[caffeine.eps]{fig:caffeine}{Caffeine by Chem\protect\TeX}

\exampleinput{caffeine.tex}{fig:caffeine-source}%
  {The Chem\protect\TeX\ Source for Caffeine}

%%%%%%%%%%%%%%%%%%%%%%%%%%%%%%%%%%%%%%%%%%%%%%%%%%%%%%%%%%%%%%%%%%%%%%

\epsbox[lithium.eps]{fig:lithium}{A lithium cation rendered
    by \protect\program{ChemStruct}}

\exampleinput{lithium.tex}{fig:lithium-source}{The ChemStruct Source for the Lithium Cation}

%%%%%%%%%%%%%%%%%%%%%%%%%%%%%%%%%%%%%%%%%%%%%%%%%%%%%%%%%%%%%%%%%%%%%%

\epsbox[mozart.eps]{fig:mozart}{A little Mozart$\ldots$}

\newpage
\exampleinput{mozart.tex}{fig:mozart-source}{The \protect\MusicTeX\ Source 
  for Figure~\protect\ref{fig:mozart}}

Another popular special-purpose application of \TeX\ is the production
of \idx{transparencies}, also 
called \idx{foils} or \idx{slides}.  There are a few
different options for this application.

\subsection{\protect\SliTeX}

\SliTeX\ is part of the standard \LaTeX\ distribution.  Input to \SliTeX\
consists of a main file and a slides file.  Individual slides
are composed in a \env{slide} environment.

\ixx{\SliTeX}{SliTeX@\SliTeX}\index{latex@\LaTeX!slide packages} has 
provisions for black-and-white slides, color slides,
and overlays.  Unlike the other slide-making formats, which rely on
\cs{special} printer commands\footnote{The $\backslash$\texttt{special}
mechanism is a way of passing arbitrary information through \TeX\ to the
\dvidriver{} that will ultimately print the document.} to incorporate color, 
\SliTeX\ produces separate output pages for each color.  For example, if you
use red to highlight words on an otherwise black-and-white slide,
\SliTeX\ will produce two output pages for the slide: one with all the
black text (excluding the words in red) and one with just the red
words.  Both of these pages will be printed in black.  You must
construct the colored slide by copying the pages onto colored
transparencies and overlaying them.  Producing slides with overlays
in the same color is accomplished with a special
``invisible'' color.  This method of producing colored transparencies
has been made obsolete by modern color printers.  Of course, nothing
prevents you from using the \cs{special} extensions of your
\dvidriver\ in \SliTeX\ to produce colored transparencies directly on
a color printer.  

\SliTeX\ has always been a separate macro package, distinct from and not
100\% compatible with \LaTeX.  With the introduction of \LaTeXe, \SliTeX\
is simply an extension package to the \LaTeXe\ core format.  Support
for a separate \SliTeX\ format is being phased out.

\subsection{\protect\FoilTeX}

\ixx{\FoilTeX}{FoilTeX@\FoilTeX} is an extension of \LaTeX\ for producing slides.  The primary
advantage of \FoilTeX\ over \SliTeX\ is that it is completely compatible
with \LaTeX.\footnote{With the caveat that it still uses the Old Font
Selection Scheme.}   Note, however, that the defaults in many cases 
are not precisely the same as \LaTeX\ because of the radically different
goal of \FoilTeX.

\FoilTeX\ provides support for running headers and footers, modified
theorem environments for better mathematics in slides, support for
AMS and PostScript fonts, and colors.

\newpage
Color slides in \FoilTeX\ are handled by \dvidriver\ \cs{special}
commands (most commonly \program{dvips} \cs{special}s).  However,
\FoilTeX\ includes a number of new and extended features for better
handling of color.  See the section ``\nameref{sec:texincolor}'' later
in this chapter for a detailed discussion of using color in \TeX.

\subsection{Seminar}

The \idx{Seminar} style is another alternative for producing slides
and notes.  Like \FoilTeX, the Seminar style is designed to work
on top of \LaTeX.  (Seminar also works with \AMSLaTeX.)\ \ The Seminar
style is designed to produce output for a PostScript printer.  It 
isn't strictly necessary to produce PostScript output, but if you do not,
many of Seminar's features will be unavailable to you.

Seminar provides for a mixture of portrait and landscape styles and
can support color using either a color-separation technique (similar
to \SliTeX's method) or direct use of PostScript color.  In either
case, the \ixx{\PSTricks}{PSTricks@\PSTricks} macro package is required.  (Consult the
section ``\nameref{sec:pictures:pstricks}'' in Chapter~\ref{chap:pictures},
{\it \nameref{chap:pictures}}, for more information.)

The Seminar style has support for a number of interesting options (including
two-up printing of slides), automatic resizing by changing a magnification
parameter, instructions for converting your \SliTeX\ slides, and several
extensive demonstration files.  Also included are explicit instructions for
placing Encapsulated PostScript drawings in your slides.

\section{\protect\TeX\ in Color}
\label{sec:texincolor}

With color printers\index{tex@\TeX!color options}\index{colors!in tex@in \TeX} and 
copiers becoming more common, the
application of color, especially in transparencies, is more important
than ever.  Unfortunately, \TeX\ knows {\em nothing\/} about color.

A little reflection about the design of \TeX\ will make it clear
why this is the case. \TeX\ produces device-independent output.
Even when color printers are as common as ``regular'' printers, if that
ever becomes the case, it will always be true that color is an inherently
device-dependent attribute.  It does not make sense for \TeX\ to
understand color.  However, this does not prevent \TeX\ from using color.   

At the lowest level, all that is required to use color in \TeX\ is some
way of telling the printer ``start printing in $<${\em color\/}$>$ here.''
This is easily accomplished with a \cs{special} command.  In the
discussion that follows, the \program{dvips} \cs{special} commands are used
as concrete examples, but conceptually, any color printer can be
used in this way.

\subsection{Setting Up Color}

Color \index{colors!in tex@in \TeX} support 
at the \dvidriver\ level is provided by \cs{special}
commands, but these are not typically convenient to enter directly
into your document.  Frequently, these commands are specified in
terms of percentages of red, green, and blue 
(RGB color) \index{colors!RGB} or 
cyan, magenta, yellow, and black (CMYK color)\index{colors!CMYK}.

Higher-level support is provided by a collection of color control sequences.
These sequences are loaded either by inputting the  file
\filename{colordvi.tex}\index{colordvi files!for Plain tex@for Plain \TeX} (in 
Plain \TeX, for example)  or using the
\filename{colordvi}\index{colordvi files!in latex@in \LaTeX} style 
file (in \LaTeX).

\program{dvips} defines the colors in terms of ``crayon names.''  If you
need very precise control of the colors, you can adjust the precise mix
of CMYK values in the file \filename{colordvi.tex} after comparing the 
output of your printer with a standard scale (typically, the PANTONE
scale).  The following color names are standard in \program{dvips}.

\begin{tabbing}{llll}
a\=CornflowerBlueaaa\=MidnightBlueaaa\=ProcessBlueaaaa\=WildStrawberry \kill
 \>Apricot         \>Emerald       \>OliveGreen   \>RubineRed      \\
 \>Aquamarine      \>ForestGreen   \>Orange       \>Salmon         \\
 \>Bittersweet     \>Fuchsia       \>OrangeRed    \>SeaGreen       \\
 \>Black           \>Goldenrod     \>Orchid       \>Sepia          \\
 \>Blue            \>Gray          \>Peach        \>SkyBlue        \\
 \>BlueGreen       \>Green         \>Periwinkle   \>SpringGreen    \\
 \>BlueViolet      \>GreenYellow   \>PineGreen    \>Tan            \\
 \>BrickRed        \>JungleGreen   \>Plum         \>TealBlue       \\
 \>Brown           \>Lavender      \>ProcessBlue  \>Thistle        \\
 \>BurntOrange     \>LimeGreen     \>Purple       \>Turquoise      \\
 \>CadetBlue       \>Magenta       \>RawSienna    \>Violet         \\
 \>CarnationPink   \>Mahogany      \>Red          \>VioletRed      \\
 \>Cerulean        \>Maroon        \>RedOrange    \>White          \\
 \>CornflowerBlue  \>Melon         \>RedViolet    \>WildStrawberry \\
 \>Cyan            \>MidnightBlue  \>Rhodamine    \>Yellow         \\
 \>Dandelion       \>Mulberry      \>RoyalBlue    \>YellowGreen    \\
 \>DarkOrchid      \>NavyBlue      \>RoyalPurple  \>YellowOrange   \\
\end{tabbing}

\subsection{Using Color}

After \program{dvips} has loaded \filename{colordvi}, typesetting text
in color is simply a matter of using the appropriate color control
sequence\index{colors!control sequences}\index{control sequences!color}.  For example, to typeset something in red, use the 
\cs{Red} control sequence in your document, like this:

\begin{shortexample}
\Red{something in red}
\end{shortexample}

Alternatively, you can change the default text color with the 
\cs{text}\texttt{\textit{color}} control sequences.  To make default
color for all text blue, enter:

\begin{shortexample}
\textBlue
\end{shortexample}

\newpage
To change the background color, use the \cs{background}
macro.  For example, to make the current and all future pages yellow,
enter:

\begin{shortexample}
\background{Yellow}
\end{shortexample}

You can enter a precise color by specifying it in terms of its CMYK
components.  The \cs{Color} and \cs{textColor} macros
exist for this purpose.  To typeset some text in a color that is
25\% cyan, 35\% magenta, 40\% yellow, and 10\% black,\footnote{I made
these numbers up.  I take no responsibility for the artistic merits
(or lack thereof) of the resulting color.} enter:

\begin{shortexample}
\Color{.25 .35 .4 .1}{some text}
\end{shortexample}

\subsection{Now I've Got Color, but I Need Black and White!}

If you have reason to print a colored \TeX\ document on a black and white
printer, you don't have to tear out all of the color commands.  \program{dvips}
includes a \filename{blackdvi} file \index{blackdvi files}(analogous 
to \filename{colordvi}\index{colordvi files}---an
input file or style file depending on your macro package), which translates
all color commands into black and white.

Alternatively, ``good'' implementations of PostScript in a black and white
printer should translate all colors into shades of grey.  This can be
an inexpensive way to preview a color document.
Most screen previewers simply ignore color commands so they print in
black and white even if the document is colored.

\subsection{Color Under \protect\LaTeXe}

At the time of this writing, 
the \LaTeXe\index{latexe@\LaTeXe!colors}\index{colors!latexe@\LaTeXe} team has 
not officially adopted a
standard for using color.  However, it is likely to follow a slightly
different model than the one described above.  The final design should provide
color selection commands that are device-independent at the \dvidriver\
level (in other words, the color commands will not insert device-specific
commands, like snippets of PostScript, into the \ext{DVI} files).

\subsection{Color Is Subtle}

Color commands implemented as \cs{special} commands may introduce
occasional problems.  For example, if \TeX\ introduces a page break in
a paragraph that you have typeset in yellow 
(\verb|\Yellow{This is a long paragraph...}|), 
the resulting output may print the page footer
(and even the header) in yellow, although that was not intended.

Circumventing these problems may require careful use of color commands
in front of text that you want to appear black.  For example, in Plain 
\TeX\ the difficulty described above can be avoided by specifying that
the page number should be printed this way:

\begin{shortexample}
\footline{\Black{\hss\tenrm\folio\hss}}
\end{shortexample}

\newpage
This definition guarantees that the page number will be set in black,
and because it is a local color change, colored text can flow across the
page around it.

You may want to make sure that other typographic elements are printed in the
current global color (which may vary).  \program{dvips} provides a 
local color macro called \cs{globalColor} for that purpose.  Every
time the text color is changed globally (with a 
\cs{text}\texttt{\textit{Color}} command), 
\cs{globalColor} is redefined to print text in that color.

\subsection{Further Reading}

Read the documentation for your \dvidriver\ carefully with respect to
color.  Because it is device-dependent, there is a lot of room for
interpretation, and it may not always be obvious why some things appear
the way they do.  And \dvidriver{}s are free to implement color in any
way they choose.

% Local Variables: 
% mode: latex
% TeX-master: "driver"
% End: 

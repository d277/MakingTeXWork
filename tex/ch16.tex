\extrachapspace=1pc
\chapter{\protect\TeX\ Utilities} 
\RCSID$Id: ch16.tex,v 1.1 2002/08/23 14:58:46 nwalsh Exp $
\label{chap:utils}

% tex-archive/systems/amiga
% tex-archive/systems/atari
% tex-archive/systems/common_tex

\ifincludechapter\else\endinput\fi

\chardef\ubarcatcode=\catcode`\_
\catcode`\_=\active\relax
\def_{\ttunder}

\def\package#1#2#3#4{%
% #1 = printable name
% #2 = exec name
% #3 = implementation
% #4 = directory
  \par\goodbreak
  {\it\bf\fontsize{11}{13pt}\selectfont #1}\par\vskip-\parskip\nopagebreak
  Directory: \textit{#4}\par\vskip-\parskip\nopagebreak
%  \vbox{
%    \hrule width \textwidth height .5pt depth 0pt
%    \vskip2pt
%    \hbox to \textwidth{\program{#1}\hss\filename{#4}}
%    \vskip-.4\baselineskip
%  }
  \nopagebreak}

\def\obsolete#1#2#3#4{}%do nothing!
\def\undescribed#1#2#3#4{}%do nothing!

This chapter offers a brief summary of a large number of \TeX\ utilities
(some more commonly used than others).
Some of these programs are mentioned elsewhere in this book in connection
with the particular tasks that they perform.  

Although many programs are described in this chapter, there is no way that it
can be entirely complete.  There are just too many programs on CTAN for me to
be familiar with {\em all\/} of them.  This list is
representative of the collection of programs present in the archives during
the winter of 1993.  Regrettably, programs not documented in English
are not included at this time.
Take the time to explore the CTAN archives yourself; you won't be
disappointed.

\begin{sidebar}{Use the Web, Luke!}
One of the most convenient online tools for searching the archives is the
World Wide Web (WWW) interface available at
\filename{http://jasper.ora.com/ctan.html}.  The WWW interface is constructed
automatically from the most recent list of files in the CTAN archives with
annotations taken from this chapter from the \filename{TeX-index}, and from
the \filename{ctan.dat} descriptions maintained by the CTAN archivists.
\end{sidebar}

This chapter attempts to describe only part of the archives.  Here are some
of the things that are {\em not\/} described in this chapter:

\begin{itemize}
  \item The thirty or more packages in the \filename{archive-tools} directory.
        This directory contains sources and executables for almost every 
        archiving tool imaginable.  If you retrieve a file from an FTP site
        and you don't have the tool necessary to unpack it, you'll probably
        find it in here.  You'll also find other archive-related programs
        like an FTP server and the \program{Gopher} sources.
  
  \item Files from the \filename{digests}, \filename{documentation}, and
        \filename{help} directories.  These are collections of electronic
        digests, documents, and articles that discuss aspects of \TeX.
        They are recommended reading.

  \item Files from the \filename{fonts} directory.  These are summarized in
        Chapter~\ref{chap:fonts}, {\it \nameref{chap:fonts}}.

  \item Files from the \filename{languages} directory.  These are summarized
        in Chapter~\ref{chap:foreign}, {\it \nameref{chap:foreign}}.

  \item The many special macro files in the \filename{macros} subtree.
        Many macro formats are described in Chapter~\ref{chap:macpack},
       {\it \nameref{chap:macpack}}.

  \item Architecture-specific applications from the \filename{systems} 
        subtree.  These include applications for Amiga, Atari, 
        Macintosh, MS-DOS, OS/2, \Unix, VMS, and Xenix systems.  The
        Common-\TeX\ distribution is also available under this directory,
        as well as the standard \TeX\ distributions by Knuth.
\end{itemize}

\section{List of Tools}

These tools are available in the CTAN archives in the directories specified.
This list is sorted by CTAN path name so that related utilities appear near
each other.

%%%%%%%%%%%%%%%%%%%%%%%%%%%%%%%%%%%%%%%%%%%%%%%%%%%%%%%%%%%%%%%%%%%%%%
\package{bib2dvi}{bib2dvi}{sh}{biblio/bibtex/utils/bib2dvi}

\program{bib2dvi} is a shell script that creates a printable listing
of an entire \BibTeX\ database.

\package{bibcard}{bibcard}{C}{biblio/bibtex/utils/bibcard}

\program{bibcard} is an X11 based database editor for \BibTeX\ databases.
This is an OpenWindows application requiring the \filename{xview} and
\filename{olgx} libraries.

\package{bibclean}{bibclean}{}{biblio/bibtex/utils/bibclean}

\program{bibclean} syntax-checks
and pretty-prints a \BibTeX\ database.

\package{bibextract}{bibextract}{}{biblio/bibtex/utils/bibextract}

\program{bibextract} extracts bibliographic entries from a list of
\BibTeX\ databases based upon a user-supplied regular expression.

\package{citefind}{citefind}{}{biblio/bibtex/utils/bibextract}

\program{citefind} extracts bibliographic entries from a list of
\BibTeX\ databases based upon a user-supplied list of keys.

\newpage
\package{citetags}{citetags}{}{biblio/bibtex/utils/bibextract}

\program{citetags} extracts citation tags from a list of \LaTeX\
source files.  These tags can be fed to \program{citefind} to produce
a bibliography database customized for a particular document.

\package{bibindex}{bibindex}{}{biblio/bibtex/utils/bibindex}

\program{bibindex} creates a bibliographic index for the \program{biblook}
program.

\package{biblook}{biblook}{}{biblio/bibtex/utils/bibindex}

\program{biblook} uses a binary index constructed by \program{bibindex} to 
perform very fast lookups into \BibTeX\ databases.

\package{bibsort}{bibsort}{}{biblio/bibtex/utils/bibsort}

\program{bibsort} sorts a \BibTeX\ database.

\package{bibtool}{bibtool}{}{biblio/bibtex/utils/bibtool}

\program{bibtool} performs a number of operations on \BibTeX\ databases.
It can pretty-print and syntax-check a database, automatically build
new keys, extract particular entries, sort a database, and perform a number
of other operations.  The operation of \program{bibtool} can be customized
with one or more resource files.

\package{aux2bib}{aux2bib}{}{biblio/bibtex/utils/bibtools}

\program{aux2bib} extracts citations from a \LaTeX\ \ext{AUX} file and
constructs a bibliography that contains only the entries required by
the document.

\package{bibify}{bibify}{}{biblio/bibtex/utils/bibtools}

\program{bibify} attempts to generate an \ext{AUX} file that contains
appropriate references for citations.  This eliminates one \LaTeX\
pass over the document and may be much faster for large documents.
\program{bibify} cannot handle documents that use multiple \ext{AUX}
files.

\package{bibkey}{bibkey}{}{biblio/bibtex/utils/bibtools}

Lists all entries in a \BibTeX\ database that contain
a particular keyword in the keyword field.

\package{cleantex}{cleantex}{}{biblio/bibtex/utils/bibtools}

Deletes temporary files (\ext{AUX}, \ext{LOF}, etc.) created by \LaTeX.

\newpage
\package{looktex}{looktex}{}{biblio/bibtex/utils/bibtools}

Lists all \BibTeX\ database entries that match a specified regular
expression.

\package{makebib}{makebib}{}{biblio/bibtex/utils/bibtools}

\program{makebib} creates a ``portable'' \BibTeX\ database by performing
string substitutions, removing comments, and optionally discarding all
entries that do not match a given list of keys.

\package{printbib}{printbib}{}{biblio/bibtex/utils/bibtools}

\program{printbib} creates a printable listing of an entire \BibTeX\ database.
\program{printbib} is also available in the
\filename{biblio/bibtex/utils/printbib} directory.

\package{bibview}{bibview}{Perl}{biblio/bibtex/utils/bibview}

\program{bibview} is an interactive Perl script that allows you to view
and search through a \BibTeX\ database.

\package{HyperBibTeX}{HyperBibTeX}{}{biblio/bibtex/utils/hyperbibtex}

\program{HyperBibTeX} is a Macintosh tool for manipulating \BibTeX\ databases.

\package{bibdestringify}{bibdestringify}{}{biblio/bibtex/utils/lookbibtex}

\program{bibdestringify} performs \BibTeX\ string substitution on a 
\BibTeX\ database.

\package{lookbibtex}{lookbibtex}{}{biblio/bibtex/utils/lookbibtex}

Lists all \BibTeX\ database entries that match a specified regular
expression.

\package{r2bib}{r2bib}{C}{biblio/bibtex/utils/r2bib}

\program{r2bib} converts \program{refer} databases into \BibTeX\
databases.

\package{ref2bib}{ref2bib}{Perl}{biblio/bibtex/utils/ref2bib}

\program{ref2bib} converts \program{refer} databases into \BibTeX\
databases.

\package{xbibtex}{xbibtex}{}{biblio/bibtex/utils/xbibtex}

\program{xbibtex} is an X11 program for manipulating \BibTeX\ databases.

\newpage
\package{Public Doman DVI Driver Family}{*}{C}{dviware/beebe}

A collection of \ext{DVI} drivers.  This code has been used as the basis
for many other \ext{DVI}-aware programs.  Table~\ref{tab:util:beebedr}
summarizes the drivers included in this package.

\begin{xtable}{l|l}
  \caption{The Public Doman DVI Driver Family\label{tab:util:beebedr}}\\
  \bf Driver & \bf Description\\[2pt]
  \hline
  \endfirsthead
  \caption[]{The Public Doman DVI Driver Family (continued)}\\
  \bf Driver & \bf Description\\[2pt]
  \hline
  \endhead
  \tstrut
  \it dvialw & PostScript (Apple LaserWriter laser printer) \\
  \it dvibit & Version 3.10 BBN BitGraph terminal \\
  \it dvica2 & Canon LBP-8 A2 laser printer \\
  \it dvican & Canon LBP-8 A2 laser printer \\
  \it dvidjp & Hewlett-Packard Desk Jet plus (from LaserJet) \\
  \it dvie72 & Epson 9-pin family 240/216-dpi dot matrix printer \\
  \it dvie72 & Epson 9-pin family 60/72-dpi dot matrix printer \\
  \it dvieps & Epson 9-pin family 240/216-dpi dot matrix printer \\
  \it dvieps & Epson 9-pin family 60/72-dpi dot matrix printer \\
  \it dvigd  & Golden Dawn Golden Laser 100 laser printer \\
  \it dviimp & imPRESS (Imagen laser printer family) \\
  \it dvijep & Hewlett-Packard LaserJet Plus laser printer \\
  \it dvijet & Hewlett-Packard 2686A Laser Jet laser printer \\
  \it dvil3p & Digital LN03-PLUS 300 dpi laser printer \\
  \it dvil3p & Digital LN03-PLUS 150 dpi laser printer \\
  \it dvil75 & DEC LA75 144h x 144v dot matrix printer \\
  \it dvil75 & DEC LA75 72h x 72v dot matrix printer \\
  \it dvim72 & Apple ImageWriter 144 dpi dot matrix printer \\
  \it dvim72 & Apple ImageWriter 72 dpi dot matrix printer \\
  \it dvimac & Apple ImageWriter 144 dpi dot matrix printer \\
  \it dvimac & Apple ImageWriter 72 dpi dot matrix printer \\
  \it dvimpi & MPI Sprinter 144h x 144v dot matrix printer \\
  \it dvimpi & MPI Sprinter 72h x 72v dot matrix printer \\
  \it dvio72 & OKIDATA Pacemark 2410 144 dpi dot matrix printer \\
  \it dvio72 & OKIDATA Pacemark 2410 72 dpi dot matrix printer \\
  \it dvioki & OKIDATA Pacemark 2410 144 dpi dot matrix printer \\
  \it dvioki & OKIDATA Pacemark 2410 72 dpi dot matrix printer \\
  \it dviprx & Printronix 300/600 60h x 72v dpi dot matrix printer \\
  \it dvitos & Toshiba P-1351 180h x 180v dpi dot matrix printer \\[2pt]
  \hline
\end{xtable}

\package{bitpxl}{bitpxl}{C}{dviware/bitpxl}

Converts HP LaserJet bitmaps into PXL files.

\package{cdvi}{cdvi}{}{dviware/cdvi}

MS-DOS \ext{DVI} file previewer.  Does not support external fonts, but
has 16 Computer Modern fonts built in.  

\package{Crudetype}{crudetype}{Web}{dviware/crudetype/version3}

A \ext{DVI}-to-text translator.  \program{Crudetype} attempts to make a
readable \ext{DVI} file.  An interactive mode is available on some platforms.
A 132-column display is expected for most output.

\package{DVItoVDU}{dvgt, dvitovdu, dvi2vdu}{C}{dviware/dvgt}

\program{DVItoVDU} is a terminal previewer for \ext{DVI} files.  The
\program{dvgt} version supports graphics preview using various graphics
terminals and limited typeset preview on other terminals such
as the VT100 and VT220.  Tektronix emulation allows
remote previewing of
\ext{DVI} files through Telnet or using Kermit from a PC, for example.

\package{DVI2PCL}{dvi2pcl}{C}{dviware/dvi2pcl}

Translates \ext{DVI} files into HP LaserJet format.  

\package{dvi2ps}{}{C}{dviware/dvi2ps}

There are a number of translators called \program{dvi2ps}.  Some appear
to be older \ext{DVI} to PostScript translators supporting the MIT
\filename{printcap} entries and Apple LaserWriter printers, while
others offer support for Asian fonts compatible with the tools used by
the Japanese \TeX\ User's Group and the \program{chfont} program,
better support for built-in printer features, and faster execution.

\package{psdvi}{psdvi}{C}{dviware/dvi2ps/psdvi}

This program translates \ext{DVI} files into resolution-independent 
PostScript files.  This allows the PostScript output to be printed on
high-resolution phototypesetting equipment.  Many other 
\ext{DVI}-to-PostScript converters assume that the output resolution is 
300dpi, which makes the PostScript more difficult, or impossible, to print
on high-resolution devices.

Because bitmapped fonts are inherently resolution-dependent, documents
that use them cannot be translated with this driver.  This means
that none of the Computer Modern or \AmS\ fonts can be used unless
you have them in PostScript Type~1 format.

\package{DVI2QMS}{dvi2qms}{C}{dviware/dvi2qms}

Translates \ext{DVI} files into output suitable for QMS 800/1200/2400 
printers.

\package{DVI2TTY}{dvi2tty}{C}{dviware/dvi2tty}

Translates \ext{DVI} files into plain ASCII text.

\newpage
\package{dvi2xx}{dvilj2, dvilj2p, dvilj, dviljp, dvi3812}{C}{dviware/dvi2xx}

Translates \ext{DVI} files into output suitable for the HP LaserJet
family of printers and compatible printers.  It also supports output
to the IBM 3812 printer.
MS-DOS and OS/2 executables are provided.

\package{dviapollo}{dviapollo}{C}{dviware/dviapollo}

A screen previewer for Apollo workstations.

\package{dvibit}{dvibit}{C}{dviware/dvibit}

A screen previewer for BBN BitGraph terminals.

\package{dvibook}{dvibook}{C}{dviware/dvibook}

Rearranges \ext{DVI} file pages into a sequence suitable for
printing and folding into a book.  Actually, it produces signatures,
which are small groups of pages that can be folded together.  A signature
is composed of several folded pages; a book is composed of many signatures
bound together.

\package{DVIChk}{dvichk}{C}{dviware/dvichk}

\program{DVIChk} examines a \ext{DVI} file and prints out the page numbers
that occur in the document in the order in which they occur.  Pages in a 
\ext{DVI} file do not have to occur in any particular order.

\package{DVICOPY}{dvicopy}{Web}{dviware/dvicopy}

This program copies a \ext{DVI} file that contains references to virtual
fonts and creates an equivalent \ext{DVI} file with all references to
virtual fonts translated into references to the appropriate non-virtual
fonts.

\package{DVIDIS}{dvidis}{C}{dviware/dvidis}

A \ext{DVI} file previewer for VaxStations running VMS.

\package{dvidoc}{dvidoc}{C}{dviware/dvidoc}

Translates \ext{DVI} files into plain ASCII.  Although it does not attempt
to form the same page breaks, it does claim to get the interword spacing
correct.

\newpage
\package{DVIDVI}{dvidvi}{C}{dviware/dvidvi}

Rearranges pages in a \ext{DVI} file.  The resulting file is a new \ext{DVI}
file containing only the selected pages.  A Macintosh executable is
available in \ctandir{dviware/dvidvi/mac}.

\package{DVIEW}{dview}{}{dviware/dview}

A very old previewer for MS-DOS.  This program relies on \ext{PXL} files
and requires the antique Almost Modern fonts to print the user
manual.

\package{DVIIMP}{dviimp}{Web}{dviware/dviimp}

Translates \ext{DVI} files into a format suitable for printing
on an Imagen printer.  This program reads \ext{GF} files rather
than \ext{PK} files.

\package{dvimerge}{dvimerge}{{\it sh}}{dviware/dvimerge}

\program{dvimerge} is a shell script (written in \program{sh}) that
uses \program{dvidvi} and \program{dviconcat} to merge two \ext{DVI}
files together.

\package{DVIMSWin}{dvimswin}{}{dviware/dvimswin}

A \ext{DVI} previewer for MS-DOS running Windows 3.1.  This program
uses \ext{PK} fonts at screen resolution.  These can be generated
by \MF\ or obtained from the \program{dvivga} distribution.

\package{dvipage}{dvipage}{C}{dviware/dvipage}

A \ext{DVI} previewer for workstations running SunView version 3.0 or
later.

\package{dvipj}{dvipj}{C}{dviware/dvipj}

This program is a modification of the \program{dvijet} driver that
supports color printing on the HP PaintJet printer.  To build
this program, you need source from the Public Doman DVI Driver Family.

\package{dvips}{afm2tfm, dvips}{C}{dviware/dvips}

The \program{dvips} program is the de facto standard \ext{DVI} to
PostScript translator.  Several versions are available, although the
most recent release seems to offer a superset of the features provided by 
all the other versions.

\newpage
\package{dvipsk}{afm2tfm, dvipsk}{C}{dviware/dvipsk}

\program{dvipsk} is a modification of \program{dvips} version 5.516 that
supports an extended path searching algorithm for \ext{PK} fonts.

\package{DVISUN}{dvisun}{C}{dviware/dvisun}

A \ext{DVI} previewer for Sun II terminals.

\package{dvitodvi}{dvitodvi}{C}{dviware/dvitodvi}

Rearranges pages in a \ext{DVI} file, producing a new \ext{DVI} file.

\package{dvitool}{dvitool}{C}{dviware/dvitool}

A \ext{DVI} previewer for workstations running SunView.

\package{dvitops}{psfont, pfbtops, aftopl, dvitops}{C}{dviware/dvitops/dvitops}

\program{dvitops} is a \ext{DVI}-to-PostScript translator.  \program{psfont}
can be used to download PostScript fonts.  \program{pfbtops} decodes
printer font binary files into printer font ASCII files.  \program{aftopl}
builds a \ext{PL} file from a PostScript \ext{AFM} file.  \ext{PL} files
can be further translated into \ext{TFM} files with the standard 
\program{PLtoTF} utility.

\package{dvitovdu}{dvitovdu}{C}{dviware/dvitovdu}

A \ext{DVI} previewer for terminals.  \program{dvitovdu} handles plain ASCII
terminals as well as some graphics terminals.  Includes VMS and \Unix\
ports.

\package{dvitovdu}{dvitovdu}{C}{dviware/dvitovdu32}

Another version of the \program{dvitovdu} \ext{DVI} previewer for
terminals.  \program{dvitovdu} handles plain ASCII terminals as well
as some graphics terminals.  This version seems to read only \ext{PXL}
files.

\package{dvitty}{dvitty}{Pascal}{dviware/dvitty}

A simple previewer for ASCII display of \ext{DVI} files.

\package{DVIVGA}{dvivga}{}{dviware/dvivga}

The \program{DVIVGA} distribution includes a program for previewing
\ext{DVI} files on PCs with EGA, VGA, or MCGA displays.
\program{DVIVGA} is derived from the Public Doman DVI Driver Family.
\pagebreak
The \program{DVIVGA} distribution also includes a full set of \ext{PK}
fonts at many resolutions: 70, 76, 84, 92, 100, 110, 121, 132, 145,
174, 208, 250, and 300dpi.  Other previewers that require \ext{PK}
fonts at screen resolution (typically near 100dpi) can also use these
fonts.

\package{DVIWIN}{dviwin}{}{dviware/dviwin}

A screen previewer for MS-DOS running Windows 3.1.  This program includes
the ability to use \emTeX\ \ext{FLI} font libraries.

\goodbreak
\package{eps}{eps}{C}{dviware/epson/eps-0.2}

Prints a \ext{DVI} file on an Epson dot-matrix printer.  Some \ext{PK}
fonts at the appropriate resolutions are included in
\ctandir{dviware/epson/epson}.

\package{ivd2dvi}{ivd2dvi}{C}{dviware/ivd2dvi}

The \TeX-\XeT\ driver produces \ext{IVD} files, which are like \ext{DVI}
files except that they use some special commands to perform reflection,
which allows them to print {\em right-to-left}.

This program translates an \ext{IVD} file into a standard \ext{DVI} file
by replacing all of the special commands with standard \ext{DVI} file
commands.

\package{dvijep}{dvijep}{C}{dviware/kane}

Another \ext{DVI}-to-HP LaserJet+ driver.

\package{dvijep\_p}{dvijep\_p}{C}{dviware/kane}

A modified version of \program{dvijep} with Centronics PP8 workarounds
\textit{removed} where they might confuse a real LaserJet+ printer.

\package{dvi2kyo}{dvi2kyo}{C}{dviware/kyocera}

A \ext{DVI} translator that produces Kyocera's native Prescribe
command language rather than relying on Kyocera's somewhat limited
HP LaserJet compatibility mode.

\package{kyodev}{kyodev}{C}{dviware/kyocera}

This \dvidriver\ produces output in Kyocera's native Prescribe format.

\package{dviplus}{dviplus}{Web}{dviware/laserjet}

An old \ext{DVI}-to-HP LaserJet Plus translator that uses \ext{PXL}
files.

\package{LN03 Driver}{ln03dvi}{C}{dviware/ln03}

Translates \ext{DVI} files into a format suitable for printing on
a Digital LN03 printer.

\package{PSPrint}{psprint}{Pascal}{dviware/psprint}

A PostScript printing program that can handle \ext{DVI} files, raw
PostScript files, or text files.  This program is a combined 
\ext{DVI}-to-PostScript translator and file-printing utility.

\Unix\ and VMS sources are provided, but a Pascal compiler is required.

\package{QMS}{dvilg8, dviqms12, ddviqms8}{}{dviware/qms}

An old VMS \ext{DVI} driver for QMS printers.

\package{QuicSpool}{pktoch, tfm2difont, tfm2ofont, dumpdesc, cati}{C}{dviware/quicspool}

This is a collection of programs for printing files to QMS and Talaris laser
printers.  Support for \ext{DVI} files is included.

\package{SCREENVIEW}{crude}{Web}{dviware/screenview}

A VMS tool based on \program{Crudetype} for previewing \ext{DVI} files
at an ASCII terminal.

\package{See\protect\TeX}{xtex, texsun, texx, iptex, mftops}{C}{dviware/seetex}

The \program{See\TeX} package contains a number of tools for working
with \ext{DVI} files.  The most substantial of these tools are the
\program{xtex} and \program{texsun} previewers for workstations
running the X Window system and the SunView window system, respectively.

\program{xtex} is unique among X11 previewers because it uses X Window
fonts.  This makes \program{xtex} quite fast.  It also means that 
\program{xtex} has to build a lot of new X Window fonts the first
few times that you use it.  The \program{mftops} program converts
\TeX\ \ext{PK} fonts into X Window fonts.  \program{xtex} uses the
\program{MakeTeXPK} program to build new \ext{PK} fonts at the
necessary resolutions.
\program{xtex} also uses \program{Ghostscript} to display the PostScript
figures in your document.  This is typically much faster than using
\program{GhostView} to preview the entire document. 
The \program{texx} previewer is a much simpler X Window previewer.
It cannot interpret PostScript, but it does recognize most of the 
\program{tpic} specials.
\program{See\TeX} also includes 
% a \program{dviselect} utility for
% selecting specific pages from a \ext{DVI} file, \program{dviconcat}
% for concatenating \ext{DVI} files together, and 
\program{iptex},
an Imagen printer driver.

\obsolete{symbolics}{symbolics}{}{dviware/symbolics}

\package{\TeX{}tool}{textool}{}{dviware/textool}

This is a \TeX\ previewer for the SunView window system.

\package{umddvi}{dvidmd, dvipr, dviselect, iptex}{C}{dviware/umddvi}

\program{dvidmd} is a \ext{DVI} previewer for DMD 5620 displays.
\program{dvipr} is a \ext{DVI} driver for Versatec printers.
\program{iptex} is a \ext{DVI} driver for Imagen printers.  Finally,
\program{dviselect} extracts pages from a \ext{DVI} file and produces
a new \ext{DVI} file containing the selected pages.

\newpage
\package{VU\protect\TeX}{vutex}{Web}{dviware/vutex}

Another VMS \ext{DVI} previewer for ASCII terminals.

\package{xdvi}{xdvi}{C}{dviware/xdvi}

A screen previewer for workstations running the X11 Window System.
MS-DOS executables, useful only if you have an X11 Window system running
on your PC, are available in the \ctandir{dviware/xdvi-dos} directory.

\package{xdvik}{xdvik}{C}{dviware/xdvik}

\program{xdvik} is a modification of \program{xdvi} that
supports an extended path searching algorithm for \ext{PK} fonts.

\package{bit2spr}{bit2spr}{C}{graphics/bit2spr}

\program{bit2spr} converts X11 \ext{XBM} bitmaps into a ``sprite'' format
that can be included directly in \TeX\ documents.  The resulting sprites
are fully portable, although they require a lot of memory, which may make
large bitmaps impractical.

\package{bm2font}{bm2font}{C/Pascal}{graphics/bm2font}

The \program{bm2font} program converts bitmapped images into \TeX,
\ext{TFM}, and \ext{PK} files.  
\program{bm2font} also generates the Plain \TeX\ or
\LaTeX\ statements necessary to insert the figure into your document.
Color images can be dithered in a number of different ways.  This
is one of the most portable ways to insert pictures into a document;
almost every available \dvidriver\ can print \TeX\ \ext{PK} fonts.

\package{fig2MF}{fig2mf}{C}{graphics/fig2mf}

Translates Fig code into \MF\ source.  This allows you to use interactive
drawing tools like \program{xfig} to construct diagrams that can be
included into your document in a portable manner with \program{fig2MF}.

\package{gnuplot}{gnuplot}{C}{graphics/gnuplot}

\program{gnuplot} is an interactive function plotting program.  Several
different types of output are supported and can be included directly
into your document.

\newpage
\package{hpgl2ps}{hpgl2ps}{C}{graphics/hpgl2ps}

This program translates HPGL Plotter language commands into encapsulated
PostScript.

\package{mactotex}{mactotex}{C}{graphics/mactotex}

This program cleans up Macintosh PostScript so it can be included
into your documents with \program{psfig}.  (Some printers have difficulty
printing PostScript output generated on a Macintosh because the PostScript
is tailored towards the Apple LaserWriter series of printers.)

\package{MFPic}{mfpic}{C}{graphics/mfpic}

\program{MFPic} is a flexible replacement for the \LaTeX\ \verb|picture|
environment.  Instead of relying on special fonts to generate pictures,
\program{MFPic} writes \MF\ code, which generates a font containing the picture
specified.  This makes \program{MFPic} more flexible than the standard
\verb|picture| environment without losing portability.

\package{pbmtopk}{pbmtopk}{C}{graphics/pbmtopk}

Translates \ext{PBM} files into \TeX\ \ext{PK} fonts.

\package{PiCTeX}{pictex}{C}{graphics/pictex}

A sophisticated macro package that works on top of Plain \TeX\
and \LaTeX.  It provides a device-independent way of producing many kinds
of figures.  The \PiCTeX\ implementation frequently generates documents that
are too complex for ``small'' versions of \TeX\ to render.  The \PiCTeX\
manual is not freely available and must be purchased before \PiCTeX\ can
be used.

\package{Glo+Index}{idxtex, glotex}{C}{indexing/glo+index}

These are tools for automatic construction of indexes and glossaries.  The
glossary building tool uses \verb|\glossary| entries from your document
in conjunction with a database of word definitions to automatically construct
a glossary.  The \program{idxtex} program provides many of the same features
as the \program{MakeIndex} program.

\package{MakeIndex}{makeindex, makeindx}{C}{indexing/makeindex}

This is the standard tool for processing \verb|\index| entries
from \LaTeX\ documents.  \program{MakeIndex} reads the entries, sorts
them, handles a number of useful features (multilevel indexes, special
sorting criteria, etc.) and produces \LaTeX\ source code, which produces
a typeset index.

\newpage
\package{RTF Translator}{rtf2text, rtf2troff, *}{C++}{support/RTF-1_06a1}

The \program{RTF Translator} package includes several programs for converting
RTF (Rich Text Format) files.  RTF files contain information about the 
structure as well as the content of a document.  Microsoft Word and
several NeXT tools can produce RTF files.

Table~\ref{tab:rtfxlate} summarizes the translators provided.  Note that
there is no \TeX\ or \LaTeX\ translator in this package at present.

\begin{xtable}{l|l}
  \caption{RTF Translators\label{tab:rtfxlate}}\\
  \bf Program  & \bf Translation \\[2pt]
  \hline
  \tstrut
  \it rtf2null   & Removes RTF codes \\
  \it rtf2text   & Translates to ASCII text \\
  \it rtf2troff  & Translates to \program{ms} macros in \program{troff}\\
  \it rtfwc      & Counts actual words in an RTF document \\
  \it rtfdiag    & Verifies the RTF parser (or the RTF document) \\
  \it rtfskel    & A skeleton for producing new translators \\[2pt]
  \hline
\end{xtable}

\package{accents}{accents}{Web}{support/accents}

\program{accents} generates a virtual font containing accented letters
arranged according to the KOI8-CS character set.  This program 
constructs the accented letters from characters
and accent marks, if necessary.  If the input font is in
Adobe Standard Encoding, \program{accents} can rearrange it into
\TeX\ Text encoding.

\package{addindex}{addindex}{C}{support/addindex}

\program{addindex} inserts \LaTeX\ \verb|\index| commands into your
document.  Presented with a list of words and index entries,
\program{addindex} reads your \LaTeX\ files and indexes every word
from the list that occurs in your document. 

\package{amSpell}{amspell}{}{support/amspell}

The \program{amSpell} program is an interactive MS-DOS spellchecker
for plain ASCII and \TeX\ documents.  

\package{atops}{atops}{}{support/atops}

A simple program to convert ASCII text files into PostScript.

\package{\auctex}{}{elisp}{support/auctex}

A comprehensive \TeX\ editing environment for GNU emacs.

\newpage
\package{basix}{basix}{tex}{support/basix}

\program{basix} is a BASIC interpreter written entirely in \TeX.
Honest. ;-)

\package{BibDB}{bibdb}{}{support/bibdb}

An MS-DOS program for editing \BibTeX\ bibliographic databases.

\package{brief_t}{}{}{support/brief_t}

The \program{brief_t} package is a \LaTeX\ editing environment for
Borland International's \program{Brief} editor.

\package{c++2latex}{c2latex, c++2latex}{C}{support/c++2LaTeX-1_1}

This program parses your C or C++ programs and creates \LaTeX\ source
code for pretty-printing them.  The syntactic elements of the source
(keywords, identifiers, comments, etc.) can be set in different fonts.
Compiling this program requires \program{flex}.

\package{C2\LaTeX}{c2latex}{}{support/c2latex}

\program{C2\LaTeX} is a filter designed to provide simple literate programming
support for C programming.  This filter massages \LaTeX-coded C comments 
and source code into a \LaTeX\ document.

\package{chi2tex}{chi2tex}{}{support/chi2tex}

\program{chi2tex} is an MS-DOS program for converting \program{ChiWriter}
documents into \TeX.

\package{detex}{detex, texspell}{C}{support/detex}

\program{detex} removes \TeX\ control sequences from a document.  The
\program{texspell} script pipes the resulting output through a spell
checker.  

\package{bibtex-mode, web-mode}{bibtex-mode.el, web-mode.el}{elisp}{support/emacs-modes}

GNU emacs macros that provide an enhanced editing environment for
\BibTeX\ databases and \program{Web} source files.  Another substantial
set of modes for editing \LaTeX\ documents is provided by \auctex,
listed separately.

\package{flow}{flow}{C}{support/flow}

\program{flow} reads a plain text description of a flow chart and 
produces the appropriate \LaTeX\ \verb|picture| environment for
printing the flow chart.

\newpage
\package{byte2tex}{byte2tex}{C}{support/foreign/byte2tex}

A translator for multilingual documents that use the upper range of
the ASCII character set as a second alphabet.  The translation performed
is controlled by a plain text description of the alphabet.  See also
\program{EDI} and \program{cyrlatex}.

\package{cyrlatex}{cyrlatex}{tex}{support/foreign/cyrlatex}

A \LaTeX\ style for using the \AmS\ Cyrillic fonts as a second alphabet
under \LaTeX.  See also \program{EDI} and \program{byte2tex}.

\package{EDI}{edi}{}{support/foreign/edi}

An MS-DOS editor for multilingual (particularly Cyrillic) documents.
\program{EDI} uses the upper range of the ASCII character
set as a programmable second alphabet.  See also \program{byte2tex}
and \program{cyrlatex}.  MS-DOS and Atari executables are available.

\package{genfam}{genfam, modern}{Perl}{support/genfam}

The \program{genfam} script reads a configuration file that describes
a set of fonts and then runs the appropriate
\MF\ commands to produce them.  A sample configuration file,
\filename{modern}, is provided for the Computer Modern Roman family.

\package{Ghostview}{ghostview}{C}{support/ghostview}

\program{Ghostview} provides an interactive interface to \program{Ghostscript},
the GNU PostScript interpreter.  \program{Ghostview} requires X11, but 
there are ports to Microsoft Windows and OS/2.

\package{Stanford GraphBase}{}{CWeb}{support/graphbase}

This is a collection of \program{CWeb} programs for studying combinatorial
algorithms.  It is related to \TeX\ only because \program{CWeb} files are
part \TeX.

\package{HPTFM2PL}{hptfm2pl, showsym}{}{support/hp2pl}

This program reads Hewlett-Packard Tagged Font Metric files and
creates \TeX\ \ext{PL} files.  The \ext{PL} files can be further
translated into \TeX\ \ext{TFM} files with the standard utility
\program{PLtoTF}.

Hewlett-Packard distributes metric information about LaserJet built-in
fonts and font cartridges in Tagged Font Metric format, so this program
gives you a way of getting \TeX\ metrics for those fonts.
To actually use them, you must have both the metrics and a \dvidriver\
that can use built-in fonts.

\newpage
\package{HP2\protect\TeX}{hp2tex, hpopt}{}{support/hp2tex}

\program{HP2\TeX} translates HPGL files into \TeX\ documents that use
the \verb|\special| commands found in \emTeX\ for drawing lines at arbitrary
angles.  This makes diagrams in HPGL format that use only straight lines 
printable directly with \TeX.  See also \program{HP2XX}.

\package{HP2XX}{hp2xx}{}{support/hp2xx}

\program{HP2XX} converts HPGL diagrams into PostScript or one of several
bitmapped formats.  It can also translate some diagrams directly into
\TeX\ if \verb|\special| commands are available for drawing lines at
any angle.

MS-DOS, Sparc, Convex 210, and HP9000 executables are available.
A Windows front-end is also available for MS-DOS.

\package{HPtoMF}{hptomf}{}{support/hptomf}

\program{HPtoMF} converts HPGL (Hewlett-Packard's plotter language) into
a variety of other vector and raster formats.

\undescribed{HTML}{html}{}{support/html}

\package{HTMLto\LaTeX}{html2latex}{}{support/html2latex}

This program converts HTML documents into \LaTeX\ source for printing.

\package{Icons}{}{}{support/icons}

This is a collection of icons for \TeX\ and \MF.  They were designed by
Donald Knuth for OpenWindows and have subsequently been translated into
Microsoft Windows, OS/2, and X11 XBM format.

\package{Imake-\TeX}{imaketex}{}{support/imaketex}

\program{Imake-\TeX} helps you create an Imakefile for \TeX\ documents.
Imakefiles are used to generate custom makefiles for an application,
in this case a \LaTeX\ document.

\package{ispell}{ispell}{C}{support/ispell}

An interactive spellchecker.  \program{ispell} is intelligent about
\TeX\ documents and can handle languages other than English.
\editorial{there are two!}

\package{jspell}{jspell}{C}{support/jspell}

\program{jspell} is an MS-DOS spellchecker.  It includes special support
for \TeX\ documents.

\newpage
\package{\TeX{}Tools}{textools}{}{support/kamal}

The \program{\TeX{}Tools} package includes several tools for manipulating
\TeX\ documents. \program{detex} strips \TeX\ commands from a file.
\program{texeqn} extracts displayed equations from a document. 
\program{texexpand} merges documents loaded with \verb|\input| or
\verb|\include| into a single file. \program{texmatch} checks for
matching delimiters.  \program{texspell} uses \program{detex} to 
spellcheck the filtered version of a document.

\package{LaCheck}{lacheck}{C}{support/lacheck}

\program{LaCheck} reads a \LaTeX\ document and reports any syntax
errors it finds.  This is faster than running the document through
\TeX\ if you are only interested in finding \TeX{}nical mistakes.

\package{\LaTeX2HTML}{latex2html}{}{support/latex2html}

Converts \LaTeX\ documents into HTML documents, suitable for
browsing with a WWW browser.  Many complex document elements are handled
automatically.

\package{\LaTeX{}Mk}{latexmk}{}{support/latexmk}

\LaTeX{}Mk is a \program{Perl} script that attempts to determine what operations
need to be performed on a document to produce a complete 
\ext{DVI} file.  \LaTeX{}Mk\ runs \LaTeX, \BibTeX, etc. until a complete
document has been built.

\package{lgrind}{lgrind}{}{support/lgrind}

\program{lgrind} formats program sources using \LaTeX.  Syntactic elements
are identified by typographic changes in the printed sources.

\package{LSEdit}{}{}{support/lsedit}

This is an add-on package for the VAX/VMS Language Sensitive Editor
(\program{LSEDI}).  It provides an editing environment for \LaTeX\
documents.

\package{make\_\hskip.25pt latex}{make\_latex}{}{support/make\_latex}

Provides a set of \program{make} rules for \LaTeX\ documents.

\package{MakeProg}{makeprog}{Web}{support/makeprog}

\program{MakeProg} is a system for doing Literate Programming in \TeX.
It provides a mechanism for combining documentation with \TeX\ macros.
Donald Knuth's original articles on Literate Programming are included
with the documentation.

\package{MathPad}{mathpad}{}{support/mathpad}

MathPad is an X11 editor for \TeX\ sources.  It was not available (or known
to me) in time for review in this edition of \textit{Making \TeX\ Work}.

\package{MC\TeX}{}{tex}{support/mctex}

A \TeX\ macro package for pretty-printing \program{Lisp}
code.

\package{MEW\LaTeX}{}{}{support/mewltx}

An extension to \program{microEMACS} for Windows.
It provides a \LaTeX\ editing environment that allows you to write,
spellcheck, process, and view your document from within \program{microEMACS}.

\package{MNU}{}{DOS}{support/mnu}

\program{MNU} is an MS-DOS menu system written with batch files.  It was
designed for running \TeX\ but can be extended to other applications.

\package{pbm2\TeX}{pbm2tex}{}{support/pbm2tex}

\program{pbm2\TeX} converts PBM files into \LaTeX\ \texttt{picture} environments.  This
provides a portable way to include bitmap images in documents.  The
documents take a long time to typeset, however, and may require a 
big \TeX.

\package{PCWri\TeX}{}{}{support/pcwritex}

Adds \TeX\ to \program{PC-Write} as a special kind of printer.  Documents
``printed'' to the \TeX\ printer can be processed with \TeX\ to get
typeset output.

\package{PM\TeX}{pmtex}{}{support/pmtex}

An OS/2 Presentation Manager application that provides a shell around
common \TeX-related activities (editing, \TeX{}ing, previewing, printing,
etc.).  

\package{PP}{pp}{}{support/pp}

A Pascal or Modula-2 pretty-printer.  It translates code into a 
Plain \TeX\ or \LaTeX\ document.  An MS-DOS executable is provided.

\package{PS2EPS}{ps2eps}{c}{support/ps2eps}

Attempts to convert PostScript output from other programs into
encapsulated PostScript output suitable for including in \TeX\
documents with \program{dvips}, for example.  Support is included for
converting output from \program{GEM} 3.0, \program{DrawPerfect} 1.1, 
\program{PSpice} 4.05, \program{OrCAD} 3.11, \program{PrintGL} 1.18, 
\program{Mathcad} 3.0, and \program{GNUPlot} 3.0.

\package{PS Utils}{psbook, psselect, pstops, psnup, epsffit}{C}{support/psutils}

A collection of utilities for transforming PostScript output files in
various ways.  \program{psbook} rearranges pages into signatures.
\program{psselect} selects ranges of pages. \program{pstops} performs
arbitrary rearrangement and selection of pages. \program{psnup}
prints multiple pages on a
single sheet of paper.  \program{epsffit} rescales an encapsulated
PostScript figure to fit within a specified bounding box.
These programs can be used to rearrange the output from PostScript
\dvidriver{}s like \program{dvips}.

\program{PS Utils} also includes scripts for displaying character
information and ``fixing'' PostScript output from other sources.
(Some PostScript output is nonstandard and must be fixed-up before
it can be used with the \program{PS Utils}.)
These scripts are summarized in Table~\ref{tab:psutils:scripts}.

\begin{xtable}{l|l}
  \caption{Additional Scripts in PS Utils 
    \label{tab:psutils:scripts}}\\
  \bf Script     & \bf Purpose \\[2pt]
  \hline
  \tstrut
  \it getafm   & Retrieves font metrics from the printer \\
  \it showchar & Prints a character and its metrics \\
  \it fixfmps  & Fixes output from \program{FrameMaker} \\
  \it fixwpps  & Fixes output from \program{WordPerfect} \\
  \it fixwfwps & Fixes output from \program{Word} for Windows \\
  \it fixmacps & Fixes Macintosh PostScript \\
  \it fixpsditps & Fixes \program{Transcript} \filename{psdit} files \\
  \it fixpspps & Fixes output from \program{PSPrint} \\[2pt]
  \hline
\end{xtable}

\package{rtf2\LaTeX}{rtf2latex}{C}{support/rtf2LaTeX}

A conversion program to translate RTF sources into \LaTeX.

\package{rtf2\TeX}{rtf2tex}{C}{support/rtf2TeX}

A conversion program to translate RTF sources into \TeX.

\package{RTF-to-\LaTeX}{rtflatex}{Pascal}{support/rtflatex}

Another conversion program to translate RTF sources into \LaTeX.

\package{S2\LaTeX}{s2l}{C}{support/s2latex}

A \program{Scribe}-to-\LaTeX\ converter.  Requires \program{lex}
and \program{yacc} to compile.

\newpage
\package{Scheme\TeX}{schemetex}{C}{support/schemetex}

\program{Scheme\TeX} provides support for Literate Programming in 
\program{Scheme}.  \program{Lex} is required to build \program{Scheme\TeX}.

\package{\TeX{}OrTho}{texortho}{C}{support/spelchek}

\program{\TeX{}OrTho} is a spellchecking filter for text files.  It 
was designed for spellchecking \TeX\ and \LaTeX\ documents, but it
is parameter-file driven so it may be possible to extend it to other
formats.

\package{Tek2EEPIC}{tek2eepic}{C}{support/tek2eepic}

A filter for Tektronix 4015 escape sequences that allows graphics output
designed for a Tektronix 4015 display to be included in a \TeX\
document.  The resulting \ext{DVI} file must be processed by a \dvidriver\
that recognizes the \textit{tpic} \verb|\special| commands.

\package{\TeX{}calc}{}{}{support/texcalc}

\program{\TeX{}calc} is an \program{InstaCalc} spreadsheet for calculating
\TeX\ typeface design sizes based upon magnification and scaling factors.

\package{\TeXinfo{}2HTML}{texi2html}{Perl}{support/texi2html}

Translates GNU \TeXinfo\ sources into HTML. 

\package{\TeX{}i2roff}{texi2roff}{C}{support/texi2roff}

This is a \TeXinfo-to-\program{nroff} converter (although it should
be possible to process most documents with \program{troff} as well).

\package{\TeX{}it}{texit}{Perl}{support/texit}

A \program{Perl} script for processing \TeX\ documents intelligently.

\package{texproc}{texproc}{C}{support/texproc}

\program{texproc} is a filter that inserts the output from a
command into your document.  This can be used, for example, to 
automatically insert the output from \program{GNUPlot} directly
into your \TeX\ document (assuming that you use one of the
\TeX-compatible output terminals in \program{GNUPlot}).

\newpage
\package{tgrind}{tgrind}{C}{support/tgrind}

\program{tgrind} is a source code pretty-printer modelled after the
standard BSD \Unix\ \program{vgrind} utility.  It produces a \TeX\
or \LaTeX\ document that typesets an attractive program listing
from your source code.

\package{tr2\LaTeX}{tr2latex}{C}{support/tr2latex}

Translates \program{troff} sources into \LaTeX.

\package{tr2\TeX}{tr2tex}{C}{support/tr2tex}

Translates \program{troff} sources into Plain \TeX.

\package{TRANSLIT}{translit}{C}{support/translit}

\program{TRANSLIT} transliterates ASCII character codes.  Single characters
can be translated into multiple characters and vice versa, in addition
to simple permutations.  This program allows files written in one 
national character set to be translated into another character set
without losing characters or meaning.

\package{TSpell}{tspell, chk, putback}{C}{support/tspell}

A spellchecking filter for \TeX\ documents.  It strips \TeX\ and
\LaTeX\ macros out of a document, runs the document through a spellchecker,
and restores the \TeX\ and \LaTeX\ macros.  It looks like it was designed
for VAX machines.

\package{umlaute}{umlaute}{\protect\TeX}{support/umlaute}

A collection of style files for processing \TeX\ documents with accented
characters using machine-native character encodings.  Support for the
ISO standard character set (ISO Latin1) and the MS-DOS code page 437
character set is provided.

\package{undump}{undump}{C}{support/undump}

\program{undump} combines a core dump and an executable program to
build a new executable using the core dump to provide initial values
for all of the program's static variables.  One use of this program
is to construct \TeX\ executables with macro packages preloaded.
This is unnecessary on most modern computers (they're fast enough
to simply load the format files).  \program{undump} is no longer
a supported program.

\newpage
\package{untex}{untex}{C}{support/untex}

Removes all \LaTeX\ control sequences from a document.  Can optionally
remove arguments to control sequences, as well as all math-mode text.
Replaces many accented characters with their IBM OEM character set
equivalents.

\package{VMSSpell}{vmsspell}{}{support/vmsspell}

A VMS spellchecking tool.  Distributed in a VMS-style archive package.

\package{Vor\TeX}{}{Emacs lisp}{support/vortex}

A GNU emacs editing mode for \TeX\ documents.  Also provides support for
\BibTeX\ databases.

\package{windex}{windex}{C}{support/windex}

\program{windex} is an aid for building indexes in \LaTeX\ documents.
It modifies the \LaTeX\ indexing macros to produce a different output
format.  The \program{windex} program can then sort the terms and
construct an index for you.

\package{WP2\protect\LaTeX}{wp2latex}{C/Pascal}{support/wp2latex-5_1}

This program attempts to translate WordPerfect v5.1 documents into
\LaTeX.

\package{xet}{xet}{C}{support/xet}

The \program{xet} program removes all commands and mathematical formulae
from Plain \TeX\ and \LaTeX\ documents.  \program{xet} has a number of options
for handling different aspects of a document (such as accents and
\LaTeX\ environments) and claims to be useful as a simple syntax checker
for \TeX\ files.

\package{xetal}{xetal}{C}{support/xetal}

This is a more recent version of \program{xet}, described above.

\package{xlatex}{xlatex}{C}{support/xlatex}

An X11 Windows application that ties together several
aspects of \TeX{}ing a document.  It provides a ``push button'' interface
to editing, processing, previewing, and printing your document.

%%%%%%%%%%%%%%%%%%%%%%%%%%%%%%%%%%%%%%%%%%%%%%%%%%%%%%%%%%%%%%%%%%%%%%
\catcode`\_=\ubarcatcode \let\ubarcatcode\relax


\extrachapspace=.6pc
\chapter[Commercial Environments]{Commercial\\Environments}
\RCSID$Id: ch14.tex,v 1.1 2002/08/23 14:58:46 nwalsh Exp $
\label{chap:commercialtex}

\ifincludechapter\else\endinput\fi

There are several commercial versions\index{tex@\TeX!commercial implementations} of \TeX\ on the market.  The advantage of commercial
programs over free ones is that they provide an automated installation
procedure and some level of technical support.  There's no reason to believe
that a commercial implementation of \TeX\ is necessarily better than a free
one (or vice versa, for that matter) because the source code for \TeX\ is in
the public domain.

The following sections highlight several commercial implementations of
\TeX.

\section{\protect\utex\ by ArborText}

\idx{ArborText} sells \ixx{\utex}{utex@\utex} for MS-DOS\index{MS-DOS!TeX implementations@\TeX\ implementations} systems as well as an implementation of
\TeX\ for \Unix\ workstations.  The ArborText previewer and \dvidriver\
software are also available for both \Unix\ and MS-DOS systems.  The following
discussion is based on experiences with \utex\ version 3.14B, the MS-DOS
implementation of ArborText's \TeX\ package.

The \utex\ package is a complete \TeX\ implementation for MS-DOS.  The
additional utilities, \ixx{\program{PLtoTF}}{PLtoTF},
\ixx{\program{TFtoPL}}{TFtoPL}, and
\ixx{\BibTeX}{bibtex@\BibTeX}, are also included.  Neither \MF\ nor
\program{MakeIndex} is provided.

The default installation of \utex\ creates format files for Plain \TeX\ and
\LaTeX.  To be as fast as possible, \utex\ creates executable
versions of \TeX\ for each of these formats; these versions have the
appropriate macro packages preloaded into the executable.  You can use the
standard notation to load alternate macro packages if you wish to save disk
space.

ArborText includes a lot of support for using \TeX\ for non-English
documents\index{international!support by ArborText}.  Several files of
international hyphenation patterns (English, French\index{French}, German\index{German},
Dutch\index{Dutch}, and Portuguese\index{language considerations}) are provided, as well as reprints of the TUGboat articles
describing new features of \TeX\ version 3.0 and virtual fonts.

Designed for an MS-DOS environment having only limited memory, \utex\ uses a
swap file (potentially located in EMS\index{EMS memory} or \idx{XMS memory})
to process large files.  Several environment variables can be used to alter
the size of internal \TeX\ data structures when a new format is being
produced.  In terms of big and small versions, \utex\ can be configured either
way.  These choices are made when you create the format file however, so you
will need to use \ixx{ini\TeX}{initex@ini\TeX} (included with \utex) to make a
``bigger'' \TeX.

One of the most interesting features of \utex\ is a quick-and-dirty preview
mode provided by the \TeX\ executable.  After at least one page of your
document has been processed, you can switch to a preview mode and see what the
page looks like while \TeX\ continues processing your document.  This
previewer is not particularly attractive, but it's quick and easy to use.  No
other previewer is provided.  ArborText sells a fully functional previewer as
a separate package.

To print your documents, you must purchase or obtain a \dvidriver\
separately.  ArborText sells two drivers, one for HP LaserJet printers and
another for PostScript printers.  The \utex\ distribution includes a complete
set of \ext{TFM} files, but it does not include any \ext{PK} files.

The ArborText previewer is summarized in the section called
``\nameref{sec:arb:preview}'' in Chapter~\ref{chap:preview}, {\it
\nameref{chap:preview}}.  The ArborText printer drivers,
\program{DVILASER/HP}\index{DVILASER/HP drivers} and
\program{DVILASER/PS}\index{DVILASER/PS drivers}, are described in
Chapter~\ref{chap:printing}, {\it \nameref{chap:printing}\/}.

\section{\protect\yyTeX}

\ixx{\yyTeX}{yyTeX@\yyTeX} is an MS-DOS\index{MS-DOS!TeX implementations@\TeX\ implementations} implementation of \TeX\ by \ixx{\YY\ Inc.}{YY Inc.@\YY\ Inc.} 
(a 386 or higher processor is required).  \yyTeX\ is
sold as a bundled package with
\ixx{\program{DVIWindo}}{DVIWindo},
\ixx{\program{dvipsone}}{dvipsone},\footnote{\program{DVIWindo}
is described in Chapter~\ref{chap:preview},
and \program{dvipsone} is described in Chapter~\ref{chap:printing}.} 
and a set of PostScript fonts (either the
Computer Modern fonts, the Lucida Bright and Lucida New Math fonts, or the
MathTime fonts).  The installation provides format files for Plain \TeX\ and
\LaTeX.

\yyTeX\ stands out among \TeX\ systems because of its memory management.
Unlike other \TeX\ systems, which exist in big and small versions, all of
\yyTeX's memory management is dynamic.  All of the buffers that \TeX\ uses
(main memory, font memory, string memory, etc.) will expand to meet the needs
of the most complex documents you create.  The memory required for hyphenation
patterns is dynamically allocated as well, which means that you can construct
multilingual formats with as many sets of hyphenation patterns as you need
(multilingual formats are discussed in more detail in
Chapter~\ref{chap:foreign}, \textit{\nameref{chap:foreign}}).  The advantage
of dynamic memory management over a fast and big \TeX\ is that \yyTeX\ starts
out small (big \TeX{}s reserve a large fixed amount of memory even when
processing simple jobs that don't require very much).  In a multitasking
environment where several programs are competing for memory, one large program
can slow the progress of the entire system.  \yyTeX's memory manager supports
XMS\index{XMS memory}, VCPI\index{VCPI memory}, and DPMI\index{DPMI memory}.
This means that it can run under MS-DOS with any of the common memory managers
and under Windows, Windows NT, and DOS sessions in OS/2\index{Microsoft Windows!TeX implementations@\TeX\ implementations}\index{Microsoft Windows NT!TeX implementations@\TeX\ implementations}\index{NT TeX implementations@NT \TeX\ implementations}.

\yyTeX\ has a couple of additional strengths for multilingual use: it supports
customizable input character translation and more than 255 internal font
numbers. Most versions of \TeX\ only support a maximum of 255 fonts in any
single document.

\section{Textures}

\ixx{\program{Textures}}{Textures} is a commercial
implementation of \TeX\ for the Macintosh\index{Macintosh!TeX implementations@\TeX\ implementations} distributed by \idx{Blue Sky Research}.
It is described fully in Chapter~\ref{chap:mac}, {\it \nameref{chap:mac}}.

\section{Turbo\protect\TeX}
\label{sec:com:turbo}

\ixx{Turbo\TeX}{TurboTeX@\TurboTeX} is a complete \TeX\ system for 
MS-DOS\index{MS-DOS!TeX implementations@\TeX\ implementations}\index{Microsoft Windows!TeX implementations@\TeX\ implementations} with full
support for Microsoft Windows.  It is distributed by the \idx{Kinch Computer Company}.
The programs included in Turbo\TeX\ version 3.0 are shown in
Table~\ref{tab:turbotex}.
The \program{dvideo} and \program{wdviwin} previewers are described
in Chapter~\ref{chap:preview}.

{\def\x{${}^1$}
\begin{xtable}{l|l}
  \caption{Turbo\protect\TeX\ Programs\label{tab:turbotex}}\\
  \bf Program & \bf Description \\[2pt]
  \hline
  \tstrut     
  \it dvialw\,\x& A PostScript \dvidriver \\
  \it dvideo  & An EGA (non-Windows) previewer \\
  \it wdviwin & A Windows previewer \\
  \it dvielq\,\x& An Epson LQ driver \\
  \it dvieps\,\x& An Epson/IBM graphics printer driver \\
  \it dvijep\,\x& An HP LaserJet PLus/Series II driver \\
  \it dvijet\,\x& An HP LaserJet/DeskJet driver \\
  \it dvilj4\,\x& An HP LaserJet 4 driver \\
  \it initex\,\x& ini\TeX\ for building new format files \\
  \it latex\,\x & \TeX\ with the \LaTeX\ format preloaded \\
  \it tex\,\x   & \TeX\ with the Plain format preloaded \\
  \it texidx  & The GNU \TeX\ indexing program \\
  \it tif2hp  & Converts TIFF bitmapped graphics to HP LaserJet bitmaps \\
  \it virtex\,\x& \TeX\ with no format preloaded \\[2pt]
  \hline
  \multicolumn{2}{l}{%
    ${}^1$\vrule height11pt width0pt\tiny Windows version included.}
\end{xtable}
}

\newpage
The Microsoft Windows version of Turbo\TeX\ is a superset of
the DOS version (you must install both, but you can delete the non-Windows
executable programs that have Windows versions, if you wish).  

\TurboTeX\ includes a complete \MF\ distribution.  In addition to the 
programs shown in Table~\ref{tab:turbomf}, the complete sources for
the Computer Modern, \LaTeX, and \AmS-fonts are also provided.

{\def\x{${}^1$}
\begin{xtable}{l|l}
  \caption{Turbo\protect\TeX\ \protect\MF\ Programs\label{tab:turbomf}}\\
  \bf Program   & \bf Description \\[2pt]
  \hline
  \tstrut
  \it gftodvi\x & Creates proof sheets of \MF\ characters \\
  \it inimf     & \MF\ with no base file preloaded \\
  \it mfscript  & A script generator (for building multiple fonts) \\
  \it mfcga     & \MF\ for CGA displays \\
  \it mfhga     & \MF\ for Hercules displays \\
  \it gftopk\x  & Converts \ext{GF} files to \ext{PK} files \\[2pt]
  \hline
  \multicolumn{2}{l}{%
    ${}^1$\vrule height11pt width0pt\tiny A standard, or otherwise 
    freely available, utility.}
\end{xtable}
}

At the time of this writing, a new version of Turbo\TeX\ is actively being
developed.  Unfortunately, it was not available for review in time for this
edition of \textit{Making \TeX\ Work}.  The Kinch Computer Company claims that
the next release will include a big \TeX\ running under Windows 3.1/Win32s and
Windows NT as a native, protected-mode 32-bit application.  A new Windows
previewer will also be available, and it takes advantage of Windows ``multiple
document interface'' to preview multiple documents in the same session.  Kinch
will also provide a set of the standard Computer Modern Fonts in TrueType and
Type 1 formats.

\section{\protect\PCTeX}
\label{sec:pctex}

\ixx{\PCTeX}{PCTeX@\PCTeX} is a complete \TeX\ system for 
MS-DOS\index{MS-DOS!TeX implementations@\TeX\ implementations} distributed by \ixx{Personal \TeX, Inc.}{Personal tex, Inc.@Personal \TeX, Inc.}  It is one of two
offerings from PTI.
The version for Windows is described later in this chapter.

\PCTeX\ is sold in two configurations.  The \PCTeX\ Starter System
includes the latest releases of \PCTeX\ and \ixx{\PCTeX/386}{PCTeX@\PCTeX!PCTeX/386@\PCTeX/386}, 
\ixx{\program{PTI View}}{PTI View} (for previewing), one printer driver with the appropriate
Computer Modern fonts in \ext{PK} format, {\it \LaTeX\ For
Everyone}~\cite{jh:latexforeveryone}, and the sources for Plain \TeX,
\LaTeX, and \AMSTeX.  The \PCTeX\ Laser System includes Big
\PCTeX/386, LaserJet, LaserJet 4, PostScript and DeskJet printer
drivers (along with a complete set of Computer Modern fonts in \ext{PK}
format at 300dpi and 600dpi), and the {\it PC \TeX\ Manual}, in addition to 
everything contained in the Starter System.

The discussion that follows is based on experiences with the \PCTeX\
Starter System with \PCTeX\ version 3.14 and the \program{PTI Laser/HP} 
\dvidriver.

\newpage
In addition to the \TeX\ executables and macro packages mentioned above,
\PCTeX\ includes these standard \TeX\ utilities: \ixx{\BibTeX}{bibtex@\BibTeX}, 
\ixx{\program{PLtoTF}}{PLtoTF}, 
\ixx{\program{TFtoPL}}{TFtoPL}, \ixx{\program{PXtoPK}}{pxtopk}, 
and \ixx{\program{PKtoPX}}{pktopx}.  It does not
include any of the other standard utilities or the \program{MakeIndex}
processor.  \MF\ is available as a separate package (although it is
being discontinued, so it may not be available for long).

Several additional features make \PCTeX\ an attractive commercial
alternative:

\begin{itemize}
  \item It uses a text-based ``menu system'' for processing documents.  
        This program
        makes the edit/format/preview cycle a simple matter of selecting the
        appropriate menu item.  Figure~\ref{fig:pcmenu} shows this system
        in use.

        \epsimage{fig:pcmenu}{The \protect\PCTeX\ menu system}  

  \item Support is provided for Bitstream fonts\index{fonts!Bitstream}.  
        \idx{Bitstream} is a vendor of high-quality
        scalable typefaces.  The utilities included with \PCTeX\ allow you
        to create \ext{TFM} and \ext{PK} files from Bitstream compressed
        outline fonts.
  \item If you prepare multilingual documents, you'll appreciate the 
        attention they receive in the \PCTeX\ documentation.  Although
        the ability to compose multilingual documents is really a feature
        of \TeX\ version 3, not \PCTeX\ in particular, PTI provides 
        step-by-step instructions for building a multilingual format file.
        Support files for English, French\index{French}, and Spanish\index{Spanish} are included.
\end{itemize}

PTI's previewer, \program{PTI View}, is described in
Chapter~\ref{chap:preview}, {\it\nameref{chap:preview}}.  The
\program{PTI Laser/HP} and \program{PTI Laser/PS} printer drivers are
discussed in Chapter~\ref{chap:printing}, {\it\nameref{chap:printing}}.
Drivers for DeskJet and Epson printers are also available.

\section{\protect\PCTeX\ For Windows}
\label{sec:pctexforwin}

\ixx{\PCTeX\ For Windows}{PCTeX@\PCTeX!For Windows}\index{Microsoft Windows!TeX implementations@\TeX\ implementations} is a new offering from \ixx{Personal \TeX, Inc.}{Personal tex, Inc.@Personal \TeX, Inc.}  \PCTeX\ For Windows is a superset of Personal \TeX's
MS-DOS version of \PCTeX.

This version of \PCTeX\ is an integrated system with a built-in editor (with
complete on-line help),
previewer, and push-button access to \PCTeX\ for composing your documents.
Figure~\ref{fig:wpctex} shows \PCTeX\ For Windows in action.

\epsimage{fig:wpctex}{\protect\PCTeX\ For Windows editing and previewing 
        a \protect\LaTeX\ document} 

\program{\PCTeX\ For Windows} includes a complete set of Computer Modern
and \AmS\ fonts in scalable TrueType format.  This allows \PCTeX\ to 
print directly to any printer (or other device) supported by Microsoft 
Windows.  To provide better access to other TrueType fonts that
you may have installed, \PCTeX\ includes a \ext{TFM} generator for
TrueType fonts.  This generator is shown in Figure~\ref{fig:wpctextttotf}.

\epsimage{fig:wpctextttotf}{\protect\PCTeX\ For Windows TrueType font 
  metric builder}

Except for the \program{PXtoPK} and \program{PKtoPX} utilities,
\program{\PCTeX\ For Windows} has all of the features of the MS-DOS
version of \PCTeX.

\newpage
\section{Scientific Word}

\ixx{\program{Scientific Word}}{Scientific Word@Scientific Word (program)}, a commercial package from 
\idx{TCI Software Research}, bridges the gap between \TeX\ and word processing
by providing a more or less WYSIWYG interface on top of \TeX.
\program{Scientific Word} requires Microsoft 
Windows\index{Microsoft Windows!TeX implementations@\TeX\ implementations}.
Figure~\ref{fig:sciword} shows a sample document being edited by
\program{Scientific Word}.

\epsimage{fig:sciword}{Editing a document with \protect\program{Scientific Word}} 

\program{Scientific Word} is a powerful program that might prove to
be invaluable in some circumstances.  An ideal candidate for 
\program{Scientific Word} is someone with little or no desire to learn
\TeX, but who wants professional quality output for documents
containing a lot of mathematics.  \program{Scientific Word} is not simply
a fast, interactive previewer; it really is a visual editor that
produces \TeX\ code behind the scenes.  You enter document elements
(even complex elements like mathematics) in an interactive push-button
fashion, and \program{Scientific Word} translates it into the appropriate
\LaTeX\ input.
This is very different
from Blue Sky Research's \program{Textures} where the user types in 
\TeX\ but has nearly instantaneous feedback.

Unfortunately, the complexity of \TeX\ vastly exceeds
\program{Scientific Word's} ability to act as a visual editor, and this
leads to a number of potentially confusing discrepancies.  For
example, in a \TeX\ document, the \verb|\parindent| and
\verb|\parskip| control sequences control the indentation of the
first line of a paragraph and the distance between paragraphs.  In
\program{Scientific Word}, the on-screen appearance of the document is
controlled separately.  This difference is apparent in
Figure~\ref{fig:sciword}, where the paragraphs appear to be indented with
no additional space between them, and Figure~\ref{fig:commercial:swpreview}
(the same document shown in the previewer) where the paragraphs are not
indented but have additional space between them.

These discrepancies arise because \program{Scientific Word} was designed with
the same philosophy as \LaTeX: separation of content and form.  The purpose of
the \program{Scientific Word} editor is to allow you to organize your thoughts
and perfect the content of your document.  It is the job of \LaTeX, in
conjunction with the document style options that you select, to perfect the
appearance of your document.  The display shown in Figure~\ref{fig:sciword} is
not incorrect in any way, it just isn't WYSIWYG.  TCI Research claims that
``by going beyond WYSIWYG, \program{Scientific Word} allows you to focus on
the creative process~$\ldots$ and not the layout commands necessary to typeset
[your document].''

\epsimage{fig:commercial:swpreview}{Preview of the Scientific Word document shown in Figure~\protect\ref{fig:sciword}}

When \program{Scientific Word} encounters \TeX\ code that it does not
understand, it leaves a labelled grey box in the display (you can see two such
boxes above the chapter title ``Unsolved Problems'' in
Figure~\ref{fig:sciword}).\footnote{I forced the issue of unrecognized control
sequences by inserting \cs{parskip} and \cs{parindent} control sequences into
the body of the document.  They really belong in a document style option.}
This means that you can edit all of the \TeX\ code in your document from
within \program{Scientific Word}, and small amounts of customization do not
require abandoning the program.  The \program{Scientific Word} technical
reference, included online with the program, describes how you can tailor
\program{Scientific Word} to recognize many of your customizations.

Behind the scenes, \program{Scientific Word} documents are processed by
a full-fledged \TeX\ processor (\ixx{\TurboTeX}{TurboTeX@\TurboTeX} by the
\idx{Kinch Computer Company}).
This includes integrated document formatting and previewing under Windows.
Starting with version 1.1 of \program{Scientific Word}, instructions are
included to switch to a different version of \TeX, or different previewers
and \dvidriver{}s, if you prefer.


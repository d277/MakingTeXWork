\chapter{Previewing}
\RCSID$Id: ch09.tex,v 1.1 2002/08/23 14:58:46 nwalsh Exp $
\label{chap:preview}

\ifincludechapter\else\endinput\fi

\bgroup % <<<<<<<<<<<<<<<<<<<<<<<<<<<<<<<<<<<<<<<<<<<<<<<<<<<<<<<<<<<<<<<<
%
\def\floatpagefraction{.3}

\index{previewing}\index{tex@\TeX!previewing}Because \TeX\ is not interactive, most \TeX\ documents are developed
iteratively.  After adding a significant amount of text or changing
the format of a document, you'll want to see what the document looks like.
Then you can add more text or try different formats.
Then, it's helpful to see it again $\ldots$

You could print the document, but that's wasteful and slow (not
to mention environmentally unfriendly).
This is where screen previewers enter the picture.  They allow you
to look at your document on a video display.

Accurate previewing is more difficult than printing your document
for several reasons.  Screen displays are much more diverse than
printers (previewing on a PC is very different from previewing on
a workstation running X11 even though printing to a LaserJet printer
is essentially the same from both places).  Also, it's very difficult
to preview documents that use printer-specific features.  For example,
if you use a PostScript figure in your document, it will be very easy
to print on a PostScript printer, but on most platforms it is much more 
difficult to preview that document on the screen (two exceptions are
the Amiga and NeXT which have integrated support for displaying PostScript
images).

In the following sections I'll explore options for previewing on several
platforms.  The X Window System is usually associated with \Unix\ workstations,
but several PC implementations (running under both MS-DOS and OS/2) are now
available.  The X11 previewers 
described here may be available
for those systems, but I haven't seen them.  \program{Ghostscript} is another
previewing option that is available on several platforms so it is described
in its own section.

Table~\ref{tab:previewers} summarizes the previewers described here.

\begin{xtable}{l|l|c|l}
  \caption{Common Previewers\label{tab:previewers}}\\
  \bf Previewer          & \bf Supplier  &  \bf OS       & \bf Comments\\[2pt]
  \hline
  \tstrut\program{xdvi}         & Free (xdvi)   & Unix/DesqView & X11 \\
  \program{\XTeX}        & Free (\XTeX)  & Unix          & X11 \\
  \program{Ghostscript}  & Free (gs)     & most          & PostScript \\
  \program{Ghostview}    & Free (Ghostview) & Unix, Windows & PostScript \\
  \program{dviscr}       & Free (\emTeX) & MS-DOS, OS/2  & \\
  \program{dvipm}        & Free (\emTeX) & OS/2          & Presentation Manager \\
  \program{dvivga}       & Free (dvivga) & MS-DOS        & EGA/VGA \\
  \program{\TeX\ Preview}& ArborText     & MS-DOS, Unix  & \\
  \program{dvideo}       & Kinch Software& MS-DOS        & EGA \\
  \program{PTI View}     & Personal \TeX & MS-DOS        & \\
  \program{dvimswin}     & Free (dvimswin) & MS-DOS      & Windows \\
  \program{dviwin}       & Free (dviwin) & MS-DOS        & Windows \\
  \program{wdviwin}      & Kinch Software& MS-DOS        & Windows \\
  \program{DVIWindo}     & \YY              & MS-DOS        & Windows \\[2pt]
  \program{dvi2tty}        & Free (dvi2tty)& Most          & ASCII \\
  \program{dvigt/dvitovdu} & Free        & Most      & \\
  \program{crudetype}    & Free      & VMS           & ASCII \\
  \hline
\end{xtable}

\section{Previewing Under X11}

\index{X11!previewing}\index{previewing!X11}The 
three most common X11 previewers are \program{xdvi}, \XTeX, and
\program{Ghostview} (really a PostScript previewer).
\program{Ghostview} runs on top of \program{Ghostscript} and is
described in the ``\nameref{sec:ghostscript}'' section of this chapter.

\subsection{Previewing with xdvi}

You can retrieve \ixx{\program{xdvi}}{xdvi} from the CTAN archives in the
directory \ctandir{dviware/xdvi}.

Figure~\ref{fig:xdvi} shows the preview process under \xdvi.  
\program{xdvi} reads the \ext{DVI} file and loads the \ixx{\ext{PK} files}{PK files@\ext{PK} files}
for any fonts that are required.  If \xdvi\ cannot find a requested
font, it will run \ixx{\MakeTeXPK}{MakeTeXPK} to create the font.  \MakeTeXPK\ is a
shell script (or a batch file called \ixx{\filename{maketexp}}{maketexp} under MS-DOS)
that tries to use \MF\ or \ixx{\program{ps2pk}}{ps2pk} to build the font.

\epsimage{fig:xdvi}{How Previewing with xdvi Works}

After forming each page, \xdvi\ passes it off to the X11 Window Manager to be
displayed.  Previewing under \xdvi\ is shown in Figure~\ref{fig:preview:xdvi}.

\epsimage{fig:preview:xdvi}{Previewing with \protect\xdvi}

Screen resolution is typically much lower than printer resolution.  Because
\xdvi\ uses the \ext{PK} files at printer resolution, it must scale them
before using them for display purposes.  The scaling process allows \xdvi\ to
use \ixx{\textit{anti-aliasing}}{anti-aliasing} to improve image quality on color
displays.  Anti-aliasing is a technique used to improve the appearance of
scaled images by using colored pixels around the edges of the image to provide
the illusion of partial pixels.  This can dramatically improve the readability
of the displayed text.  If the \ps\ font files are available, \xdvi\ can
display documents that use \ps\ fonts; otherwise, it performs font
substitution.  

A recent addition to \xdvi\ is the ability to preview documents that
include PostScript figures.  The figures are rendered behind the scenes
by \program{Ghostscript} and displayed by \xdvi.  
In my opinion, this addition really makes \xdvi\ one of the finest X11 
previewers available.  It is fast, uses printer fonts, has anti-aliasing for 
superb readability on color displays, and can include PostScript figures.

\subsection{Previewing with \protect\xtex}

The \xtex\ previewer\index{xtex (previewing)@\xtex\ (previewing)}\index{previewing!xtex@\xtex} is
very similar to \xdvi.  The
primary difference is that \xtex\ uses X11 fonts for display.
This means that \xtex\ must build fonts at the appropriate
resolution.  After the fonts have been built, \xtex\ is typically
a very fast previewer.

You can retrieve \xtex\ from the CTAN archives in the
directory \ctandir{dviware/xtex}.

\epsimage{fig:xtex}{How Previewing with \protect\xtex\ Works}

Figure~\ref{fig:xtex}
shows the preview process under \xtex.  Like \xdvi,
\xtex\ uses the \MakeTeXPK\ program to build \ext{PK} files for fonts
that are unavailable.  Additionally, \xtex\ uses the \TeXtoXfont\ shell
script, or batch file, to convert \ext{PK} fonts into X11 display fonts.

\newpage
The \xtex\ previewer relies on the X11 Window Manager to build
the display.  However, if \ps\ figures are present in your
document, \xtex\ will attempt to display them.  When \xtex\ is
built, you can specify that \ghostscript\ or another \ps\
interpreter be used to handle \ps\ figures.  

An example of \xtex's display is shown in Figure~\ref{fig:preview:xtex}.

\epsimage{fig:preview:xtex}{Previewing with \protect\xtex}

\section{Previewing with Ghostscript}
\label{sec:ghostscript}

Previewing with \program{Ghostscript}%
\index{previewing!Ghostscript}\index{Ghostscript!previewing} is 
quite different from previewing with \xdvi\ and \xtex.  Most previewers
process the \ext{DVI} file to build the display.  \ghostscript\ is a
general-purpose program for displaying \ps\ files.  Under X11, an additional
program called \ghostview\ provides more sophisticated control of previewing.

\newpage
\program{Ghostscript} and \ixx{\program{Ghostview}}{Ghostview} are products of the
\idx{Free Software Foundation (FSF)}.  You can retrieve them from the GNU archives on
\verb|prep.ai.mit.edu| in the directory \filename{/pub/gnu} or from any
mirror of those archives.

\program{Ghostscript} reads and interprets the \ps\ file created by a
program such as \dvips.  It provides its own means of handling font
substitution if the appropriate fonts are unavailable.  Because \dvips\
converts the \TeX\ \ext{DVI} file into \ps, \ghostscript\ can display all
of the elements of the document including \ps\ figures and other PostScript 
printer-specific commands.

An example of \program{Ghostview}'s display is shown in
Figure~\ref{fig:preview:ghostview}.

\epsimage{fig:preview:ghostview}{Previewing with Ghostview}

%\subsection{Commercial PostScript Interpreters}
%
%After conversion to PostScript, the issue is really one of previewing
%PostScript, not \TeX\ output in particular.  Therefore, commercial PostScript
%previewers can also be used for this purpose.

\section{Previewing with \protect\emTeX}
\label{preview:emtex}

Figure~\ref{fig:emdvi} in Chapter~\ref{chap:printing},
{\it\nameref{chap:printing}}, shows the relationship between the
various components involved when processing a \ext{DVI} file with
\emTeX\index{emTeX@\emTeX!previewing}\index{previewing!emTeX@\emTeX}.  Previewing 
and printing are very similar operations with
\emTeX.  To preview, you use the \ixx{\program{dviscr}}{dviscr} driver, and the
result is displayed on the screen.  To print, use one of the other
\dvidriver{}s, and the result is a file that can be sent directly to
your printer.
Chapter~\ref{chap:freetex}, {\it\nameref{chap:freetex}}, discusses
\emTeX\ in more detail.

\program{dvidrv} runs the appropriate \dvidriver:
\program{dviscr} for previewing, \ixx{\program{dvihplj}}{dvihplj} for
printing to an HP LaserJet printer, or \ixx{\program{dvidot}}{dvidot} for printing to
other dot matrix printers.

The \dvidriver\ reads the \ext{DVI} file and loads fonts from
\ixx{\ext{PK} files}{PK files@\ext{PK} files} or \ixx{\ext{FLI} font libraries}{FLI font libraries@\ext{FLI} font libraries}.  If your document
uses graphics, they are loaded from \ext{PCX}\index{PCX file format} or 
\ext{MSP} files\index{MSP file format}.

If your document uses a font that is not available, the
\dvidriver\ writes the commands necessary to build the
font to the \ext{MFJ} file\index{MFJ files}.\footnote{Some setup 
is required to obtain
this functionality.  By default, the driver asks for the name
of a font to substitute in place of the missing font.}  Before
performing font substitution, the driver will ask if you wish
to build the missing fonts.  If you elect to build them, the
\dvidriver\ stops and returns control to the \ixx{\program{dvidrv}}{dvidrv}
program.

\program{dvidrv} notices that the \dvidriver\ stopped
because of missing fonts and runs \program{\idx{MFjob}} to build them.  After
building the fonts, the previewer is run again to display
the document.

The \program{dvidrv} program is quite simple.  With some
care, it can be replaced by a batch file that does
more work.\footnote{Particularly in networked environments where
\program{dvidrv} assures that temporary filenames won't collide
between users.}  For example, I have replaced \program{dvidrv} with
a \program{\idx{4DOS}} batch file called \program{dvidxx} that can
automatically build \ps\ fonts by calling \program{ps2pk} in
addition to building \MF\ fonts with \program{MFjob}. 
The \program{dvidxx} batch file is printed in Example~\ref{ex:dvidxx}
in Appendix~\ref{app:examples}, {\it \nameref{app:examples}}.

A full-screen preview with \program{dviscr} is shown in
Figure~\ref{fig:preview:dviscr}.

\epsimage{fig:preview:dviscr}{Previewing with \protect\emTeX's dviscr}

Under OS/2, the \ixx{\program{dvipm}}{dvipm} previewer offers more power. 
A sample
\program{dvipm} display is shown in Figure~\ref{fig:preview:dvipm} later
in this chapter.

\epsimage{fig:preview:dvipm}{Previewing with \protect\emTeX's dvipm}

The \program{dviscr} and \program{dvipm} previewers use \textit{\idx{anti-aliasing}}
to obtain better image quality on a color display.  However, this translates 
into a poorer quality image when captured for display in a black-and-white
book.

\section{Previewing with dvivga}

\program{dvivga}\index{dvivga previewing}\index{previewing!dvivga} is 
an MS-DOS previewer for EGA and VGA displays.
You can retrieve \program{dvivga} (and a complete set of the Computer Modern
fonts in \ext{PK} format) from the CTAN archives in the directory
\ctandir{dviware/dvivga}.  \program{dvivga} requires \ext{PK} fonts
at screen-resolutions (around 100dpi); that is why a special set of fonts
is provided.  If you are using a dot-matrix printer with a similar resolution,
the special fonts may already be installed on your system.

\program{dvivga} does not support any \cs{special} commands for including
pictures or figures, but it does support configurable font-substitution for 
fonts that are unavailable.

Figure~\ref{fig:dvivga:vga}
shows a preview of the sample document from
Chapter~\ref{chap:macpack}.  This image is from a VGA display.  

\epsimage{fig:dvivga:vga}{Previewing with dvivga}

\section{\protect\TeX\ Preview}
\label{sec:arb:preview}
\def\tpre{\program{\TeX\ Preview}}

\tpre\index{TeX Preview previewing@\TeX\ Preview previewing}\index{previewing!TeX Preview@\TeX\ Preview} is 
the \idx{ArborText} \ext{DVI} previewer for MS-DOS.  A similar previewer
is available for \Unix\ workstations running the X Window System (versions for
Motif and Open Look environments are also available).
The following discussion is based on experiences with \tpre\
version 6.1.2, the MS-DOS implementation of ArborText's \TeX\
previewer.

\tpre\ supports a wide range of graphics adapters, including EGA, VGA,
and Hercules adapters as well as the Olivetti Monochrome Graphics
adapter, the Tecmar Graphics Master, the Genius VHR Full Page Display
Monitor, the ETAP Neftis Monitor, the Toshiba 3100, and the AT\&T
PC6300 display.

Figure~\ref{fig:tpre:vga}
shows a preview of the sample document from
Chapter~\ref{chap:macpack}, {\it \nameref{chap:macpack}}.  This image
is from a VGA display.  Three additional features of the driver, not exercised
in this example, are the ability to scale fonts to any size, display rulers
and bitmapped graphics, and a ``two-up'' mode for viewing two
pages at a time.

\epsimage{fig:tpre:vga}{Previewing with ArborText's Previewer}

When \tpre\ is displaying a \ext{DVI} file, you can move around the page
and between pages; you can change the magnification, search for text
in the \ext{DVI} file, show the attributes of the character under the
cursor (font, dimensions, magnification, etc.), and switch to another file.
A configuration file allows you to specify a wide range of options to
control how \tpre\ appears when it starts up.

ArborText supplies a full set of Computer Modern Roman \ext{PK} files
at screen resolutions as well as a complete set of Times Roman and
Helvetica fonts at screen resolutions.  The additional PostScript
fonts are derived from official Adobe sources and allow you to preview
documents that will be printed on PostScript printers (provided that
they use only Times, Helvetica, and Computer Modern fonts).  They are
designed specifically to work with \ext{TFM} files provided with
\program{DVILASER/PS}, ArborText's PostScript \dvidriver.

If you want to use additional fonts, for example the \AmS-fonts or
Computer Modern fonts at unusual sizes, you may wish to generate
\ext{PK} files at the appropriate resolutions.  \tpre\ will perform
font substitution for missing fonts (you can control what
substitutions are made) and can use the \ext{DVI} driver metric
information, so generating additional fonts is not necessary.

\tpre\ understands the virtual font mechanisms introduced in \TeX\ version
3.0.  Several additional utilities provided with \tpre\ allow
you to construct virtual fonts.  These utilities are summarized in
Table~\ref{tab:tpre:utils}\index{aftovp}\index{gftopk}\index{packpxl}%
\index{pktopx}\index{pxtopk}\index{tftovp}\index{unpkpxl}\index{VFtoVP}%
\index{VPtoVF}.

{\def\x{${}^1$}%
\begin{xtable}{l|l}
  \caption{\protect\tpre\ Utilities\label{tab:tpre:utils}}\\
  \bf Utility      & \bf Description \\[2pt]
  \hline
  \endfirsthead
  \caption[]{\protect\tpre\ Utilities (continued)}\\
  \bf Utility      & \bf Description \\[2pt]
  \hline
  \endhead
  \tstrut
  \it aftovp   & Converts \ext{VPL} from \ext{AFM} file \\
  \it gftopk\,\x & Converts \ext{GF} files into \ext{PK} files \\
  \it packpxl  & Creates packed (byte-aligned) \ext{PXL} file \\
  \it pktopx\,\x & Converts \ext{PK} files into \ext{PXL} files \\
  \it pxtopk\,\x & Converts \ext{PXL} files into \ext{PK} files \\
  \it tftovp   & Converts \ext{VPL} from \ext{TFM} file \\
  \it unpkpxl  & Creates standard, word-aligned \ext{PXL} file \\
  \it vftovp\,\x & Converts \ext{VF} files into \ext{VPL} files \\
  \it vptovf\,\x & Converts \ext{VPL} files into \ext{VF} files \\[2pt]
  \hline
  \multicolumn{2}{l}{%
    ${}^1$\vrule height11pt width0pt\tiny A standard, or otherwise 
    freely available, utility.}
\end{xtable}
}

ArborText's \tpre\ program recognizes \cs{special} commands for drawing
change bars and for rotating {\em any} \TeX\ ``box'' through a multiple
of 90 degrees.  These are the same \cs{special} commands recognized by
other ArborText \dvidriver{}s.

\section{dvideo}

The MS-DOS based EGA 
previewer \program{dvideo}\index{dvideo previewing}\index{previewing!dvideo} 
is distributed as part
of the \ixx{\TurboTeX}{TurboTeX@\TurboTeX} distribution by the 
\idx{Kinch Computer Company}.  
\TurboTeX\ is described more completely
in the section called ``\nameref{sec:com:turbo}'' in 
Chapter~\ref{chap:commercialtex}, {\it \nameref{chap:commercialtex}}.
\TurboTeX\ also includes a Microsoft Windows previewer.

Figure~\ref{fig:ttexd:ega}
shows a \program{dvideo} preview of the sample
document from Chapter~\ref{chap:macpack}.  

\epsimage{fig:ttexd:ega}{Previewing with \protect\program{Turbo\TeX} 
  \protect\program{dvideo} (using limited selection of fonts)}%

The preview displayed here uses the default set of fonts distributed
with \TurboTeX.  This does not include several of the large sizes used
by this example.  In practice, you will have to purchase or build many
fonts before you can use \TurboTeX.

\newpage
\section{PTI View}

\program{PTI View}\index{PTI View!previewing}%
\index{previewing!PTI View} 
is the MS-DOS previewer that comes with \PCTeX,
distributed by \ixx{Personal \TeX, Inc.}{Personal tex, Inc.@Personal \TeX, Inc.}
\PCTeX\ is described in the section called ``\nameref{sec:pctex}''
in Chapter~\ref{chap:commercialtex}, \textit{\nameref{chap:commercialtex}}.

Figure~\ref{fig:ptipreview}
shows a \program{PTI View} preview of the
sample document from Chapter~\ref{chap:macpack}.

\epsimage{fig:ptipreview}{Previewing with Personal \protect\TeX's Previewer}

\program{PTI View} is distributed with a complete set of the Computer
Modern fonts in \ext{PK} format (\program{PTI View} can use the same
\ext{PK} fonts as your printer, regardless of the printer's
resolution).  It can display the preview in many video modes,
including several high-resolution super VGA modes.

\newpage
\section{Previewing Under Windows}

Most MS-DOS previewers are inappropriate for previewing in 
Windows\index{previewing!Microsoft Windows}\index{Microsoft Windows!previewing}
because they assume that they can control the entire display. Recently,
several commercial and free previewers for Windows have become available.
They are described in this section.

\subsection{dvimswin}

The \program{dvimswin}\index{dvimswin!previewing}\index{previewing!dvimswin}
previewer is a Windows version of
\program{dvivga}.  It uses the same screen-resolution fonts for
displaying your \ext{DVI} file and offers font-substitution for
missing fonts.

You can retrieve \program{dvimswin} from the CTAN archives in
the directory \ctandir{dviware/dvimswin}.
The \program{dvimswin} previewer is shown in
Figure~\ref{fig:preview:dvimswin}.

\epsimage{fig:preview:dvimswin}{Previewing with dvimswin}

\newpage
\subsection{dviwin}

The \program{dviwin}\index{dviwin previewing}\index{previewing!dviwin}
previewer is another free Microsoft Windows
previewer.  \program{dviwin} can use either screen or printer resolution
fonts 
to display your \ext{DVI} file.
You can retrieve \program{dviwin} from the CTAN archives in
the directory \ctandir{dviware/dviwin}.

What makes \program{dviwin} unique is its support for \cs{special} commands.
\program{dviwin} understands \cs{special} commands for including pictures
and figures, as well as the \emTeX\ and \filename{tpic} drawing primitives.

\program{dviwin} has built-in support for \ext{PCX}\index{PCX file format}, \ext{BMP}\index{windows@Windows!BMP file format}, and
\ext{MSP}\index{MSP file format} graphic formats.  
Additionally, it can use any graphics filter
installed in your Windows environment.  Many commercial programs include
additional filters to handle the images that they construct.  
\program{dviwin} also supports \ixx{\emTeX\ font libraries}{emTeX@\emTeX!font libraries} and customizable
automatic font generation.  By using \program{dviwin}, you can print your
\ext{DVI} files on any Microsoft Windows-supported printer.
The \program{dviwin} previewer is shown in Figure~\ref{fig:preview:dviwin}.

\epsimage{fig:preview:dviwin}{Previewing with dviwin}

\newpage
The \program{dviwin} distribution includes two additional utilities:
\ixx{\program{clipmeta}}{clipmeta} for creating \ext{MSP} graphic files from any image
captured in the Windows clipboard, and \ixx{\program{wbr},}{wbr} a text-file browser.

\subsection{wdviwin}

As mentioned above, the \ixx{\TurboTeX}{TurboTeX@\TurboTeX} package includes a Windows
\dvidriver, 
\program{wdviwin}\index{wdviwin (previewing)}\index{previewing!wdviwin},
distributed by the \idx{Kinch Computer Company}.  
Sample output from this previewer is shown in
Figure~\ref{fig:preview:wttpreview}.
This sample shows the status window,
the preview window, and a few of the available tools.

\epsimage{fig:preview:wttpreview}{Previewing with \protect\TurboTeX's wdviwin}

\subsection{DVIWindo}

The 
\program{DVIWindo}\index{previewing!DVIWindo}\index{DVIWindo!reviewing@previewing} previewer by \ixx{\YY}{YY Inc.@\YY\ Inc.} is unique among the 
previewers used here.  \program{DVIWindo} has no support for \ext{PK} fonts;
it relies entirely on scalable fonts\index{fonts!scalable} provided 
by Microsoft Windows.
(This means either PostScript Type~1 fonts rendered 
by \ixx{\program{Adobe Type Manager}}{Adobe Type Manager (ATM)} or built-in TrueType font
\index{fonts!TrueType} support.)  As a result, to use
the \program{DVIWindo} previewer, you must purchase the Computer
Modern fonts in TrueType or Adobe Type~1 format (or not use them at all).

The \program{DVIWindo} previewer is shown in Figure~\ref{fig:preview:dviwindo}.
The pull-down menu shown in this image is the ``\TeX\ Menu,'' a new feature
of \program{DVIWindo} 1.1.  This menu, which can be customized to include
any programs you wish, allows \program{DVIWindo} to function as a \TeX\ 
shell.  Once \program{DVIWindo} is running, you can edit files and format
documents with \TeX\ directly from this menu.

\epsimage{fig:preview:dviwindo}{Previewing with \protect\YY's DVIWindo}

The real advantage of using scalable fonts is that you can resize the
image in arbitrary ways.\footnote{Most previewers support resizing, but they
are generally limited to the resolutions for which \ext{PK} fonts are
available.}
For example,
Figure~\ref{fig:preview:dviwindo2}
shows a much-enlarged version of
the same page.  Similar enlargement with non-scalable fonts would
require that the \ext{PK} fonts exist at very high-resolutions
(occupying considerable disk space) or produce very jagged output.
The jaggedness of the image shown here is the result of
magnifying the captured screen image, not the previewer.  Several
useful information boxes are also shown in these images.

\epsimage{fig:preview:dviwindo2}{DVIWindo preview much enlarged}

\program{DVIWindo} has several other interesting features:

\begin{itemize}
  \item Portions of a document may be copied into the Windows clipboard 
    and then pasted into other applications.  This allows you to construct
    complex mathematics in \TeX, for example, and paste them into another
    Windows application.  The pasted material will appear as a single
    graphical object that can be moved, resized, and cropped.  The material
    is not rendered as a bitmap, so it can be resized without loss of quality!

  \item TIFF images are displayed in the document.  If you use
     \program{dvipsone} (\YY's PostScript printer driver) and create both
     TIFF and EPS (encapsulated PostScript) versions of an image,
     \program{DVIWindo} will automatically display the TIFF image, and
     \program{dvipsone} will automatically print the EPS image.

  \item Colored text and rules can be incorporated into a document with
    \cs{special} commands.
\newpage
  \item You can create ``hypertext'' buttons in your document.  Selecting a 
    button (an area of text or a rule) automatically moves you to a destination
    marker in the document.  These buttons
    are only meaningful to \program{DVIWindo}, but they do allow you to move
    around quickly in a document while you are writing it.  Of course, they
    could also be very handy if you use \ext{DVI} files for online
    documentation.  
\end{itemize}

In addition to the \program{DVIWindo} executable, several programs are provided
to help you work with PostScript fonts under Windows.  They are summarized
in Table~\ref{tab:dviwindo}.
    
{\def\x{${}^1$}%
\begin{xtable}{l|l}
  \caption{DVIWindo Utilities\label{tab:dviwindo}}\\
  \bf Utility      & \bf Description \\[2pt]
  \hline
  \endfirsthead
  \caption{DVIWindo Utilities (continued)}\\
  \bf Utility      & \bf Description \\[2pt]
  \hline
  \endhead
  \tstrut
  \it pfatopfb\,\x  & Convert \ext{PFA} files into \ext{PFB} files\\
  \it pfbtopfa\,\x  & Convert \ext{PFB} files into \ext{PFA} files\\
  \it tifftags  & Show the tags used in a \ext{TIFF} image\\
  \it reencode  & Change the encoding of a PostScript font\\
  \it afmtotfm\,\x  & Convert \ext{AFM} files into \ext{TFM} files\\
  \it tfmtoafm  & Convert \ext{TFM} files into \ext{AFM} files\\
  \it afmtopfm  & Convert \ext{AFM} files into \ext{PFM} files\\
  \it pfmtoafm  & Convert \ext{PFM} files into \ext{AFM} files\\
  \it safeseac  & Circumvents problem with accented letters in PS 
                    fonts under Windows\\
  \it cleanup   & Removes inactive Windows from the desktop\\
  \it sysseg    & Displays information about Windows system memory\\[2pt]
  \hline
  \multicolumn{2}{l}{%
    ${}^1$\vrule height11pt width0pt\tiny A standard, or otherwise 
    freely available, utility.}
\end{xtable}
}

\section{Previewing on a TTY}
\label{sec:preview:tty}

\index{previewing!TTY}Graphical workstations 
and personal computers with graphics capabilities
are a natural environment for previewing \TeX\ output.  Unfortunately,
they aren't always available.  This section describes several previewers
that work in less sophisticated environments.

\subsection{dvi2tty}

The \program{dvi2tty} 
program\index{previewing!dvi2tty}\index{dvi2tty previewing} attempts 
to convert \TeX\ output into 
ASCII text.  This program is designed to provide an approximation of
\ixx{\troff}{troff}'s \ixx{\program{nroff}}{nroff} processor.
To get the best results,
you will have to reformat your document using a limited subset of 
\TeX's capabilities.  A \LaTeX\ style file is included for this purpose.
You can retrieve \program{dvi2tty} from the CTAN archives in
the directory \ctandir{dviware/dvi2tty}.

The output from \program{dvi2tty} is imperfect in many ways, but it can
provide a workable ASCII document.  I used it to produce plain ASCII
documentation in my \ixx{\program{Sfware}}{Sfware} package, for example.  
\program{dvi2tty} also provides a quick-and-dirty method of applying some
standard text processing tools, like \ixx{\program{grep}}{grep}, to \TeX\ output.

\subsection{dvgt/dvitovdu}

\index{dvgt}\index{dvitovdu}\index{previewing!dvgt/dvitovdu}These 
programs share a common history.  As a result, they offer an
overlapping set of features.  The most recent work has been done on the
\program{dvgt} processor.
You can retrieve \program{dvgt} from the CTAN archives in the directory
\ctandir{dviware/dvgt}.

Unlike \program{dvi2tty}, which is a conversion program, \program{dvgt} is
an interactive previewer.  
One very neat feature of \program{dvgt} is its ability to preview \TeX\
output on a number of graphics terminals, including Tektronix 4010,
VT240, VT640, Gigi, Regis, VIS500, VIS550, VIS603, and VIS630 terminals.  
The importance of this
feature is that many versions of \ixx{\program{Kermit}}{Kermit} 
and \ixx{\program{NCSA Telnet}}{NCSA Telnet} 
(and possibly other communications programs) support one of these terminal 
types.  This means that you can preview documents 
even when you are away from your workstation, by using dial-up access with either
a graphics terminal or plain ASCII. 

Even when output is limited to plain ASCII,
\program{dvgt} attempts to make its output resemble the printed page.
To do this, it frequently drops characters out of the middle of words
and performs other space-saving abbreviations.  The result is a crude, but
workable preview, on a plain ASCII terminal.

Figure~\ref{fig:dvgttek}
shows an example of \program{dvgt} display in
tek4010 emulation.  This is a somewhat contrived example since the
emulation is being performed by an xterm.

\epsimage{fig:dvgttek}{Previewing with dvgt under Tektronix 4010 emulation}

\program{dvitovdu} is an older version of the program.  It is available in
source code form in both C and Pascal.  
You can retrieve \program{dvitovdu}
from the CTAN archives in the directory \ctandir{dviware/dvitovdu}.

\subsection{crudetype}

\program{crudetype}\index{previewing!crudetype}\index{crudetype previewing}
is
another plain-ASCII previewer.  It provides features
similar to \program{dvgt} and \program{dvitovdu}.  It is written in \Web,
but it cannot be translated to C with the \program{web2c} conversion tool
(so you will have to have a Pascal compiler).

The \program{crudetype} program was written with VMS in mind, although it
may be portable to other systems.
You can retrieve \program{crudetype}
from the CTAN archives in the directory \ctandir{dviware/crudetype}.

\egroup % <<<<<<<<<<<<<<<<<<<<<<<<<<<<<<<<<<<<<<<<<<<<<<<<<<<<<<<<<<<<

\chapter{Running \protect\TeX} 
\RCSID$Id: ch03.tex,v 1.1 2002/08/23 14:58:46 nwalsh Exp $
\label{chap:running}

\ifincludechapter\else\endinput\fi

The heart of processing a document with \TeX\ is 
running\index{tex@\TeX!running}\index{running tex@running \TeX} the \TeX\ program.
In this chapter, you'll learn what happens between the time you first start
the \TeX\ program and the time it finishes.
You'll also learn what files \TeX\ really needs (in addition to
your document) and what to do when \TeX\ finds things in your document that
it doesn't understand.

\section{What Do You Run?}

The first thing you have to know is what program to run.
Unfortunately, the actual file you have to execute varies between
platforms and implementations. If you have built and/or installed
\TeX\ yourself, you probably already know what program to run.  
You'll have to ask your system administrator for help if you can't
figure out what the name of the \TeX\ executable is on your computer.
The rest of this chapter assumes that the command \filename{tex}
runs \TeX.  You should substitute the name of the executable
program on your own system for \filename{tex}\index{tex command} in 
the examples that follow.

\label{sec:bigandsmall}

Most implementations of \TeX\ have some hard-coded \idx{memory limits}.
These limitations may cause problems if you are trying to run a very
complex document through \TeX.
To combat this problem, some distributions include two versions
of the \TeX\ executable.  One version is 
a ``small'' \TeX\index{small tex@small \TeX}\index{tex@\TeX!small} that generally runs
faster but has less memory available to process your
documents.  The other
version, a ``big'' \TeX\index{big tex@big \TeX}\index{tex@\TeX!big}, 
can process more complex documents
but may run more slowly.  If you get a ``\TeX\ capacity exceeded''
error, and you have a big \TeX\ available on your system, try
processing your document with the big \TeX.

If you still get an error, you have a document that is simply too
complex for your implementation of \TeX\ to handle, or you have an
error in one of the macros in your document.  By examining the error
log, described later in this chapter in the section 
``\nameref{sec:logfiles},'' you can determine what macro \TeX\ is
interpreting when the error occurs.  If the error occurs in a macro
that you wrote, check to make sure the macro functions the way
you intended by using it in a small test document.

\section{What Files Does \protect\TeX\ Need?}

Naturally, \TeX\ needs your input file in order to process it.
However, \TeX\ must be able to find several
other files\index{files for tex@files for \TeX}\index{tex@\TeX!file requirements} as 
well.  The files that \TeX\ needs are normally created
during the installation process.  Here is a common directory layout for
\TeX\ on a \Unix\ system:

\begin{exindent}
\begin{tabbing}
\filename{/usr/local/lib/tex/formats}\ \ \ \ \ \ \= Format files \kill
\filename{/usr/local/lib/tex/pool}      \> Pool files \\[2pt]
\filename{/usr/local/lib/tex/formats}   \> Format files \\[2pt]
\filename{/usr/local/lib/tex/inputs}    \> ``System'' input files \\
\filename{/usr/local/lib/tex/macros}    \>\ \ \  (styles, macros, and miscellaneous\\
\filename{/usr/local/lib/tex/lib}       \>\ \ \   files distributed with \TeX)\\[2pt]
\filename{/usr/local/lib/tex/ps}        \> PostScript support files \\[2pt]
\filename{/usr/local/lib/tex/fonts/pk}  \> \ext{PK} fonts\\[2pt]
\filename{/usr/local/lib/tex/fonts/vf}  \> Virtual fonts\\[2pt]
\filename{/usr/local/lib/tex/fonts/tfm} \> \ext{TFM} metrics\\
\end{tabbing}
\end{exindent}

\vspace{-6pt}
A similar layout is frequently used on other operating systems,
except that the \TeX\ files are often stored in a top-level
directory (for example, \filename{C:$\backslash$TEX} under MS-DOS).  Because \TeX\ is
very flexible and has many different implementations, the exact directory structure varies.  Pool files are sometimes
placed in the format file directory (which is sometimes called
\filename{fmt} or \filename{fmts}, rather than \filename{formats}).
Input files may occur in one or more of the standard places listed above
% in Figure~\ref{fig:run:dirlayout} 
(\filename{inputs},
\filename{macros}, and \filename{lib}) as well as a number of other
places (\filename{texinputs} is another common name).

The files under the \TeX\ tree usually come from outside distributions,
so you don't have to change them often. (You definitely \emph{shouldn't}
put your personal macro files in that tree, even if you do have access
to it.)

The following sections describe, in more detail, each of the files that \TeX\ needs.

\subsection{Pool Files}

Pool files contain string constants used by \TeX\ at runtime.
Each time \TeX\ is compiled, it creates a \ixx{pool file}{pool files} unique
to the version being compiled.  The pool file from one version of \TeX\ will
not work with a different version.

In contrast to format files (discussed in the next section),
if you don't have a pool file, there's nothing you can do about it.
If you obtained precompiled programs (from the Internet, from a friend,
or commercially) and you don't have a pool file, you received an
incomplete distribution.

If you did not install \TeX\ yourself, but find that the pool file is
missing, contact the system administrator who performed the
installation.  He or she did something wrong.

\subsection{Format Files}
\label{run:sec:formatfiles}

As mentioned in the section ``\nameref{sec:tex:controlsequences}'' in
Chapter~\ref{chap:tex}, \textit{\nameref{chap:tex}}, all of the control
sequences and macros that your document uses must be defined
somewhere.  One way to do this is to use the \cs{input} command to
load all of the definitions at the beginning of each document.
However, with large macro packages like \LaTeX, this process can be very time
consuming.  \ixx{Format files}{format files} are a way of predefining control
sequences and macros so they don't have to be interpreted by \TeX\
every time they're used.  If \TeX\ didn't use format files, you'd
have to wait for \TeX\ to interpret all of \LaTeX\ every time you
processed a document that used \LaTeX.

\label{run:sec:initexintro}
Format files are created by a special version of \TeX, usually called
\ixx{ini\TeX}{initex@ini\TeX}.  Some implementations combine \TeX\ and ini\TeX\ into
one program.  In this case, you must select ini\TeX\ with a special
option when you run \TeX.  Ini\TeX\ interprets all the control
sequences in a macro package and builds the in-memory data structures
that the \TeX\ program needs.  After loading the whole macro package,
ini\TeX\ writes the memory image it has constructed to a format
file.  When \TeX\ later loads the format file, it simply copies it
into memory; no interpretation is necessary.  This is why loading a
format file is such a fast way to define control 
sequences\index{control sequences!loading through format files}.  This is
also why format files are not usually portable from one system to
another,\footnote{The exception occurs in some \Unix\ versions of \TeX\
where provisions are made for different architectures to share the
same format files.  The exact same version of \TeX\ must be running on
both architectures for this to work.} or even between different
versions of \TeX\ on the same system.  Different versions of \TeX\
are stored differently 
in memory, and this difference in loading makes the
format files incompatible.  For this reason, you need a ``big''
ini\TeX\ to make format files for a big \TeX\ and a ``small'' ini\TeX\ for
a small \TeX.

%\begin{sidebar}[Help, I Need a Format File]
You will find a complete discussion of macro packages and 
instructions for building format files for many of the common macro 
packages in Chapter~\ref{chap:macpack},
\textit{\nameref{chap:macpack}}.
%\end{sidebar}

\subsection{User Files}
\label{sec:userfiles}

When you run \TeX\index{user files}, you have to tell it what file to process.
If you specify a complete pathname, \TeX\ will load the specific file that you
request.  For example, I could process the document \filename{myreport.tex} in
the directory \filename{/home/norm/tex/} by running:

\begin{ttindent}
\$ \textbf{tex /home/norm/tex/myreport}
\end{ttindent}

This example applies to both \Unix\ and PC implementations of \TeX.
Even though MS-DOS and OS/2 typically use the backslash to separate
directory names, {\em always\/} use a forward slash when entering filenames
for \TeX.

\newpage
If you specify a simple filename without a path, \TeX\ looks for
the file in several user-defined and, possibly, system-defined locations.
In the following example, \TeX\ will attempt to locate the file 
\filename{myreport.tex} in order to process it:

\begin{ttindent}
\$ \textbf{tex myreport}
\end{ttindent}

If the file \filename{myreport.tex} exists in several directories, \TeX\
will process the first file it finds without looking for any others.
\TeX\ always prints the complete name of the file it processes both
on the display and in the log file, so you can always tell what file was
really processed.

The most common way to specify user-defined locations is by setting the
\envvar{TEXINPUTS}
environment variable\footnote{The exact name of the
\ixx{environment variable}{environment variables!format}\index{variables!environment!format} differs between
implementations.}  to a list of subdirectories where \TeX\ documents are kept.
Setting \envvar{TEXINPUTS} specifies where \TeX\ should look for
\emph{documents} (files that you input with \cs{input} or some other
construct, style files, and macros).  To change where \TeX\ looks for other
kinds of files (fonts, formats, pool files, etc.), setting different
environment variables or performing some other customization is required.  The
format of the environment variable differs according to the platform you use.
On \Unix\ systems, it is a list of directory names separated by colons.  Here
is a typical example:

\begin{shortexample}
.:/usr/local/lib/tex/inputs:/home/norm/tex/inputs
\end{shortexample}

On MS-DOS and OS/2 systems, it is a list of directory names separated by
semicolons.  A typical example looks like this:\footnote{You will notice
that I've used backslashes\index{bs (backslash)@\bs (backslash)|nn}
to separate directory components in the
path.  Environment variables are handled by the system-dependent portions of
\TeX, so it's okay to use backslashes here.  You can also use forward slashes,
but I use backslashes because that is more typical of MS-DOS and OS/2
environments.}

\begin{shortexample}
.;\tex\inputs;\tex\styles;\tex\macros
\end{shortexample}

Consult the documentation for your particular implementation of \TeX\
for more information about system-defined locations
where \TeX\ looks for input files.  Unfortunately, this is not always
well documented.  For example, an undocumented
feature of version 1.4s of \emTeX\
always searches in the directory \filename{\\emtex\\texinputs} even if
it does not occur in the \envvar{TEXINPUTS} path.  In fact, there is
no way to tell it {\em not\/} to look there, short of renaming the
directory (which is what I did).

If \TeX\ cannot find a file you specify, for example
\filename{rpt-data.tex}, it displays a prompt like this one:

\begin{shortexample}
! I can't find file `rpt-data.tex'.
<*> \input rpt-data

Please type another input filename:
\end{shortexample}

The general form of \TeX\ error messages is discussed in
the section called ``\nameref{sec:errors}'' later in this chapter.  \TeX\ 
announces an error by printing the error message on a line that begins with 
an exclamation mark.  Below the error, \TeX\ provides the context in which 
the error occurred. In the example above, \TeX\ encountered the command 
\cs{input rpt-data} and tried to find the file 
\filename{rpt-data.tex} which
it could not locate.

Notice that \TeX\ automatically added the 
extension \filename{.tex}\index{tex files} to
the name of the file it was looking for.  In any context where you
specify a filename, \TeX\ will append \filename{.tex} unless you
specify an alternate extension.

\TeX\ responds to a ``can't find file'' error by asking you to type
the name of a different file.  Some implementations of \TeX\ allow you
to abort by typing Ctrl-C or Ctrl-D at this point,\footnote{MS-DOS and
OS/2 implementations of \TeX\ may use Ctrl-Z (possibly followed by
a carriage return) instead of Ctrl-C or Ctrl-D.} but other
implementations insist that you enter a filename.  In this case,
you'll find it convenient to create an empty file called
\filename{nul.tex}\index{nultex files@nul.tex files} in 
a directory in your \envvar{TEXINPUTS} path so
you can get around this requirement by supplying a dummy answer to
the prompt.  \filename{nul.tex} is part of many standard distributions.

\section{The Command Line}
\label{sec:clineopts}

Except for the name of a document, \TeX\ has very few command-line
options\index{command line!options}.  The only 
option that is regularly used is the name of a
format file, but the use of this option is frequently buried inside
a batch file or shell script.

It is helpful to think of \TeX, \LaTeX, \SliTeX, etc.~as
different text processors (they aren't really; they're all \TeX\ with
different format files).  This illusion is easy to provide with shell
scripts or batch files.  For example, on an MS-DOS system, you could
easily have three batch files called \filename{tex.bat}, \filename{latex.bat},
and \filename{slitex.bat}:

\begin{itemize}
  \item The \filename{tex.bat} file runs the \TeX\ executable
        with the \filename{plain}
        format file specified as an option.
  \item The \filename{latex.bat} file runs \TeX\ with the \filename{lplain}
        format file.
  \item The \filename{slitex.bat} file runs \TeX\ with the \filename{splain}
        format file.
\end{itemize}

Now typing \textbf{tex\ \textit{filename}} processes Plain \TeX\ documents;
typing \textbf{latex\ \textit{filename}} processes \LaTeX\ documents; and
typing \textbf{slitex\ \textit{filename}} processes \SliTeX\ 
documents.\footnote{On \Unix\ systems, the same effect is often achieved 
with symbolic links.  Most \Unix\ implementations look for different 
standard format files based upon the name of the executable that starts 
them.}  The role of format files and macro packages is described
fully in Chapter~\ref{chap:macpack}, \textit{\nameref{chap:macpack}}.

\subsection{Command-line Options}

A formal specification of the \TeX\ command 
line\index{command line!options} looks like this:

\begin{exindent}
\begin{alltt}
\$ tex <\textit{switches}> <&\textit{format}> <|\textit{document}> <\textit{tex-commands}>
\end{alltt}
\end{exindent}

If this looks confusing, have no fear. I'll explain what it means in
English.

After the name of the \TeX\ program (or batch file), the first things
that can go on the command line are implementation-dependent switches
and options.  For example, implementations of \TeX\ that combine
ini\TeX\ and \TeX\ into a single program may use
\textbf{/I} as the switch to specify that ini\TeX\ 
processing\index{initex@ini\TeX!requesting on command line} is 
desired (ini\TeX,
you may remember, was described in the section
``\nameref{run:sec:formatfiles},''
earlier in this chapter).  There are
no system-independent switches for \TeX.  Consult the documentation
that comes with your implementation for more information about
system-dependent switches.

After any system-dependent switches, the first thing that you can put
on the \TeX\ command line is the name of
the \ixx{format file}{format files!requesting on command line} to use.  If
you specify this option, you must include it before any other options,
and you must put an ampersand (\&)\index{ampersand}
in front of the format file name.
If you do not specify a format file, \TeX\ will use a default format,
usually Plain \TeX.

After the format, \TeX\ looks for the name of a document to process.
If \TeX\ finds a filename on the command line, it will process the
document\index{documents!tex@\TeX!specifying on command line} contained 
in that file before looking at any other options
that may follow.

Finally, you can insert arbitrary \TeX\ commands on the command line
by typing them just as you would in a document.  The section
``\nameref{sec:texcomp}'' in Chapter~\ref{chap:writing},
\textit{\nameref{chap:writing}}, describes one particular instance where
this is very useful, but it isn't something that you are likely to do
very often.

\subsection{Command-line Cautions}
\label{sec:runtex:cmdcautions}

There are some special restrictions on file names used on the command
line and on the way \TeX\ interprets the command line.  These
restrictions are summarized here.

\subsubsection{Misinterpretation of the command line}
\label{run:sec:misinterpret}

Command lines that are typed at the operating system prompt are
actually seen by your operating system's ``command processor'' before
they are seen by \TeX.  You must be aware of special processing that
might be performed by the command processor.  Under \Unix, for example,
the backslash\index{bs (backslash)@\bs (backslash)!special meaning for} is frequently interpreted 
as a \textit{shell escape}
character\index{shell!escape character}, and 
the ampersand\index{ampersand} has another special 
meaning related to
\idx{job control}.  The ampersand is also special under some MS-DOS and OS/2
command processors where it is the default command separation
character and will not be passed to \TeX\ as you would expect.

To insert these troublesome characters literally, you must enclose the
relevant sections of the command line 
in \ixx{quotation marks}{quotation marks!use with special characters} or 
use some
form of shell escape mechanism.  Under \Unix, place the name of the
format file, including the ampersand, in double or
\linebreak
\newpage
single quotes and
use two backslashes in a row if you include \TeX\ control sequences on
the command line.  For example, type:

\begin{exindent}
\begin{alltt}
\$ \textbf{tex '&lplain' \bs{\bs}nonstopmode \bs{\bs}input doc1}
\end{alltt}
\end{exindent}

instead of

\begin{exindent}
\begin{alltt}
\$ \textbf{tex &lplain {\bs}nonstopmode {\bs}input doc1}
\end{alltt}
\end{exindent}

When using OS/2, place a \caret\ in front of an ampersand
to prevent it from being interpreted as a command separator.

\subsubsection{Filenames and \protect\TeX}

\index{filenames}\index{running tex@running \TeX!understanding filenames}%
\index{command line!specifying filenames}%
It 
is easy for \TeX\ to recognize the presence of a format file on the
command line; it must be the first option and it must begin with an
ampersand.\footnote{In the unlikely event that you have a \TeX\ document
stored in a file with a name beginning with an ampersand, use the following
trick to process that file: instead of running \textit{tex \textnormal{\&}file}, run
\textit{tex {\bs}input \textnormal{\&}file} (or even 
\textit{tex \textnormal{\&}format {\bs}input \textnormal{\&}file}).}
After the format file,
\TeX\ determines whether the next option is a file name by looking at
the first character of the option.  If the first character is not a
backslash, it is a filename; otherwise it is a \TeX\ command, and \TeX\
assumes that no filename is present.

On some systems, notably MS-DOS and OS/2, the 
backslash\index{bs (backslash)@\bs (backslash)!in MS-DOS and OS/2}
is
used to delimit the subdirectory components of a file name.  On other
systems, a forward slash is used.  Regardless of the system you use,
filenames passed to \TeX\ {\em must\/} use forward slashes to
delimit filenames.

For example, \TeX\ will not process the file \filename{letter.tex}
in the \filename{{\bs}tex{\bs}docs} directory of the current drive if
you execute the following command line on an MS-DOS system:

\begin{exindent}
\begin{alltt}
\$ \textbf{tex {\bs}tex{\bs}docs{\bs}letter}
\end{alltt}
\end{exindent}

Instead, you must type the following, regardless of the operating system
you're using:

\begin{exindent}
\begin{alltt}
\$ \textbf{tex /tex/docs/letter}
\end{alltt}
\end{exindent}

If you don't, \TeX\ will complain that the control sequence
\cs{tex} is undefined.  Even more confusing errors may result
if the first subdirectory happens to be a valid control sequence.

Unfortunately, filenames containing forward slashes are not always
recognized by operating systems that use backslashes to delimit
filenames.  For example, under MS-DOS, \verb|IF EXIST C:/CONFIG.SYS|
returns false,\index{forward slash in MS-DOS and OS/2}
even when a file called \filename{config.sys} exists in
the root directory of drive C:.  Therefore, it is most convenient to
use filenames with backslashes when other commands will be used (for
example in a batch file).  Filenames in batch files must have any
backslashes translated into forward slashes before being passed to
\TeX.  MS-DOS's command processor isn't really powerful enough, but
the task is quite doable with \idx{JP Software}'s \program{4DOS} command
processor.  \program{4DOS}\index{4DOS!use in translating  backslashes} is a replacement for MS-DOS's normal
command processor; it offers many advantages over the normal
processor.  You can get a shareware version of \program{4DOS} from
many large MS-DOS archive sites on the Internet\footnote{For example,
{\tt oak.oakland.edu} in the directory \filename{/pub/msdos/4DOS}.} as well
as most large bulletin board systems.  The following lines, 
written in \ixx{\program{4DOS}}{4DOS}'s
extended batch language, will translate all backslashes in the
environment variable \envvar{TEXFN} to forward slashes:

\begin{shortexample}
:FixBacks
  Set Count=%@Index[%TexFn,\]
  If "%Count" eq "-1" Goto FixDone
  Set Place=%@Eval[%Count+1]
  Set TexFn=%@SubStr[%TexFn,0,%Count]/%@SubStr[%TexFn,%Place]
  Goto FixBacks
:FixDone
\end{shortexample}

These lines also work under OS/2 with JP Software's \program{4OS2}
command processor (akin to \program{4DOS};
shareware versions of \program{4OS2} are also available).  
The same task can be completed in a single line
with \ixx{\program{REXX}}{REXX} under OS/2:

\begin{shortexample}
TEXFN = TRANSLATE(TEXFN, '\', '/')
\end{shortexample}

\program{REXX} is an interpreted systems-programming language distributed
with OS/2.  \program{REXX} runs under many other IBM operating systems
in addition to OS/2.  Several versions of \program{REXX} are available for
MS-DOS and \Unix\ systems, too.

\subsubsection{Multiple documents}

Keep in mind\index{running tex@running \TeX!multiple documents} that \TeX\ 
interprets only the first option after the
format file name as a document name.  If you type the command line:

\begin{exindent}
\begin{alltt}
\$ \textbf{tex doc1 doc2}
\end{alltt}
\end{exindent}

\TeX\ will process only \filename{doc1.tex} as a document.  The result
of the above 
command line\index{command line!specifying multiple documents} will 
be exactly as if you had typed the
literal word ``doc2'' at the very end of the file \filename{doc1.tex}.

To get \TeX\ to process both \filename{doc1} and \filename{doc2} as documents,
type a command like the following:

\begin{exindent}
\begin{alltt}
\$ \textbf{tex doc1 {\bs}input doc2}
\end{alltt}
\end{exindent}

or, for complete clarity,

\begin{exindent}
\begin{alltt}
\$ \textbf{tex {\bs}input doc1 {\bs}input doc2}
\end{alltt}
\end{exindent}

In either case, the structure of your document files must allow them
to be concatenated together.  \TeX\ does not process \filename{doc1} and then
process \filename{doc2} separately.  Instead, it processes all of \filename{doc1} and
appends the text in the file \filename{doc2} directly onto the end of
\filename{doc1}.

\section{\protect\TeX\ Without Options}
\label{sec:runtex:texwoopts}

If you run \TeX\ without specifying any 
options\index{running tex@running \TeX!without options} at all, it prints out some
introductory information, then produces a \ixx{\texttt{**} prompt}{** prompt@\texttt{**} prompt} and stops.  
Here's what \emTeX's \filename{tex386} produces:

\begin{shortexample}
This is emTeX (tex386), Version 3.141 [3c-beta8]
**
\end{shortexample}

This is the only situation where the \verb|**| prompt occurs.  \TeX\
now expects you to enter text.  \TeX\ will interpret anything you type
after the \verb|**| prompt {\em exactly\/} as if you had typed it on
the command line.  All of the rules regarding slashes (described
in ``\nameref{run:sec:misinterpret}'' earlier) apply to commands that
you type at the \verb|**| prompt.  Remember, however, that input entered at the
\verb|**| prompt is not seen by the operating system, 
so don't use quotation marks or doubled backslashes.

%%%%this is not universally true.
%If you do enter options on the command line, \TeX\ prints
%the \verb|**| prompt followed by the options that it saw and then continues
%processing without waiting for input.  This is an
%easy way to tell if the operating system got in your way by
%interpreting part of the command line that you intended for \TeX.

The last line of a document is usually a command telling \TeX\ that the
document is finished.  \TeX\ responds by writing the last page of
the document and ending (returning control to the operating system).
If a document doesn't end with such a command,
\TeX\ will wait for more input.  When \TeX\ runs out of input, it
displays the \ixx{\texttt{*} prompt}{* prompt@\texttt{*} prompt} (single asterisk).  Anything
you type at the \verb|*| prompt is interpreted exactly as if it
occurred in the document that \TeX\ was processing.  You cannot specify
an alternate format or name a document at the \verb|*| prompt.  You
can, however, use \cs{input} to insert another file.

If \emTeX\ processes a document called \filename{unended.tex}:

\begin{shortexample}
% This is UNENDED.TEX
Now is the time for all good men
to come to the aid of their country.
% Note: there's no \bye command to end 
% TeX's processing...
\end{shortexample}

the following output appears on
the terminal:

\begin{shortexample}
This is emTeX (tex386), Version 3.141 [3c-beta8]
**unended
(unended.tex)
*
\end{shortexample}

You can exit from the \TeX\ program at the \verb|*| prompt by telling
\TeX\ that it should stop processing the current document.  Exactly how
you get out depends on the format file you use. The following table shows
the stop commands for some \TeX\ macro 
packages\index{Plain tex@Plain \TeX!stop command}\index{latex@\LaTeX!stop command}\index{SliTeX@\SliTeX!stop command}\index{Lollipop!stop command}\index{tex{}info stop command@\TeX{}info stop command}.

\vspace{-4pt}
\begin{xtable}{l|l}
  \bf Package & \bf Command \\
  \hline
  \tstrut
  Plain \TeX  & \cs{bye} \\
  \LaTeX      & \verb|\end{document}|\index{end@\protect \texttt {\protect \bs end}!document} \\
  \SliTeX     & \verb|\end{document}| \\
  Lollipop    & \cs{Stop} \\
  \TeX{}info  & \verb|@bye| \\
  \hline
\end{xtable}

If you do not know which format is in use, you can almost always get out with
the following steps:\footnote{Typing the break character (\,\key{Ctrl-C}
or \key{Ctrl-Break}\,, for example) sometimes works as well, but some 
implementations of \TeX\ don't respond to the break character at every
prompt.  On \Unix\ systems, use \key{Ctrl-D}.  Em\TeX{} responds to
\key{Ctrl-Z} followed by \key{Return}.} 

\begin{enumerate}
  \item Type a control sequence that \TeX\ will not recognize
        (control sequences \cs{undefined} and 
        \cs{gobbledygook} will work).
        \TeX\ will respond with the 
        question mark prompt.

  \item Enter \textbf{x} at the prompt.
        The question mark prompt is discussed fully in the section
        ``\nameref{sec:qprompt},'' later in this chapter.
\end{enumerate}

To continue the above example, if I type \cs{undefined} at
the \verb|*| prompt, \TeX\ will respond:

\begin{ttindent}
This is emTeX (tex386), Version 3.141 [3c-beta8]
**unended
(unended.tex)
*\textbf{{\bs}undefined}
! Undefined control sequence.
<*> {\bs}undefined

?
\end{ttindent}

If I enter \textbf{x} at the question mark prompt, \TeX\ will print
several informative messages and then end, returning control to the
operating system.

\begin{ttindent}
This is emTeX (tex386), Version 3.141 [3c-beta8]
**unended
(unended.tex)
*{\bs}undefined
! Undefined control sequence.
<*> {\bs}undefined

? \textbf{x}
No pages of output.
Transcript written on unended.log.
\end{ttindent}

One word of caution: some macro packages redefine the meaning of the
backslash (\bs)\index{bs (backslash)@\bs (backslash)!redefining} character so that it doesn't 
function as the
beginning of a control sequence (for example, the \TeX{}info format
uses the at-sign, ``@'').  In this case, you must precede the
undefined control sequence by the escape character, even if it
isn't the backslash.

\section{What About Errors?}
\label{sec:errors}

When you write \TeX\ documents, you will occasionally make 
mistakes\index{tex@\TeX!errors}\index{errors in tex@errors in \TeX}\index{error messages} and 
as a result, \TeX\ won't be able to process your document.
There are six broad classes of mistakes you're likely to make:

\begin{enumerate}
  \item Naming documents or files that \TeX\ cannot find.
  \item Misspelling the name of a \TeX\ control sequence.
  \item Failing to close an environment or forgetting to insert a
        closing brace after an opening brace.
  \item Using math operators outside of math mode or forgetting to close
        an opened math environment.
  \item Requesting a font that \TeX\ cannot find.
  \item Everything else.
\end{enumerate}

\TeX\ is legitimately criticized for having error messages that are
very difficult to understand.  \TeX\ frequently provides far more
information than you really need or want, and the excess information
often obscures the actual cause of the error.

Nevertheless, understanding what \TeX\ does when it encounters an error will
help make error messages easier to understand.  Let's begin with an example.
Example~\ref{ex:badfonttex} shows a simple \LaTeX\ document using the New Font
Selection Scheme (NFSS), which contains an error.\footnote{The details of
\LaTeX\ and the NFSS are discussed in Chapter~\ref{chap:macpack}, {\it
\nameref{chap:macpack}}.  It is simply convenient to use a concrete example
in this case.  Do not be concerned if you do not use \LaTeX\ or the NFSS.}
The error is that the control sequence \cs{Large}, which has been redefined
to request a 17pt font, contains a typo: \cs[fontsize]{fontsiz} should be
\cs{fontsize}.\footnote{In practice, redefining a control sequence like
$\backslash$\texttt{Large} in a document is a bad idea.  I've done it here
only to provide an example of an error that occurs inside another macro.}

\begin{example}{ex:badfonttex}{The Document BADFONT.TEX}
\documentstyle{article}
% The following definition changes the font that LaTeX 
% uses for the 'Large' font.  I have introduced a typo
% into the definition, ``\fontsiz'' should be ``\fontsize''.
%
% The  first time a \Large font is requested, an error will occur.
%
\renewcommand{\Large}{\fontsiz{17}{20pt}\selectfont}
\begin{document}

This text precedes the first section header.

% Note: LaTeX uses the \Large font in section
% headers...this will fail in a confusing way
% because the error is deep within the definition
% of \section where \Large is used...
%
\section{First Section}

This is the first and only sentence of the first section.

\end{document}
\end{example}

\newpage
When \TeX\ processes this file, using \LaTeX\ with the New Font
Selection Scheme, it produces these error messages:

\begin{shortexample}
This is TeX, Version 3.1415 (C version 6.1)
LaTeX Version 2.09 <25 March 1992> with NFSS2
(badfont.tex (/work/nutshell/texguide/styles/latex/article.sty
Standard Document Style `article' <14 Jan 92>.
(/work/nutshell/texguide/styles/latex/art10.sty)) (badfont.aux)
(/usr/local/lib/tex/inputs/nfss2/T1cmr.fd)
! Undefined control sequence.
\Large ->\fontsiz 
                  {17}{20pt}\selectfont 
<argument> \reset@font \Large 
                              \bf 
\@sect ...x \ifdim \@tempskipa >\z@ \begingroup #6
                                      \relax \@hangfrom {\hskip ...
l.21 \section{First Section}
                            
? 
\end{shortexample}

These messages exemplify
the kind of confusing error messages that \TeX\ produces.  Remember
the following rule:
Always look at the first and last line of the \TeX\
error message when trying to figure out what went wrong
and where it went wrong.  In this case, the first line is:

\begin{shortexample}
! Undefined control sequence.
\end{shortexample}

and the last line is:

\begin{shortexample}
l.21 \section{First Section}
\end{shortexample}

The error is that the control sequence \cs[fontsize]{fontsiz} is not defined and
\TeX\ was processing line 21 of the file when it occurred.

\subsection{Log Files}
\label{sec:logfiles}

You don't have to remember or write down the error messages that \TeX\
produces.  When \TeX\ processes a document, it produces 
a transcript\index{log files} of 
everything that occurs; you can refer to this transcript later if
you need to recall what errors occurred when you processed your
document.  \TeX\ stores this transcript in a file which has the same
name as the document and the extension \filename{.log}.  
For example,
if you process \filename{main.tex}, \TeX\ produces a
transcript in \filename{main.log}.

\begin{note}{NOTE}
Log files go in the current directory.
\TeX\ always places the log file in the current directory, even if
you specify a path when you format your document.  
For example,
if you process \filename{lectures/main.tex}, \TeX\ produces a
transcript in \filename{main.log}, not in \filename{lectures/main.log}
as you might expect.
\end{note}

\subsection{Interpreting \protect\TeX\ Error Messages}

The first line of a \TeX\ error message begins with an exclamation
point followed by the text of the 
message\index{error messages!interpreting}.  The lines that
follow it show the context in which the error occurred.\footnote{The number 
of lines of context shown in the error message is determined by the value of the
built-in parameter $\backslash$\texttt{errorcontextlines}.}  In
the previous example, the error message indicates an
\verb|Undefined control sequence|.  This means that \TeX\
encountered a control sequence which was not previously defined.

The final line, which says \verb|l.21 \section{First Section}|, occurs right above the question mark prompt.  
It identifies the line in your document that \TeX\
was processing when it encountered the error.  The error occurred when \TeX\
was at line 21 of the file, and that line began with 
\verb|\section{First Section}|.

Between the first and last lines, \TeX\ prints a detailed description
of how the error occurred.  This is necessary because the error may have
occurred inside the replacement text of a macro that you used.  When
\TeX\ encounters a control sequence, like \cs{section}, it has to
look up the definition to figure out how to
typeset your document.  The definition of a control sequence may
contain other control sequences which also have to be interpreted.  It
is possible for \TeX\ to be several levels deep, as it was in this
case, when an error occurs.  To give the person who wrote the
definition of the control sequence an opportunity to figure out what
went wrong, \TeX\ prints out a \ixx{\textit{trace back}}{trace back} of what happened
before the error.  As you gain experience with a particular macro
package, you'll find the intervening lines more
meaningful.

Correcting an error depends entirely on the nature of the error.  In
this case, all you need to do is correct the typo.

As stated above, there are six general classes of errors you might encounter
when you run \TeX.  The following sections briefly describe each class.

\subsection{Naming a File \protect\TeX\ Cannot Find}

Missing documents are discussed in the section called
``\nameref{sec:userfiles},''
earlier in this chapter.

If you request a \ixx{format file}{format files!missing} that 
cannot be found, \TeX\ issues the
following warning message:

\begin{shortexample}
Sorry, I can't find that format; will try the default
\end{shortexample}

\TeX\ then attempts to typeset your document with Plain \TeX.\footnote{It
is possible to make other formats the default, but in practice I've never
seen it done.}  To correct this problem, run \TeX\ with the correct
format file name.  If the format file isn't available, you will have
to build it.  Chapter~\ref{chap:macpack}, {\it\nameref{chap:macpack}},
describes how to build format files for several common macro packages.

\subsection{Misspelling a Control Sequence Name}

Misspelling 
a \ixx{control sequence name}{control sequences!misspelling names of} is 
one of the most common errors.  
For example, you might type
\verb|\ipnut{chap1}| in your document instead of
\verb|\input{chap1}|.  The solution is straightforward and usually
easy to identify.  In many cases, you can simply proceed after
encountering this error.  Although \TeX\ may not typeset your document
correctly, you can continue to look for other errors.  The section called
``\nameref{sec:qprompt},'' later in this chapter, describes how to
continue after an error.

In some cases, \TeX\ may become badly confused by a misspelled control
sequence name, in which case you should give up and fix the spelling error
before trying to process your document further.  This may happen if
you misspell a \LaTeX\ environment name (\verb|\begin{itemze}| instead of
\verb|\begin{itemize}|, for example),\ff{Technically, \LaTeX\
environment names are not a control sequence names, but they behave in very
much the same way in \LaTeX.} which will make \LaTeX\ misinterpret
many of the control sequences which follow.

Sometimes \TeX\ will complain that a control sequence is undefined
when you {\em know\/} that the control sequence is spelled correctly.
When this occurs, make sure that you are using the correct format file
(see the section ``\nameref{sec:clineopts}'' earlier in this chapter),
loading the correct macro files, and using
the correct style options.

\subsection{Failure to Close an Environment}

Failure to close an environment\index{environment!failing to close} is 
another very common error.  There
are several distinct errors in this category:

\begin{itemize}
  \item Failure to insert a closing brace (\verb|}|) for each opening
        brace may cause a ``\TeX\ capacity exceeded'' error when \TeX\ 
        processes
        your document.  This happens because sometimes \TeX\ tries to read 
        everything
        between braces into memory.  If the closing brace is absent, \TeX\
        may run out of memory.

        If the braces are supposed to enclose the argument to a macro,
        you may also get this error:\index{braces!forgetting a closing brace}

\begin{ttindent}
! Paragraph ended before \textit{macro} was complete.
\end{ttindent}

        In order to help detect errors of this type, \TeX\ doesn't ordinarily
        allow the argument of a macro to consist of more than one paragraph,
        so the first blank line after the place where you failed to type
        the closing brace may produce this error.

  \item Failure to close a begin/end environment pair
        causes \LaTeX\ to complain about a mismatch when it encounters
        the next \verb|\end{|\texttt{\textit{environment}}\verb|}|
        command.\index{end@\protect \texttt {\protect \bs end}!environment}

  \item If your document ends with an open environment, \TeX\ will warn
        you that:

\begin{ttindent}
! ({\bs}end occurs inside a group at level \textit{n})
\end{ttindent}

        where \texttt{\textit{n}} is the number of open groups, usually 1.

  \item Failure to close a mathematics environment will result in the
        error: 
        
        \begin{shortexample}
! Missing $ inserted
        \end{shortexample}

        when \TeX\
        reaches a macro that does not make sense in mathematics mode
        (like \cs{section}) or when a surrounding group ends.
\end{itemize}

\subsection{Math Outside of Math Mode}

\TeX\ has a lot of operators for special treatment of mathematical
formulas.  These operators must occur 
inside \ixx{\emph{mathematics mode}}{mathematics mode},\index{mathematics!operators}
which is usually delimited by dollar signs in your 
document.\index{mathematics!minor mode!using math operators outside}

If you attempt to use math operators, superscripts, subscripts, or other
math-mode control sequences outside of mathematics mode, the following
error will occur:\index{\$@\protect \texttt {\protect \bs \$}}

\begin{shortexample}
! Missing $ inserted 
\end{shortexample}

This is your clue that
a mathematics environment has not been closed properly or that you failed
to open one before using a math-mode operator. 

For example, the underscore character is usually defined to be a
math-mode operator which starts a subscript, in other words
\verb|H${}_2$O| produces ``H${}_2$O'' in your document.\footnote{The empty curly
braces are necessary in this example because otherwise there wouldn't
be anything in front of the subscript command for \TeX\ to subscript
below.  Typically, another digit or symbol would come in front of the
subscript command.  In this case, I didn't want the ``H'' and ``O'' in the
math environment because letters are printed in math-italics in math 
mode.}  If you use the underscore outside of mathematics mode, such
as in regular text:

\begin{shortexample}
The file ``test_one'' contains the ...
\end{shortexample}

\TeX\ will respond:

\begin{shortexample}
! Missing $ inserted
<inserted text>
                $
<to be read again>
                  _
The file ``test_
                one'' contains the ...
?
\end{shortexample}

In \LaTeX, the easiest way around this problem is to enclose the offending
text in a ``verbatim'' macro, like this:

\begin{shortexample}
  The file ``\verb+test_one+'' contains the ...
\end{shortexample}

Note that the argument to the \cs{verb} macro is delimited by any two
identical characters (in this case, two ``\verb|+|'' signs).

In Plain \TeX\ or another format, the problem can be circumvented in
similar ways; consult the reference for the format you are using.
Also consult Table~\ref{tab:acttype} in 
Chapter~\ref{chap:tex}, \textit{\nameref{chap:tex}},
for a list of special characters and how to type them in your documents.

\subsection{Missing Fonts}

The first\index{fonts!missing} time that 
you use each font, \TeX\ loads font metric
information about the font.  The font metric information, stored in a
\ext{TFM} file, includes information about the sizes of each character
as well as kerning and ligature information.  These topics are
discussed fully in Chapter~\ref{chap:fonts}, {\it\nameref{chap:fonts}}.

If you request a font that does not exist, for example
\filename{crm10} (a misspelling of \filename{cmr10}), 
\TeX\ cannot find a \ext{TFM} file for the font and
therefore displays:

\begin{shortexample}
Font \myfont=crm10 not loadable: Metric (TFM) file not found
\end{shortexample}

This means that \TeX\ attempted to associate the font described by the
\ext{TFM} file \filename{crm10.tfm} with the control sequence
\cs{myfont}, but the \ext{TFM} file didn't exist.  You
must have a \ext{TFM} file for every font that you use.

A second kind of error---actually, a warning---occurs when you are
using the New Font Selection Scheme 
(NFSS)\index{New Font Selection Scheme!version 1 (NFSS)!errors}.  The NFSS performs font
substitution, if possible, when an unknown font is requested.  The
NFSS is described in Chapter~\ref{chap:macpack},
{\it \nameref{chap:macpack}}, in the ``\nameref{sec:latex}'' section.

\subsection{Everything Else}

There are, unfortunately, lots of other errors that can occur.  Some
of the errors are directly related to \TeX\ while others are warning and
error messages associated with particular macro packages.  There is
no way to catalog every one of them or suggest what can be done in every
case.  

The best advice I can offer is to isolate the problem
in as small a document as possible, consult the references you have available
very carefully, and, if all else fails, forward your problem to one of the
electronic forums that deal with \TeX\ 
(the \ixx{\texttt{Info-TeX} mailing list}{Info-TeX mailing list@\texttt{Info-TeX} mailing list}
and the \ixx{\texttt{comp.text.tex} newsgroup}{comp.text.tex newsgroup@\texttt{comp.text.tex} newsgroup}, for example).

\section{The Question Mark Prompt}
\label{sec:qprompt}
\index{? (question mark) prompt}

When \TeX\ encounters an error, it displays an error message and a
summary of the error, as described in the section called
``\nameref{sec:errors}'' earlier in this chapter.  Following the
error, \TeX\ normally stops and displays the question mark prompt.

\newpage
If you type a question mark of your own at the prompt, \TeX\ displays
the actions available to you:

\begin{ttindent}
? ?
Type <return> to proceed,
S to scroll future error messages,
R to run without stopping, Q to run quietly,
I to insert something, E to edit your file,
1 or ... or 9 to ignore the next 1 to 9 tokens
of input, H for help, X to quit.
?
\end{ttindent}
%    \caption{The \protect\TeX\ Question Mark Prompt}
%    \label{fig:qprompt}
%  \end{center}
%\end{figure}

You can type any of the following responses at the question mark
prompt:\index{? (question mark) prompt}
\begin{itemize}

\item{Type \key{Return} to proceed.}

Simply typing Return will cause \TeX\ to ignore the error and proceed.
Depending on the nature of the error, this may cause more errors
immediately or later on.

\item{Type \key{S} to scroll future error messages.}

Typing \key{S} tells \TeX\ to continue and not to stop for most future errors.
\TeX\ will continue to print the error messages, both to the terminal
and to the log file, but it will not display the question mark prompt
again.

\TeX\ will still stop and ask about missing files.

\item{Type \key{R} to run without stopping.}

The \key{R} option is just like \key{S} except that it tells \TeX\ to ignore
missing files as well.  \TeX\ will proceed blindly forward as best as it
can.  You will still see all of the error messages scroll by as \TeX\
proceeds.

\item{Type \key{Q} to run quietly.}

This option is just like \key{R} except that error messages are not
displayed on the screen.  The messages are saved in the log
file, however, even though they are not displayed.

\item{Type \key{I} to insert something.}

If you notice a simple typo, you can correct it with the \key{I} command.
For example, suppose that \TeX\ complains of an undefined control
sequence: \verb|\cte{kn:art1}|.
You recognize that this should have 
been \verb|\cite{kn:art1}|.
You can insert the correct control sequence by responding
\verb|i\cite| to the question mark prompt.  \TeX\ will ignore the
misspelled control sequence and insert the (correct) sequence \cs{cite}
in its place.

Inserting words or commands at the question mark prompt \emph{does not}
change your input file.  It simply instructs \TeX\ to \emph{pretend} that your
file contained a different sequence of words and commands.  You must change
the input file with an editor, or the same error will occur the next time
you format the document.

\newpage
\item{Type \key{E} to edit your file.}

This option terminates \TeX.  If your system is configured appropriately,
an editor will be loaded automatically, and the cursor will be placed
at the offending line in your input file.  This is not possible in
all environments.  

\item{Type \key{{\em n}}\,, where {\em n} is a number between 1 and 9.}

This option tells \TeX\ to ignore some of the input.  After skipping
over the number of \textit{tokens} you request, \TeX\ returns to the
question mark prompt so that you can delete more tokens.  

What's a \idx{token}?  When \TeX\ reads an input file, it breaks each
line down into the smallest, indivisible chunks that have meaning.
These are called tokens.  For the most part, tokens are individual characters.
The exceptions are control sequences\index{control sequences!as tokens}, which are single tokens, and white
spaces which are also single tokens.  There {\em are\/} other exceptions
(and more technical definitions of ``token''), but that's the gist of it.

\item{Type \key{H} for help.}

Typing \key{H} displays a slightly more verbose description of the error
that occurred and, usually, suggests the nature of the corrective action
that you might take.

\item{Type \key{X} to quit.}

Typing \key{X} tells \TeX\ to stop immediately and ends the \TeX\ program.
If there are any completed pages (pages processed before the error,
in other words), they are written to the \ext{DVI} file before \TeX\
ends.

\end{itemize}

% Local Variables:
% mode: latex
% TeX-master: "driver"
% End:


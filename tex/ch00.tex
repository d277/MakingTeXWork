\chapter*{Preface} \makeatletter\@mkboth{Preface}{Preface}\makeatother
\RCSID$Id: ch00.tex,v 1.1 2002/08/23 14:58:45 nwalsh Exp $
\addcontentsline{toc}{chapter}{Preface}
\chaptermark{Preface} 
\pagenumbering{roman}
\setcounter{page}{25} % this is after the page-broken TOC...

\ifincludechapter\else\endinput\fi

\TeX\ is a tool for creating professional quality, typeset pages of any kind.
It is particularly good, perhaps unsurpassed, at
\ixx{typesetting mathematics}{typesetting!mathematics}%
\index{mathematics!typesetting}; 
as a result, it is widely used in scientific writing.  Some of its other
features, like its ability to handle
\ixx{multiple languages}{typesetting!multilingual documents}%
\index{international!typesetting} in the same document and the fact that the
content of a document (chapters, sections, equations, tables, figures, etc.)
can be separated from its form (typeface, size, spacing, indentation, etc.)
are making \TeX\ more common outside of scientific and academic circles.

Designed by \ixx{Donald Knuth}{Knuth, Donald} in the late 1970s, more than a
decade of refinement has gone into the program called ``\TeX'' today.  The
resulting system produces publication-quality output while maintaining
portability across an extremely wide range of platforms.

Remarkably, \TeX\ is free.  This fact, probably as much as any other, has
contributed to the development of a complete ``\TeX\ system'' by literally
thousands of volunteers.  \TeX, the program, forms the core of this
environment and is now supported by hundreds of tools.

\section{Why Read This Book?}

This book is for anyone who uses \TeX.  Novices will need at least one other
reference, because this book does not describe the nuts and bolts of writing
documents with \TeX\ in any great detail.

If you are new to \TeX, there is much to learn.  There are many books that
describe how to use \TeX.  However, the focus of this book is mostly at a
higher level.  After digesting Chapter~\ref{chap:overview},
\textit{\nameref{chap:overview}}, you should be able to proceed through the
rest of the book without much difficulty even if you have never seen \TeX\
before.  So, if you are a system administrator interested in learning enough
about these programs to install and test them for your users, you should be
all set.  If you are interested in learning how to write documents with \TeX,
this book will be helpful, but it will not be wholly sufficient.

Why do you need this book at all?  Although many individual components of the
\TeX\ system are well documented, there has never before been a complete
reference to the whole system.  This book surveys the entire world of \TeX\
software and helps you see how the various pieces are related.

A functioning \TeX\ system is really a large collection of programs that
interact in subtle ways to produce a document that, when formatted by \TeX,
prints the output you want.  All the different interactions that take place
ultimately result in less work for you, the writer, even though it may seem
like more work at first.  Heck, it may {\em be\/} more work at first, but in
the long run, the savings are tremendous.

Many books about \TeX\ refer the reader to a ``local guide'' for more
information about previewing and printing documents and what facilities exist
for incorporating special material into documents (like special fonts and
pictures and figures).  In reality, very few local guides exist.

The \TeX\ environment is now mature and stable enough to support a more
``global guide.''  That is what this book attempts to be.  It goes into detail
about previewing and printing, about incorporating other fonts, about adding
pictures and figures to your documents, and about many other things overlooked
by other books.

Because fonts play a ubiquitous role in typesetting, this book is also about
\MF, the tool that \ixx{Donald Knuth}{Knuth, Donald} designed for creating
fonts.

\section{Scope of This Book}

Here's how the book is laid out.

\textbf{Part~\ref{part:intro}: {\it \nameref{part:intro}}}.

Chapter~\ref{chap:overview}, {\it \nameref{chap:overview}}.  If you don't know
anything about \TeX\ at all, this chapter will help you get started.  If
you're a system adminstrator charged with the task of installing and
maintaining \TeX\ tools, you'll get enough information to do the job.

Chapter~\ref{chap:editing}, {\it \nameref{chap:editing}}.  An overview of some
environments that can make working with \TeX\ documents easier.  It describes
some editors that ``understand'' \TeX, how to integrate \TeX\ into a
``programmer's editor,'' spellchecking, revision control, and other aspects
of \TeX{}nical editing.

Chapter~\ref{chap:running}, {\it \nameref{chap:running}}.  The mechanics of
running \TeX, the program.  It discusses what things \TeX\ needs to be able to
run, how to start \TeX, command-line options, leaving \TeX, and recovery from
errors.

Chapter~\ref{chap:macpack}, {\it \nameref{chap:macpack}}.  Overview of \TeX\
macro packages.  This chapter describes how to make a format file, the major
general-purpose writing packages, some special-purpose writing packages,
how to make slides for presentations, and how to handle color in \TeX.

\textbf{Part~\ref{part:elements}: {\it \nameref{part:elements}}}.

Chapter~\ref{chap:fonts}, {\it \nameref{chap:fonts}}.  This chapter explores
the issues that need to be addressed when using fonts.  Many of these issues
are not particularly \TeX-related, but \TeX\ is very flexible, and it's
important to understand the tradeoffs that must be made.  This chapter also
examines some \TeX-specific issues: font selection, files that \TeX\ needs,
automatic font generation, and virtual fonts.

Chapter~\ref{chap:pictures}, {\it \nameref{chap:pictures}}.  How many ways are
there to include pictures and figures in a \TeX\ document?  Lots and lots of
ways.  This chapter examines them all.

Chapter~\ref{chap:foreign}, {\it \nameref{chap:foreign}}.  \TeX\ is well
qualified to do international typesetting.  This chapter looks at the issues
that are involved: representing international symbols in your input file, what
\TeX\ produces, getting the right fonts, multiple languages in the same
document, and macro packages and style files that solve some of these
problems.  Some strategies for dealing with very difficult languages (like
Japanese and Arabic) are also explored.

Chapter~\ref{chap:printing}, {\it \nameref{chap:printing}}.  What goes in has
to come out.  This chapter tells the what, where, why, and how of printing
your documents.

Chapter~\ref{chap:preview}, {\it \nameref{chap:preview}}.  Save paper; preview
your documents before you print them.

Chapter~\ref{chap:online}, {\it \nameref{chap:online}}.  Online documentation
is becoming increasingly popular.  This chapter explores different ways that both
typeset and online documentation can be produced from the same set
of input files.

Chapter~\ref{chap:mf}, {\it \nameref{chap:mf}}.  Sometimes it is necessary or
desirable to create a special version of a standard \TeX\ font.  Maybe you
really need the standard 10pt font at 11.3pt.  This chapter will tell you how
to work with existing \MF\ fonts.  It {\em won't} tell you how to create your
own fonts; that's a whole different story.

Chapter~\ref{chap:bibtex}, {\it \nameref{chap:bibtex}}.  Maintaining a
bibliographic database can be a great timesaver.  This chapter looks at the
\BibTeX\ program and other tools for building and using bibliographic
databases.  It also discusses the creation of indexes and glossaries.

\textbf{Part~\ref{part:tools}: {\it \nameref{part:tools}}}.

Chapter~\ref{chap:freetex}, {\it \nameref{chap:freetex}}.  Many \TeX\
environments are freely available.  This chapter describes public domain,
free, and shareware versions of \TeX.

\newpage
Chapter~\ref{chap:commercialtex}, {\it \nameref{chap:commercialtex}}.  A
large, complex system like \TeX\ can be overwhelming (although I hope less so
after you read this book ;-).  One of the advantages of selecting a commercial
implementation of \TeX\ is that some form of customer support is usually
provided.  Still other commercial implementations offer features not found in
any free releases.  This chapter describes several commercial \TeX\ releases.

Chapter~\ref{chap:mac}, {\it \nameref{chap:mac}}.  Many issues
discussed in this book apply equally to all platforms, including the Macintosh
platform, but the Mac has its own special set of features.  This chapter looks
at some versions of \TeX\ and other tools designed specifically for use on the
Mac.

Chapter~\ref{chap:utils}, {\it \nameref{chap:utils}}.  This chapter lists many
of the the utilities available in the CTAN archives and provides a brief
description of what they do.

Appendix~\ref{chap:fileext}, {\it \nameref{chap:fileext}}.  Lots of files can
be identified by their extensions.  This appendix lists the extensions that
are most often seen in conjunction with \TeX\ and describes what the
associated files contain.

Appendix~\ref{app:fonts}, \textit{\nameref{app:fonts}}.  Examples of many
\MF\ fonts available from the CTAN archives.

Appendix~\ref{app:resources}, \textit{\nameref{app:resources}}.  A complete
list of the resources described in this book.  

Appendix~\ref{app:examples}, \textit{\nameref{app:examples}}.  This appendix
contains examples (scripts, batch files, programs) that seemed too long to
place in the running text.

{\it Bibliography}.  Where I learned what I know.  Also, where you can look
for more information about specific topics.

\section{Conventions Used in This Book}

The following typographic conventions are used in this book:

\begin{conventions}

\item [{\it Italic}] is used for filenames, directories, user commands, program names, and macro packages (except where the distinctive logo type is used).  Sometimes italics
      are used in the traditional way for emphasis of new ideas and important
      concepts.

\item [{\tt Typewriter}] is used for program examples, FTP sites, \TeX\
      control sequences, and 
      little bits of \TeX\ syntax that appear in running text (for example,
      typewriter text in a reference to the \LaTeX\ \verb|picture|
      environment is a clue that \LaTeX\ literally uses the word ``picture''
      to identify this environment).

\item [{\tt\bf Typewriter Bold}] is used in examples to show the user's actual
      input at the terminal.

\newpage
\item [{\tt\it Typewriter Italic}] identifies text that requires a
      context-specific substitution.  For example, \textit{\texttt{filename}}
      in an example would be replaced by some particular filename.

\item [Footnotes] are used for parenthetical remarks.  Sometimes, lies are
      spoken to simplify the discussion, and the footnotes restore the lie to
      the truth. (And sometimes they don't ;-)

% 
% ``And sometimes they don't ;-)'' used to be a footnote.  And it was funnier
% that way.  But for some reason, a footnoterule appeared on the following
% page, and I couldn't get rid of it except by removing this footnote.
%

\end{conventions}

Filename extensions, like ``\textit{.tex}'' in \filename{book.tex}, are shown
in uppercase letters when referring to a particular type of file. For example,
a \TeX\ Font Metric or \ext{TFM} file would be \filename{somefile.tfm}. The
actual extension of the file may be different (upper or lowercase, longer or
shorter) depending on your operating environment.

When the shell prompt is shown in an example, it is shown as a dollar sign,
{\tt\$}.  You should imagine that this is your system's
prompt, which might otherwise be \verb|%|, \verb|C>|,
\verb|[C:\]|, or a dialog box.

When spaces are important in an example, the ``{\ttspace}'' character is used
to emphasize the spaces.  Effective,\ttspace isn't\ttspace it?

In some places, I refer to specific keys that you should press.  When it's
important that I mean pressing particular keys and not typing something, I
emphasize the keys.  For instance, an example that includes \key{Enter} means
that you should literally press the Enter or Return key.  The sequence
\key{Ctrl-D} means that you should press and hold the ``Control'' and ``d''
keys simultaneously.  Control-key combinations aren't case sensitive, so you
needn't press the shift key.

%Sometimes I say things that I don't expect you to take seriously.
%I've put the Usenet international standard smiley face
%symbol\footnote{;-)!} ``;-)'' next to my little jokes if I'm afraid that
%they aren't very good and that you might take me seriously where I'm
%joking.

\section{How to Get \protect\TeX}

\TeX\ and the other programs mentioned in this book are available from a
number of places.  It's impossible to list all of the places where you might
find any given tool, but there is one place where you will almost certainly
find {\em every\/} tool: the \ixx{Comprehensive \TeX\ Archive Network (CTAN)}{Comprehensive tex Archive Network@Comprehensive \TeX\ Archive Network|see{CTAN}}.

This network is a fully-mirrored anonymous FTP hierarchy in three countries.
Always use the \ixx{FTP site}{CTAN!FTP access} that is geographically closest
to you.  The following table lists the current members of CTAN as of
July, 1993:

\begin{xtable}{l|l|l|l}
  \bf Geographic & & & \bf Top Level \\ 
  \bf Location & \bf Site & \bf IP Address & \bf Directory \\[2pt] 
  \hline 
  \tstrut
  United States & \tt ftp.shsu.edu & 192.92.115.10 & \it /tex-archive \\
  England & \tt ftp.tex.ac.uk & 131.151.79.32 &  \it /tex-archive \\ 
  Germany & \tt ftp.uni-stuttgart.de & 129.69.8.13& \it /tex-archive \\[2pt] 
  \hline
\end{xtable}

\newpage
You may also access the CTAN archives by \ixx{electronic mail}{CTAN!e-mail access} if you do not have FTP access.  For up-to-date instructions about the
mail server, send the single-line message \verb|help| to
\filename{fileserv@shsu.edu}.

\subsection{Where Are the Files?}

Every CTAN mirror site has the same well-organized 
\ixx{directory structure}{CTAN!directory structure}.  
The top-level
directory also contains a complete catalog of current files organized by name,
date, and size.  The catalogs are named \filename{FILES.byname},
\filename{FILES.bydate}, and \filename{FILES.bysize}, respectively, in the top level
directory.  The top-level directory contains the following subdirectories:

\begin{xtable}{>{\it}l|l}
  \textnormal{\bf Directory} & \bf Description of Contents \\[2pt] 
  \hline 
  \tstrut
  tools    & Archiving tools (\program{unzip}, \program{tar}, 
             \program{compress}, etc.) \\ 
  biblio   & Tools for maintaining bibliographic databases \\  
  digests  & Electronic digests (\TeX{}hax, UK\TeX, etc.)\\  
  info     & Free documentation, many good guides \\  
  dviware  & Printing and previewing software \\  
  fonts    & Fonts for \TeX \\  
  graphics & Software for working with pictures and figures \\  
  help     & Online help files, etc. \\
  indexing & Indexing and glossary building tools \\  
  language & Multi-national language support \\  
  macros   & Macro packages and style files \\  
  misc     & Stuff that doesn't fit in any other category \\ 
  support  & Tools for running and supporting \TeX\\  
  systems  & OS-specific programs and files\\  
  web      & Sources for \TeX\ programs (in \Web) \\[2pt] 
  \hline
\end{xtable}

The archives at \texttt{ftp.shsu.edu} and \texttt{ftp.tex.ac.uk} also support
\ixx{\program{gopher}}{CTAN!gopher access} access to the archives.  The UK
\program{gopher} supports indexed access to the archives.  A \ixx{World Wide Web (hypertext) interface}{CTAN!hypertext access (World Wide Web)} to the
archives is available from:

\begin{shortexample} 
http://jasper.ora.com/ctan.html
\end{shortexample}

This interface includes brief descriptions of many packages and the ability to
perform keyword and date searches.

\subsection{Getting Software Without FTP}

The \ixx{electronic alternatives}{CTAN!other electronic access} to FTP,
described in the section ``\nameref{sec:gettingexamples}'' of this chapter are
also viable alternatives for getting software from the CTAN archives.

In addition, there are a number of ways to get distributions through
\ixx{nonelectronic}{CTAN!nonelectronic access} channels.  The names and
addresses of these sources are listed in Appendix~\ref{app:resources},
\textit{\nameref{app:resources}}.

\vfill\eject
You can get many of the popular \TeX\ distributions on diskette from the
\ixx{\TeX\ Users Group (TUG)}{TeX User's Group (TUG)@\TeX\ Users Group (TUG)!getting TeX from@getting \TeX\ from}.  Emacs, \program{Ghostscript}, and other packages by the
\ixx{Free Software Foundation (FSF)}{Free Software Foundation (FSF)!getting tex@getting \TeX} are available on tape directly from the FSF.  You may also find large
bulletin board systems that support \TeX\ (for example, Channel1 in Cambridge,
MA)

\subsection{Getting Examples From This Book}
\label{sec:gettingexamples}

All of the substantial code fragments and programs printed in this book are
available online.  The \ixx{examples}{examples (getting)} in this book are
all in \ixx{\program{Perl}}{Perl TeX examples@Perl \TeX\ examples}, a language
for easily manipulating text, files, and processes. 
I decided to use \program{Perl} simply because it
is available for every platform
discussed in this book.  It is the only ``universal'' scripting language that
will work under MS-DOS, OS/2, \Unix, and the Macintosh.  All of the scripts in
this book can be converted to a different scripting language (the various
\Unix\ shells or something like \program{4DOS}'s extended batch language for MS-DOS and
OS/2) if you prefer.  I've tried to write the \program{Perl} scripts in a
straightforward way so that any given task won't be too difficult.

The examples are available electronically in a number of ways: by FTP,
FTPMAIL, BITFTP, and UUCP. The cheapest, fastest, and easiest ways are listed
first. If you read from the top down, the first one that works is probably the
best. Use FTP if you are directly on the Internet. Use FTPMAIL if you are not
on the Internet but can send and receive electronic mail to Internet sites
(this includes CompuServe users). Use BITFTP if you send electronic mail via
BITNET. Use UUCP if none of the above work.

\begin{note}{NOTE}
The examples were prepared using a \Unix\ system. If you are running
\Unix, you can use them without modification. If you are running on another
platform, you may need to modify these examples to correct the end-of-line
markers. For example, whereas under \Unix\ every line ends with a line feed
character (the carriage return is implicit), under DOS every line must end
with explicit carriage return and line feed characters.
\end{note}

\subsubsection{FTP}

To use \ixx{FTP}{FTP (getting examples via)}, you need a machine with direct access
to the Internet. A sample session is shown below.

\begin{ttindent}
\${\bf ftp ftp.uu.net} Connected to ftp.uu.net.  
220 ftp.UU.NET FTP server (Version 6.34 Oct 22 14:32:01 1992) ready.  
Name (ftp.uu.net:prefect): {\bf anonymous} 
331 Guest login ok, send e-mail address as password.  
Password:{\bf prefect@guide.com}\ \ \ \ \ \ \ {\rm\it\small (use your user name and host here)} 
230 Guest login ok, access restrictions apply.  
ftp>{\bf cd /published/oreilly/nutshell/maketexwork} 
250 CWD command successful.  
ftp>{\bf get README} 
200 PORT command successful.  
150 Opening ASCII mode data connection for README (xxxx bytes).  
226 Transfer complete.  
local: README remote: README 
xxxx bytes received in xxx seconds (xxx Kbytes/s) 
ftp>{\bf binary}\ \ \ \ \ {\rm\it\small (select binary mode for compressed files)} 
200 Type set to I.
{\rm\it\small \ \ (Repeat get commands for the other files.}  
{\rm\it\small \ \ \ They are listed in the README file.)}  
ftp>{\bf quit} 221 Goodbye.  
\$
\end{ttindent}
\vspace{6pt}

\subsubsection{FTPMAIL}

\ixx{FTPMAIL}{FTPMAIL (getting examples via)} is a mail server available to anyone
who can send electronic mail to, and receive it from, Internet sites. This
includes most workstations that have an email connection to the outside world
and \ixx{CompuServe}{CompuServe (getting examples via)} users.  You do not need to
be directly on the Internet.

Send mail to {\it ftpmail@decwrl.dec.com}. In the message body, give the
name of the anonymous FTP host and the FTP commands you want to run.  The
server will run anonymous FTP for you and mail the files back to you.  To get
a complete help file, send a message with no subject and the single word
\texttt{help} in the body. The following is an example mail session that should get
you the examples. This command sends you a listing of the files in the
selected directory and the requested example files. The listing is useful if
there's a later version of the examples you're interested in.

\begin{ttindent}
\${\bf mail ftpmail@decwrl.dec.com} 
Subject: {\bf{}reply prefect@guide.com}\ \ \ \ {\rm\it\small (where you want files mailed)} 
{\bf{}connect ftp.uu.net}
{\bf{}chdir /published/oreilly/nutshell/maketexwork} 
{\bf{}dir} 
{\bf{}get README} 
{\bf{}quit}
\end{ttindent}

A signature at the end of the message is acceptable as long as it appears
after \texttt{quit}.

\subsubsection{BITFTP}

\ixx{BITFTP}{BITFTP (getting examples via)} is a mail server for BITNET users. You
send it electronic mail messages requesting files, and it sends you back the
files by electronic mail. BITFTP currently serves only users who send it mail
from nodes that are directly on BITNET, EARN, or NetNorth. BITFTP is a public
service of Princeton University.

To use BITFTP, send mail containing your FTP commands to {\it
BITFTP@PUCC}. For a complete help file, send \texttt{HELP} as the message body.

The following is the message body you should send to BITFTP:

\begin{ttindent} 
FTP ftp.uu.net 
NETDATA USER anonymous 
PASS{\it your Internet e-mail address}      {\rm\it\small (not your BITNET address)} 
CD /published/oreilly/nutshell/maketexwork 
DIR 
GET README 
QUIT
\end{ttindent}

Questions about BITFTP can be directed to {\it MAINT@PUCC} on BITNET.

\subsubsection{UUCP}

\ixx{UUCP}{UUCP (getting examples via)} is standard on virtually all \Unix\ systems
and is available for IBM-compatible PCs and Apple Macintoshes. The examples
are available by UUCP via modem from UUNET; UUNET's connect-time charges
apply. You can get the examples from UUNET whether you have an account or
not. If you or your company has an account with UUNET, you will have a system
with a direct UUCP connection to UUNET. Find that system, and type (as one
line):

\begin{ttindent}
\$ \textbf{uucp uunet\ttbackslash!~/published/oreilly/nutshell/maketexwork/README \ttbackslash} 
\textit{\ \ \ \ yourhost}\ttbackslash!~/\textit{yourname}/
\end{ttindent}

The README file should appear some time later (up to a day or more) in the
directory {\it /usr/spool/uucppublic/yourname}. If you don't have an account,
but would like one so that you can get electronic mail, contact UUNET at
703-204-8000.

If you don't have a UUNET account, you can set up a UUCP connection to UUNET
in the United States using the phone number 1-900-468-7727. As of this
writing, the cost is 50 cents per minute. The charges will appear on your next
telephone bill. The login name is \textit{uucp} with no password. Your entry may vary
depending on your UUCP configuration.

\subsubsection{Gopher}

If you are on the Internet, you can use the
\ixx{\program{gopher}}{gopher (getting examples via)} facility to learn about
online access to examples through the O'Reilly Online Information Resource.
Access {\it gopher.ora.com} as appropriate from your site.

\section{Versions of \protect\TeX}

The most recent versions of \ixx{\TeX}{tex@\TeX!versions} and
\ixx{\MF}{metafont@\MF!versions} are version 3.1415 and version 2.71,
respectively.  Version 3 of \TeX\ introduced several new features designed to
improve support for non-English languages (including the use of 8-bit input
and some refinements to hyphenation control).  If you use an older version of
\TeX, you should upgrade.

\ixx{Donald Knuth}{Knuth, Donald} has specified that \TeX's version number
converges to $\pi$, therefore version 3.1415 is only the fourth minor revision
after version 3.  The next minor revision will be version 3.14159.  Similarly,
\MF's version number converges to $e$ (2.7182818284$\ldots$).

\section{Implementations and Platforms}
\label{sec:implementation}

The interface that \TeX\ presents to the writer is very consistent.  Most of
the examples described in this book are applicable to every single
\ixx{implementation}{tex@\TeX!implementations}\index{tex@\TeX!platforms}
of \TeX.  However, \TeX\ is not a closed system.  It is possible to step
outside of \TeX\ to incorporate special elements into your document or take
advantage of the special features of a particular environment.  These
extensions can dramatically restrict the portability of your documents.

Many of the topics covered in this book offer alternatives in those areas that
are less portable.  Therefore, it is natural to ask what implementations
are really covered.

Before outlining which implementations are covered, let me suggest that this
book will be useful even if you are using an implementation not ``officially''
covered here.  The reality of the situation is this: many, many tools have
been ported with \TeX.  Many of the tools mentioned in this book are
available on platforms that are not specifically discussed.  
Time and equipment constraints prevented 
\ixx{Amiga}{amiga tex implementations@Amiga \TeX\ implementations}, 
\ixx{Atari}{atari tex implementations@Atari \TeX\ implementations}, 
\ixx{NeXT}{next tex implementations@NeXT \TeX\ implementations}, 
\ixx{VMS}{vms tex implementations@VMS \TeX\ implementations}, and
\ixx{Windows NT}{windows nt tex implementations@Windows NT \TeX} 
implementations of \TeX\ from being specifically addressed in this 
edition of the book.

\subsection{UNIX}

\ixx{\Unix}{unix tex implementations@\Unix\ \TeX\ implementations} is probably
the most common \TeX\ platform.  The emphasis in this book is on \Unix\
workstations running X11, producing output for PostScript and HP LaserJet
printers.

\ixx{\program{Linux}}{linux tex implementations@Linux \TeX\ implementations} and other personal computer implementations of \Unix\ are
not addressed specifically; however, with the successful port of X11 to
\program{Linux}, I'm confident that every \Unix\ tool here can be, or has
been, ported to \program{Linux} (and probably other PC \Unix\ environments).

The only implementation of the \TeX\ program for \Unix\ considered in any
detail is the free implementation distributed in {\it web2c}.  This
distribution is described in the section called ``\nameref{free:web}'' in
Chapter~\ref{chap:freetex}, {\it\nameref{chap:freetex}}.  Most of the other
\Unix\ tools discussed here are also free.

\subsection{MS-DOS}

With very few exceptions, the tools in this book are available under
\ixx{MS-DOS}{MS-DOS}.
Because PCs are very popular, a lot of effort has gone into porting \Unix\
tools to MS-DOS.  Some packages, however, require a 386SX (or more powerful)
processor.  For the most part, I focus on PCs running MS-DOS only; however,
\ixx{Microsoft Windows}{microsoft windows@Microsoft Windows!tex implementations@\TeX\ implementations} 
and \ixx{DesqView}{desqview@DesqView!tex implementations@\TeX\ implementations}
are not entirely ignored.

There are quite a few options when it comes to selecting an implementation of
the \TeX\ program under MS-DOS.  Several free implementations are discussed
%(\ixx{\emTeX}{emTeX@\emTeX!tex implementations@\TeX\ implementations},
%\ixx{\sbTeX}{sbTeX@\sbTeX}, \ixx{\gTeX}{gTeX@g\TeX}) 
as well as some commercial implementations.
%, like \ixx{\utex}{microtex@\utex}
%\ixx{\TurboTeX}{TurboTeX@\TurboTeX}, and \ixx{\PCTeX}{PCTeX@\PCTeX}.  
For more
information about these implementations, consult Chapter~\ref{chap:freetex},
\textit{\nameref{chap:freetex}},
and Chapter~\ref{chap:commercialtex}, {\it\nameref{chap:commercialtex}}.

\subsection{OS/2}

In this book, OS/2 is treated primarily as a superset of MS-DOS.  When
possible, I look at OS/2-specific versions of each utility, but rely on MS-DOS
as a fall-back.

Extensions to em\TeX\ for 
\ixx{OS/2}{OS/2!tex implementations@\TeX\ implementations} 
are explored, as are editing environments such as \program{\idx{epm}}.
The multi-threaded nature of OS/2 allows more complete porting of \Unix\
tools.  When better ports are available for OS/2, they are discussed.

\subsection{Macintosh}

The \ixx{Macintosh}{Macintosh!TeX implementations@\TeX\ implementations} is very different from the systems described above.
Chapter~\ref{chap:mac}, {\it \nameref{chap:mac}}, discusses the Macintosh
environment in detail.

There are four implementations of \TeX\ for the Macintosh. Three are freely
available, and one is commercial: \ixx{\cmactex}{cmactex@\cmactex} is free,
\ixx{\oztex}{oztex@\oztex} and \ixx{\directex}{directex@\directex} are 
shareware, and \ixx{\program{Textures}}{Textures} is a commercial package 
from \idx{Blue Sky Research}.

%\subsection{Windows NT}
%
%A \program{web2c} port of \TeX\ software for \ixx{Windows NT}{windows nt@Windows NT!tex implementations@\TeX\ implementations} \index{NT}} is
%now available.  For more information about \program{web2c}, consult
%Chapter~\ref{chap:freetex}.  The Windows NT port is not addressed
%specifically.

\section{We'd Like to Hear From You}

We have tested and verified all of the information in this book to the best of
our ability, but you may find that features have changed (or even that we have
made mistakes!).  Please let us know about any errors you find, as well as
your suggestions for future editions, by writing:

\begin{flushleft}
   O'Reilly \& Associates, Inc. \\ 103 Morris Street, Suite A \\ Sebastopol,
   CA 95472 \\ 1-800-998-9938 (in the US or Canada) \\ 1-707-829-0515
   (international/local) \\ 1-707-829-0104 (FAX) \\
\end{flushleft}

You can also send us messages electronically.  To be put on the mailing list
or request a catalog, send email to:

\begin{xtable}{ll}
  {\it nuts@ora.com} & (via the Internet) \\ 
  {\it uunet!ora!nuts} & (via UUCP) \\
\end{xtable}

To ask technical questions or comment on the book, send email to:

\begin{xtable}{l}
  {\it bookquestions@ora.com} (via the Internet)
\end{xtable}

\section{Acknowledgments}

This book would not exist if I had not received support and encouragement from
my friends and colleagues, near and far.  I owe the deepest debt of gratitude
to my wife, Deborah, for patience, understanding, and support as I progressed
through what is easily the most all-consuming task I have ever undertaken.

The earliest draft of this book came about because my advisor at
the University of Massachusetts, Eliot Moss, allowed me to tinker with the
\TeX\ installation in the Object Systems Lab and was always able to suggest
ways to make it better.  My friends and colleages at UMass, Amer Diwan, Darko
Stefanovi\'c, Dave Yates, Eric Brown, Erich Nahum, Jody Daniels, Joe McCarthy,
Ken Magnani, Rick Hudson, and Tony Hosking, asked all the hard questions and
didn't seem to mind when I used them as guinea pigs for my latest idea.

I'm indebted also to 
Eberhard Mattes, 
Geoffrey Tobin,
George D. Greenwade, 
Peter Schmitt, 
Sebastian Rahtz, and
Tomas Rokicki, 
who provided
technical review comments on the materials presented here.  
Jim Breen and Ken Lunde
offered invaluable feedback on Chapter 7.

And I'd like to thank a lot of people at O'Reilly for their help and
enthusiasm; in particular, my editor, Debby Russell, offered advice, helpful
criticism, and support beyond the call of duty (Debby keyed most of the index
for this book as production deadlines drew near and other arrangements fell
through); Chris Tong organized the raw entries into a usable index; Lenny
Muellner, Donna Woonteiler, and Sheryl Avruch allowed me to work on the book
when it wasn't technically my job; Stephen Spainhour copyedited it into
English with the help of Leslie Chalmers and Kismet McDonough (Stephen offered
helpful suggestions along the way, too); Jennifer Niederst helped me get the
design right; and Chris Reilly created the figures and screen dumps.  I
enjoyed working with everyone at O'Reilly so much that I left UMass and joined
the production department myself ;-).

Several companies provided review copies of their software while I was writing
this book.  I would like to thank ArborText, Blue Sky Research, Borland
International, The Kinch Computer Company, LaserGo, Personal \TeX, TCI
Software Research, and \YY, for their generosity.

Finally, I'd like to thank the entire Internet \TeX\ community.  Countless
thousands of questions and answers on the Net refined my understanding of how
\TeX\ works and what it can do.

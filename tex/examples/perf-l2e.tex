% Format: LaTeX2e
\documentclass{report}

\setlength{\parskip}{\baselineskip}
\setlength{\parindent}{0pt}

\begin{document}

\chapter{Unsolved Problems}

\section{Odd Perfect Numbers}

A number is said to be \emph{perfect} if it
is the sum of its divisors.  For example, $6$ is
perfect because \(1+2+3 = 6\), and $1$, $2$, and $3$
are the only numbers that divide evenly into $6$ 
(apart from 6 itself).

It has been shown that all even perfect numbers
have the form \[2^{p-1}(2^{p}-1)\] where $p$
and \(2^{p}-1\) are both prime.

The existence of \emph{odd} perfect numbers is 
an open question.
\end{document}

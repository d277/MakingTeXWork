% Format: Lollipop
\DefineHeading:Chapter
   breakbefore:yes whiteafter:12pt
   line:start PointSize:20 Style:bold
      literal:Chapter Spaces:1  ChapterCounter
      line:stop
   vwhite:36pt
   line:start PointSize:24 Style:bold title
      line:stop
   vwhite:24pt
   Stop
\DefineHeading:Section
   whitebefore:{20pt plus 2pt} whiteafter:14pt
   line:start PointSize:14 Style:bold
      ChapterCounter . SectionCounter
      Spaces:1 title line:stop
   label:start ChapterCounter . SectionCounter
      label:stop
   Stop
\GoverningCounter:Section=Chapter
\AlwaysIndent:no
\Distance:parskip=12pt
\Distance:hoffset=.75in
\Distance:voffset=.5in
\Start
\Chapter Unsolved Problems

\Section Odd Perfect Numbers

A number is said to be {\it perfect\/} if it
is the sum of its divisors.  For example, $6$ is
perfect because $1+2+3 = 6$, and $1$, $2$, and $3$
are the only numbers that divide evenly into $6$ 
(apart from $6$ itself).

It has been shown that all even perfect numbers
have the form $$2^{p-1}(2^{p}-1)$$ where $p$
and $2^{p}-1$ are both prime.

The existence of {\it odd\/} perfect numbers is 
an open question.
\Stop

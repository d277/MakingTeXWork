\chapter[International Considerations]{International\\Considerations} 
\RCSID$Id: ch07.tex,v 1.1 2002/08/23 14:58:46 nwalsh Exp $
\label{chap:foreign}

\bgroup % <<<<<<<<<<<<<<<<<<<<<<<<<<<<<<<<<<<<<<<<<<<<<<<<<<<<<<<<<<<<<<<<
%

\font\jfcG=wjisc6
\font\jfcb=wjiscb
\font\jfbI=wjisb8
\def\pmj{{\tt pmJ}{\jfcG\char252\jfcb\char220\jfbI\char236}}

\font\dcr=dcr10
\def\guillemet{{\dcr\char"13\relax}}

Although the standard \TeX\ macro packages and the Computer Modern fonts were
designed to typeset documents written primarily in English, \TeX\ enjoys
widespread international 
use\index{international!typesetting}\index{typesetting!multilingual documents}\index{international!considerations for tex@considerations for \TeX}.  

From a technical standpoint, languages can be divided into two categories:
those that are ``like English'' (meaning that they use a relatively small
number of characters and are typeset horizontally, left to right) and those
that are not.  German, French, and Russian are all ``like English'' in this
sense.  Hebrew, Chinese, and 
Japanese are not (Hebrew\index{Hebrew} is 
typeset right to left,
and Chinese\index{Chinese} and 
Japanese\index{Japanese} use 
thousands of characters).\footnote{Chinese and
Japanese are also typeset vertically.  At present, \TeX\ does not support
vertical typesetting, although there is at least one effort underway to provide
that feature.  See the section called ``\nameref{sec:asciijtex}'' for
more information on vertical typesetting.}

This chapter explores some of the issues that arise when \TeX\ is used to
typeset languages other than English.  For simplicity, we'll look at languages
like English first, and then describe some 
environments for typesetting much more complex languages.

\section{Typesetting in Any Language}

In order to typeset\index{typesetting!steps} any 
language with \TeX, three things have to happen:

\begin{enumerate}
\item \TeX\ has to read the input file and perform the correct mapping from
the input file's character set to its internal representation of each
character.  The character
set used in the input file will vary depending upon the language. For instance,
if you're
writing a document in French\index{French}, it 
is as natural to use ``\'e'' in your
input file as it is to use any other letter.

\item \TeX\ has to typeset the document according to the rules of the
language being used.  Naturally, this means that there must be some
way of declaring what language is being used, and appropriate macros
have to exist to embody the rules of that language.  Users familiar only
with English may not recognize the importance of language-specific
rules because English has so few rules.  Other languages have many.
In German\index{German}, for 
example, if the consonants ``ck'' in a word are broken
by a hyphen, the ``c'' becomes a ``k'' (``k-k'').  
In French\index{French}, small
amounts of extra
space are placed around various punctuation marks.

A good reference manual for internationalization is
\textit{Software Internationalization and Localization: An 
Introduction}~\cite{eu:international}.

\item The \ext{DVI} file that results from typesetting the document must
be printed correctly.  In other words, all of the accented characters and
symbols used by the language must be available (or constructed) for
previewing and printing.
\end{enumerate}

Early attempts to write documents in languages other than English were
hampered by several limitations in the \TeX\ program.  In particular, fonts
were limited to 128 characters, and only a single set of hyphenation patterns
could be loaded (effectively preventing multilingual documents from being
hyphenated correctly).  These technical problems were corrected in \TeX\
version 3.x (first released in 1990).  The remaining difficulties---mostly
a lack of standardization and the need to develop relevant language-specific
macros---are being addressed by the TUG\index{TeX User's Group (TUG)@\TeX\ Users Group (TUG)} \idx{Technical Working Group on Multiple Language Coordination (TWGMLC)}.

\subsection{Reading Input Files}

The first point to consider when typesetting is that every input file
is written in some character set.  For example, because this book is
written in English and I work in the United States, the source code
for this book is written in 7-bit ASCII\index{ASCII character set}.  If 
this book were written in
another language, a different character set, perhaps ISO Latin1, would
be more appropriate.

When \TeX\ reads your input file, characters like ``\'e'' and
``\guillemet'' have to be translated into a form that \TeX\ can use.
For example, ``\'e'' should be translated into \verb|\'e| and, if the
DC fonts are in use, ``\guillemet'' should be translated into
character 19; otherwise, if the \idx{DC fonts} are not in use,
``\guillemet'' should be translated into \verb|$<<$| which will give
the approximate result.  The DC fonts are discussed in the section
called ``\nameref{sec:for:printing},'' later in this chapter.

It is always possible to access characters from another symbol set by
using a control sequence\index{control sequences!using to access other languages}.  Table~\ref{tab:accentedchar} shows the 
standard \TeX\ control sequences for accessing accented characters and
characters from other alphabets.\footnote{The accent macros are
shown with a lower case e; naturally, any letter that needs to
be accented can be used in place of the e.}

\begin{xtable}{l|l||l|l}
  \caption{Standard Control Sequences
     for Symbols from Other Character Sets\label{tab:accentedchar}}\\
  \bf Control       &             & \bf Control       & \\
  \bf Sequence      & \bf Symbol  & \bf Sequence      & \bf Symbol \\[2pt]
  \hline
  \tstrut
  \verb|\`e|    &  \`e    & \verb|\oe| &  \oe  \\
  \verb|\'e|    &  \'e    & \verb|\ae| &  \ae  \\
  \verb|\"e|    &  \"e    & \verb|\o|  &  \o   \\
  \verb|\u{e}|  &  \u{e}  & \verb|\aa| &  \aa  \\
  \verb|\.e|    &  \.e    & \verb|\l|  &  \l   \\
  \verb|\t{ee}| &  \t{ee} & \verb|\OE| &  \OE  \\
  \verb|\d{e}|  &  \d{e}  & \verb|\AE| &  \AE  \\
  \verb|\H{e}|  &  {\fontfamily{cmr}\selectfont\H{e}}  & \verb|\O|  &  \O   \\
  \verb|\^e|    &  \^e    & \verb|\AA| &  \char'302  \\
  \verb|\v{e}|  &  \v{e}  & \verb|\L|  &  \L   \\
  \verb|\=e|    &  \=e    & \verb|?`|  &  ?`   \\
  \verb|\~e|    &  \~e    & \verb|!`|  &  !`   \\
  \verb|\c{e}|  &  \c{e}  & \verb|\ss| &  \ss  \\
  \verb|\b{e}|  &  \b{e}  &            &       \\[2pt]
  \hline
\end{xtable}                     

Table~\ref{tab:newaccentedchar} shows
new control sequences proposed by the TWGMLC for characters not available
in standard \TeX\ and \LaTeX\ distributions.\footnote{These characters are 
available in the DC fonts and were not previously available in standard \TeX.}

\begin{xtable}{ll|c|l}
  \caption{New Control Sequences Proposed by 
    TWGMLC\label{tab:newaccentedchar}}\\
  \multicolumn{2}{l|}{\bf Control Sequences} & \bf Symbols & \bf Description\\[2pt]
  \hline
  \tstrut
  \verb|,,| & \verb|``|   & \dcr\char18\ \ \ \char16 & 
    German quotations ({\dcr\char18}G\"ansef\"u\ss{}chen{\dcr\char16})\\
  \verb|<<| & \verb|>>|   & \dcr\char19\ \ \ \char20 &
    French quotations (guillemets) \\
  \verb|\dh| & \verb|\DH| & \dcr\char240\ \ \ \char208 &
    Icelandic eth \\
  \verb|\dj| & \verb|\DJ| & \dcr\char158\ \ \ \char208 &
    Serbocroation dj \\
  \verb|\ng| & \verb|\NG| & \dcr\char173\ \ \ \char141 &
    Sami ng \\
  \verb|\th| & \verb|\TH| & \dcr\char254\ \ \ \char222 &
    Icelandic thorn \\
  \verb|\k e| &   & \dcr\char166 & Polish ogonek subscript \\
  \verb|\r u| &   & \dcr\char183 & Czech circle accent \\
  \verb|\v{d}| & \verb|\V{D}| & \dcr\char164\ \ \ \char132 &
    Czech d and D with ha\v{c}ek \\
  \verb|\v{l}| & \verb|\V{l}| & \dcr\char169\ \ \ \char137 &
    Slovakian l and L with ha\v{c}ek \\
  \verb|\v{t}| & \verb|\V{T}| & \dcr\char180\ \ \ \char148 &
    Czech t and T with ha\v{c}ek \\[2pt]
  \hline
\end{xtable}

The only technical problem associated with using language-specific
character sets in your input files is that you must have some way of telling
\TeX\ to perform the appropriate substitutions.  One method is to use a
special style file like
\filename{isolatin1}.\footnote{\filename{isolatin1.sty} is available from
the CTAN archives in \ctandir{macros/latex/contrib/misc/}.}
This style uses 
``active characters'' to map the ISO Latin1 input
character set to \TeX's representation.  It could be adapted to other
character sets as well.  Another possibility is to rely on system-dependent
extensions to \TeX.  For example, \emTeX\ provides extensive support
for ``code pages,'' which address this problem.

The only other problem created by using different input character sets is one
of compatibility.  If you write files using the ISO Latin1 character set and
send
them to someone who uses a different character set, the 
file will appear to be incorrect.\footnote{If you use electronic mail 
to send files that use any 
characters other than the printable subset of 7-bit ASCII (space through
tilde),
you are bound to run into problems.  You can combat
this problem by using a wrapper (like uuencoding or MIME messages) when you
send the mail, but those tools are outside the scope of this book.  Ask your
system administrator for more assistance with sending binary mail.}  

\subsubsection{Appearances can be deceiving}

A document stored on disk is really just a file containing a series of
characters, each represented by a unique numerical value.  For
an editor to display a document, each numerical value must
be translated into a visual representation of the character.  Frequently
this translation is performed by the operating system or computer hardware.
In an analogous way, each numeric value must be translated into a printable
character when the document is typeset.

Figure~\ref{fig:charmap} shows how this translation is performed for
display by the operating system and for printing by \TeX\ (using the
\filename{isolatin1} style, for example).  This figure shows the disparity
that occurs if the two translation tables are not the same.

\epsimage{fig:charmap}{Character mapping example}

How can this arise?  Well, suppose, for example, that a colleague is
writing a document in French\index{French}.  He 
has a \TeX{}nical problem that he
would like me to investigate.  I agree to take a look, and he sends the
file to me.  My colleage is using the ISO Latin1 character set in his
input file because it contains many symbols that are convenient for
writing French (including the guillemets).  I receive the file and
edit it on my PC.  The file that I see displayed looks like gibberish.
That's because I'm using the IBM OEM encoding on my PC, which
is sufficient for English.  All of the special characters in the
ISO Latin1 character set appear incorrect.  Bewildered, I \TeX\ and
preview the document to see what it's supposed to look like.  To my
surprise, the previewed document looks fine.

In this case, I can correct the problem by changing the ``code page'' used
on my PC or by translating the input file with a program like 
\ixx{GNU \program{recode}}{GNU recode}.\footnote{\program{recode} is available from
\texttt{prep.ai.mit.edu} and other places where GNU software is archived.}

\subsection{Changing the Rules}

In order to select languages, the TWGMLC has proposed a set of
language switching macros.  These are shown in
Table~\ref{tab:langswitch}.\footnote{These are ``low-level'' macros.
A higher-level interface will be provided for each language.  See the
section called ``\nameref{sec:babel}'' later in this chapter for more
information.}

%\input{fg-lang.tab}

Selecting a language has three effects:

\begin{itemize}
  \item It establishes the correct hyphenation 
  environment\index{hyphenation!language-specific}.

        Language-specific hyphenation patterns are loaded, if necessary,
        and correct values for the minimum length of a hyphenated
        word fragment are set.

        At the time of this writing, hyphenation patterns are already
        available, or under development, for Armenian, Bulgarian,
        Cambodian, Catalan, Croation, Czech, Danish, Dutch, English
        (U.K. and U.S.), Esperanto, Estonian, Finnish, French, German,
        Greek (both modern and ancient), Hungarian, Icelandic,
        Italian, Kirundi, Latin, Lithuanian, Norwegian, Polish,
        Portuguese, Russian, Slovak, Swahili, Swedish, Yiddish, and
        Yoruba.

  \item It loads the correct fonts and special characters.

        Even languages which use the same alphabet may have different
        fonts\index{fonts!language-specific}\index{special characters!language-specific}\index{characters!special!language-specific}
        in order to provide specific features of the language.
        For example, the ``fi'' \ixx{ligature}{ligatures} makes 
        sense only when typesetting
        English, and the \cs{th} and \cs{TH} macros 
        make sense only when typesetting languages that need 
        ``{\dcr\char254}'' and ``{\dcr\char222}.''

       
\vfill
\newpage
{
\tabcolsep=3pt\fontsize{9}{10pt}\selectfont\def\b{\ttbackslash\tt\small}
\begin{xtable}{c|l||c|l||c|l||c|l}
  \caption{Language Switch Macros Proposed by TUG\label{tab:langswitch}}\\
 \bf\tiny Macro &      & \bf\tiny Macro &         & \bf\tiny Macro &          & \bf\tiny Macro & \\
 \bf\tiny Switch & \bf Language & \bf\tiny Switch & \bf Language & \bf\tiny Switch & \bf Language & \bf\tiny Switch & \bf Language \\[2pt]
\hline
\tstrut
  \b AB & Abkhazian    & \b FY & Frisian     & \b MG & Malagasy     & \b SB & Sorbian    \\
  \b OM & Afan Oromo   & \b GL & Galician    & \b ML & Malayalam    & \b ES & Spanish    \\
  \b FF & Afar         & \b KA & Georgian    & \b MS & Malay        & \b SU & Sudanese   \\
  \b AF & Afrikaans    & \b DE & German      & \b MT & Maltese      & \b SW & Swahili    \\
  \b SQ & Albanian     & \b EL & Greek       & \b MI & Maori        & \b SV & Swedish    \\
  \b AM & Amharic      & \b KL & Greenlandic & \b MR & Marathi      & \b TL & Tagalog    \\
  \b AR & Arabic       & \b GN & Guarani     & \b MO & Moldavian    & \b TG & Tajik      \\
  \b HY & Armenian     & \b GU & Gujarati    & \b NA & Nauru        & \b TA & Tamil      \\
  \b AS & Assamese     & \b HA & Hausa       & \b NE & Nepali       & \b TT & Tatar      \\
  \b AY & Aymara       & \b HE & Hebrew      & \b NO & Norwegian    & \b TE & Telugu     \\
  \b AZ & Azerbaijani  & \b HI & Hindi       & \b OC & Occitan      & \b TY & Thai       \\
  \b BA & Bashkir      & \b HU & Hungarian   & \b OR & Oriya        & \b BO & Tibetan    \\
  \b EU & Basque       & \b IS & Icelandic   & \b PS & Pashto       & \b TI & Tigrinya   \\
  \b BN & Bengali      & \b ID & Indonesian  & \b FA & Persian      & \b TO & Tonda      \\
  \b DZ & Bhutani      & \b IA & Interlingua & \b PL & Polish       & \b TS & Tsonga     \\
  \b BH & Bihari       & \b IE & Interlingue & \b PT & Portuguese   & \b TR & Turkish    \\
  \b BI & Bislama      & \b IU & Inuktitut   & \b PA & Punjabi      & \b TK & Turkmen    \\
  \b BR & Breton       & \b IK & Inupiak     & \b QU & Quechua      & \b TW & Twi        \\
  \b BG & Bulgarian    & \b GA & Irish       & \b RM & Rhaeto-Roman & \b GB & U.K. English \\
  \b MY & Burmese      & \b IT & Italian     & \b RO & Romanian     & \b US & U.S. English \\
  \b BE & Byelorussian & \b JA & Japanese    & \b RU & Russian      & \b UG & Uigur      \\
  \b KM & Cambodian    & \b JW & Javanese    & \b SE & Sami         & \b UK & Ukrainian  \\
  \b CA & Catalan      & \b KN & Kannada     & \b SM & Samoan       & \b UR & Urdu       \\
  \b ZH & Chinese      & \b KS & Kashmiri    & \b SG & Sangho       & \b UZ & Uzbek      \\
  \b CO & Corsican     & \b KK & Kazakh      & \b GD & Scots Gaelic & \b VI & Vietnamese \\
  \b HR & Croatian     & \b RW & Kinyarwanda & \b SR & Serbian      & \b VO & Volapuk    \\
  \b CS & Czech        & \b KY & Kirghiz     & \b ST & Sesotho      & \b CY & Welsh      \\
  \b DA & Danish       & \b RN & Kirundi     & \b TN & Setswana     & \b WO & Wolof      \\
  \b NL & Dutch        & \b KO & Korean      & \b SN & Shona        & \b XH & Xhosa      \\
  \b EO & Esperanto    & \b KU & Kurdish     & \b SD & Sindhi       & \b YI & Yiddish    \\
  \b ET & Estonian     & \b LO & Laothian    & \b SI & Singhalese   & \b YO & Yoruba     \\
  \b FO & Faroese      & \b LV & Latvian     & \b SS & Siswati      & \b ZA & Zhuang     \\
  \b FJ & Fiji         & \b LN & Lingala     & \b SK & Slovak       & \b ZU & Zulu       \\
  \b FI & Finnish      & \b LT & Lithuanian  & \b SL & Slovenian    &     &            \\
  \b FR & French       & \b MK & Macedonian  & \b SO & Somali       &     &            \\[2pt]
  \hline
\end{xtable}
}
\newpage

  \item It defines special primitive operations, if appropriate.  

        For example, right-to-left typesetting primitives are
        necessary only for languages like Hebrew\index{Hebrew} which are typeset right-to-left.
\end{itemize}

\subsection{Printing the Result}
\label{sec:for:printing}

The Computer Modern Fonts\index{Computer Modern fonts!language extensions}
are insufficient for typesetting languages
other than English.  In order to overcome this difficulty, the \TeX\
User's Group has extended the Computer Modern encoding vector and
established a new standard.

The new standard fonts are known variously as 
the \idx{Cork fonts}\index{fonts!Cork}, the
\idx{DC fonts}\index{fonts!DC}, and 
the \idx{EC fonts}\index{fonts!EC}.  These are all synonymous.  The new
standard was created following discussions at the \TeX\ User's Group
meeting in Cork, Ireland in 1990, hence the name Cork.  The
\ixx{\MF\ fonts}{metafont fonts@\MF\ fonts}\index{fonts!metafont@\MF}, which embody this 
encoding, will eventually become the EC
fonts.  The current versions, available now, are still being refined
(in the sense that some of the letter forms are being refined; the
encoding will not change).  These are called the DC
fonts.

There is a distinction between the standard encoding vector and
the \MF\ fonts that replace Computer Modern.  Therefore, I will
refer to the standard encoding as the ``\idx{Cork Encoding}'' and to the
\MF\ fonts as the ``DC fonts''.\footnote{Because the ``EC fonts'' don't
exist yet, I won't mention them again.}

At the time of this writing, the DC fonts are not a complete superset
of Computer Modern because the DC Math fonts have not yet been released.
(The \ixx{DC fonts}{DC fonts!Math fonts}\index{fonts!DC Math} will contain 
the upper-case Greek\index{Greek} alphabet, which
is currently missing from the DC fonts.)
When the DC Math fonts are released, the DC fonts will be a complete
superset of Computer Modern.  The only apparent difference will be
that the accents on the DC fonts are not at exactly the
same height as the accented characters constructed with the \cs{accent}
primitive using Computer Modern.\footnote{Actually, the issue of accents is 
a difficult one.  Different languages which have the same letters do not 
always place accents at the same height.  This is yet another problem that
will have to be resolved.}

The Cork Encoding is shown in Table~\ref{tab:font:corkencoding} in
Appendix~\ref{app:fonts}, {\it \nameref{app:fonts}}.  There are 255 symbols in
this vector with one blank for special purposes.  Unlike the Computer Modern
fonts, which have different encoding vectors in some typefaces (Computer Modern
Roman is not the same as Computer Modern Typewriter, for example), all of the
DC fonts have the same encoding vector.

% \input{fg-cork.tab}

Several people have commented that the Cork Encoding suffers from a
``design error'' because it places characters in positions 0-31, which
are frequently inaccessible in other applications, and because it places a
nonstandard character at position 32, where a space usually occurs.
This is not a design error.  Bear in mind that the DC fonts are
designed to be \TeX\ output fonts.  Font creators, working in other
environments (for example, TrueType or PostScript) are free to divide
the Cork Encoding into two separate font files and provide a virtual
font for \TeX\ that establishes the correct encoding.  The motivation
for putting as many symbols as possible in a single font is that \TeX\
cannot \ixx{kern}{kerning} across fonts.

This is not meant to imply that the DC fonts should always be 
virtual\index{fonts!virtual}\index{virtual fonts}\index{fonts!real}.
In fact, the DC fonts should be the ``real'' fonts upon which virtual
fonts are based.  
A virtual Computer Modern font based upon the real
DC fonts is infinitely preferable to a virtual DC font built on Computer Modern
because:

\begin{itemize}
  \item You need a ``real'' font to make virtual fonts, and the Computer
        Modern fonts don't contain enough {\em real\/} characters.
  \item Expressing accented characters in \MF\ is much better than
        building accents inside a virtual font.  The virtual font
        has less information to work with (it has only boxes).
  \item Different languages use accents at different heights.  A simple
        ``patch'' to the \MF\ code for a real font with accents is
        far superior to introducing another set of virtual fonts
        for every language.
\end{itemize}

From a purely practical point of view, the correct way to deal with 
these and related problems is to use the babel style files.  

\section{The Babel Styles}
\label{sec:babel}

The \idx{babel styles}\index{style files!babel} are a 
collection of style files for \LaTeX\ that
provide features for typesetting in many languages.
The babel styles are
compatible with Plain \TeX\ and all versions of \LaTeX. (In particular,
they are being adopted as the standard multilingual styles in \LaTeXe\
and will be the standard in \LaTeX3 when it is released.)

To date, babel styles exist for 
Catalan,        Croatian,       Cyrillic,       Czech,
Danish,         Dutch,          English,        Esperanto,
Finnish,        French,         Galician,       German,
Italian,        Hungarian,      Norwegian,      Polish,
Portuguese,      Romanian,       Russian,        Slovak,
Slovenian,      Spanish,        Swedish,  and   Turkish,
as well as several dialects (American as a dialect of English,
for example).

Example~\ref{ex:usinglangs} shows the skeletal
structure of a document using the English and French styles.  Within the
document, the \cs{selectlanguage} control sequence is used
to switch
between languages.\footnote{Another control sequence,
\texttt{\ttbackslash iflanguage}, is provided so that you can write macros
which are sensitive to the language in use when they are expanded.}  
The language that is in effect by default
is determined when the format file is created.

\newpage
\begin{example}{ex:usinglangs}{A Sample Multilingual Document 
    Using English and French}
\documentstyle[english,francais]{article}
\begin{document}
This is a document which uses both English and 
French. \selectlanguage{french} Mais, je ne parle plus 
fran\c{c}ais. \selectlanguage{english}  So I won't try 
to make this example very long.
\end{document}
\end{example}

Selecting a language automatically has the following effects:

\begin{itemize}

\item It selects \ixx{hyphenation patterns}{hyphenation!patterns} for 
      the language.   This means
      that paragraphs of text will be hyphenated according to the
      conventions of the language in use.\footnote{Paragraphs that
      contain multiple languages will be hyphenated according to the
      rules of the language in effect when the paragraph ends.}
      Switching hyphenation patterns is possible only if the format
      file being used by \TeX\ contains hyphenation rules for
      language\index{hyphenation!language-specific}.

\item It automatically translates the names of all the document elements
      into the selected language.  For example, if you insert
      the \cs{tableofcontents} when French\index{French}  is the selected
      language, the table of contents will be called the
      ``Table des mati\`eres'' instead of ``Table of Contents.''

\item It alters the format of the date produced by the \cs{today} macro
      to fit the conventions of the selected language.
      In American, \cs{today} is ``January 30, 1994''; in 
      English it is ``30th January 1994''; and in French it is
      ``30 janvier 1994.''\index{dates (language-specific)}

\item It defines particular typing shortcuts to make writing the selected
      language more convenient for the typist.  For example, 
      the French\index{French}
      style makes several punctuation characters into macros so that extra
      space is automatically inserted before them according to French
      typographic conventions\index{typing shortcuts (language-specific)}.
\end{itemize}

\subsection{Building Multilingual Babel Formats}
\label{sec:babelfmt}

Building a \ixx{multilingual format file}{multilingual format files} is 
very much like building a
format\index{format files!multilingual} for 
a single language.  The only difference is that instead of loading 
a single set of hyphenation patterns (generally from a file called
\filename{hyphen.tex}), you will need hyphenation patterns for each language 
that you want
to use.  These can be obtained from the CTAN archives in the directory
\ctandir{tex-archive/language/hyphenation}.

If you have a file called \filename{hyphen.tex} on your system, rename it.
This file is distributed as part of the standard \TeX\ distribution and
contains American English hyphenation patterns, so \filename{ushyphen.tex}
is an appropriate name.

Next, create a file called \filename{language.dat} that contains one line
for each language you want to use.  Each line should list the language
name and the file containing hyphenation patterns for that language.
For example, an appropriate \filename{language.dat} file for the format 
used to typeset Example~\ref{ex:usinglangs} might contain these lines:

\begin{ttindent}
english ehyphen.tex
francais fr8hyph.tex
\end{ttindent}

Now, proceed to construct the format file using \ixx{ini\TeX}{initex@ini\TeX} according to the 
instructions distributed with the format or by following the suggestions
in Chapter~\ref{chap:macpack}, \textit{\nameref{chap:macpack}}.  When
ini\TeX\ complains that it cannot find \filename{hyphen.tex}, provide the
alternate name \filename{babel.hyphen}.  This will use \filename{language.dat} to
load the appropriate hyphenation patterns and associate them with the
languages you specified.

\begin{note}{NOTE}
The first language that you list in \filename{language.dat} will be the
default language for the format file that you create.
\end{note}

\section{\protect\TeX\ Pitfalls}

\TeX\ 3.x, the \idx{Cork Encoding}, and language-specific macro files are not 
``magic bullets'' that can solve all of the problems that arise in typesetting
multilingual documents or writing macros that are useful in all language
contexts.
Some of the deficiencies are really insoluble without changing the \TeX\
program in ways that are not allowed by Knuth.  Two such problems are
mentioned here:

\begin{itemize}
  \item The \cs{uppercase} and \cs{lowercase} primitives are 
        problematic.  

        There is a strict one-to-one mapping between
        lowercase and uppercase letters in \TeX.  Unfortunately, accented
        letters may require different mappings.  Consider these examples:

        \begin{tabular}{lll}
          \verb|\"I with two dots| & \"I with two dots & What you typed \\
          \verb|\"i with two dots| & \"i with two dots & Result of \cs{lowercase} \\
          \verb|\"\i\ with two dots| & \"\i\ with two dots & Correct lowercase
        \end{tabular}

        Because no information about the accent is known, the result
        of passing your text to the \cs{lowercase} primitive is not
        correct.

        This problem can be minimized by using an input character set
        which contains the accented letters that you need.  This allows
        you to establish the appropriate one-to-one relationships.

        Some of the characters chosen for the Cork encoding were 
        driven by this weakness as well.  The only reason that
        ``{\dcr\char"DF}'' is a glyph is so that it can be the
        \cs{uppercase} character for ``{\dcr\char"FF}.''

  \item \TeX\ doesn't distinguish between a dash used in a compound word
        (for example, ``wish-fulfillment'') and a dash used for hyphenation.

        This distinction isn't necessary in English because English doesn't
        have any end-of-word \idx{ligatures}.  Imagine a language
        where ``sh'' should become ``x'' at the end of a 
        word.\footnote{This is a concocted example; to my knowledge, 
        it doesn't actually appear 
        in any language.}  A compound word like
        ``push-ready'' should be typeset ``pux-ready'' whereas a
        word like ``pushover'' should remain ``push-over'' if it is
        hyphenated across a line break.
\end{itemize}

\section{Very Complex Languages}

The following sections describe \TeX\ packages (collections of macros,
fonts, and other files) that allow you to typeset languages very
different from English.  

\subsection{Japanese}

Typesetting Japanese\index{Japanese} 
involves solving several problems.  The first is
the task of entering Japanese text with an editor.  There are many editors on
many platforms that can handle Japanese input.  Although there are also
established ways to romanize Japanese text so that it can be displayed on
terminals that do not provide support for Japanese input, these are bound to
be inconvenient for anyone seriously writing in Japanese.  If you are in a
position to edit Japanese text, you are probably already aware of several 
good editors.

The second problem is that typesetting Japanese with \TeX\ requires many,
many fonts.  The fact that a single font can hold no more than 256 symbols
means that dozens (perhaps hundreds) of fonts are required to represent
all of the myriad symbols used in everyday Japanese writing.  There are
some hard-coded limits on the number of fonts that a single \TeX\
document can use, and it is possible to bump into them pretty quickly
when typesetting a language like Japanese.

Another problem is printing the output.  Assembling a collection of fonts
that contain high-quality glyphs for all of the necessary characters is a
time consuming and potentially expensive task.  At present, the 
freely-available fonts are of relatively low quality.

A complete discussion of these issues, and many others can be found in
\textit{Understanding Japanese Information Processing}~\cite{kl:japinfoproc}.

\subsubsection{ASCII Nihongo \protect\TeX}
\label{sec:asciijtex}

ASCII Nihongo \TeX\index{ASCII  Nihongo tex (jtex)@ASCII  Nihongo \TeX\ (\jtex)}\index{Japanese!ASCII Nihongo tex (jtex)@ASCII Nihongo \TeX\ (\jtex)} (also 
known as \ixx{\jtex}{jtex@\jtex}) is a complete, Japanized version of
\TeX.  Instead of trying to shoehorn Japanese into traditional \TeX\ programs,
all of the programs have been modified to accept files containing standard
Japanese text (two bytes per character).  This section describes the 
\idx{ASCII Corporation}'s version of \jtex.  See the section called
``\nameref{sec:nttjtex}'' for information about NTT's version of \jtex.  You
can get the ASCII version of \jtex\ from \verb|ftp.ascii.co.jp| (133.152.1.1).

ASCII Nihongo \TeX\ is based on \TeX\ 2.9 and can read input files coded with
JIS, Shift-JIS, EUC, and KUTEN.  The \ext{DVI} files produced by ASCII \jtex\
are not standard \ext{DVI} files.  In order to support the large character 
set for Japanese writing, the \ext{DVI} files use commands that are not 
output by standard \TeX, so many drivers do not support them.
You cannot process \ext{DVI} files produced by ASCII \jtex\ with
most standard \dvidriver{}s.

Release notes with version 1.7 of \jtex\ indicate that it will be the last
public release of \jtex.  Another product, called p\TeX\ (for Publishing \TeX)
may be released at some time in the future.  One advantage of p\TeX\ will be
the ability to typeset vertically.

The primary disadvantage of the \jtex\ system is that there are no 
freely-available fonts for it.  The authors assume that you will be using fonts
resident in your printer.  You may be able to purchase Japanese fonts from
some font vendors, although I've seen no detailed instructions for using them
with \jtex.  Other, albeit more minor, disadvantages are the need to build and
maintain an entire parallel \TeX\ distribution and the fact that standard
\ext{DVI} drivers cannot process \jtex\ \ext{DVI} files.

\subsubsection{NTT \jtex}
\label{sec:nttjtex}

\jtex\index{NTT jtex@NTT \jtex}\index{Japanese!NTT jtex@NTT \jtex} is a 
complete, Japanized version of \TeX.  Instead of trying to
shoehorn Japanese into traditional \TeX\ programs, all of the programs have
been modified to accept files containing standard Japanese text (two bytes per
character).  This section describes NTT's version of 
\jtex.  See the section called ``\nameref{sec:asciijtex}'' for information
about ASCII Corporation's version of \jtex.
You can get the NTT version of \jtex\ from \verb|ftp.math.metro-u.ac.jp|
(133.86.76.25)

NTT \jtex\ is based on \TeX\ 3.14 and can read input files coded with JIS,
Shift-JIS, and EUC.  In addition to support for commercial Japanese fonts, NTT
\jtex\ includes a set of fonts generated from 24x24 dot bitmaps (JIS C-6234).
Unlike ASCII \jtex, NTT produces standard \ext{DVI} files.

\subsubsection{Poor Man's Japanese \protect\TeX}

\ixx{Poor Man's Japanese \TeX\ (\,\pmj)}{Poor Man's Japanese tex (pmj)@Poor Man's Japanese \TeX\ (\pmj)}\index{Japanese!Poor Man's Japanese tex@Poor Man's Japanese \TeX} is a freely-available Japanese typesetting system
that sits on top of standard \TeX.  The Japanese sections of the
input file must use the Shift-JIS encoding; a conversion program is supplied
to convert JIS encoded input files into Shift-JIS.  If you use another
encoding, such as EUC\index{EUC encoding}, you 
will have to find some way to convert it into
Shift-JIS before you can use \pmj.

\pmj\ solves the font problem in a clever way: \MF\ outlines for
Japanese characters are mechanically produced from freely-available
bitmaps.  This results in relatively low quality characters, but at
least they're free!

The Paulownia Court, the opening passage from the 800-year-old novel
\textit{The Tale of the Genji}, is shown in Figure~\ref{fig:pmj}.  This sample
was typeset by \pmj.

\epsbox[pgenji.eps]{fig:pmj}{Poor Man's Japanese}

\newpage
The \pmj\ documentation lists the following advantages and disadvantages:

\pmj\ Advantages

\begin{itemize}
  \item It is available now.
  \item It is free.
  \item It works with standard \TeX.
  \item It is device independent, but has relatively poor quality fonts.  The
    relative lack of quality is magnified by higher resolution output devices,
    unfortunately.
  \item It uses a set of free fonts mechanically produced from bitmaps.
\end{itemize}

\pmj\ Disadvantages

\begin{itemize}
\item It is somewhat crude and unlikely to be improved upon to any great extent.
\item It uses low quality fonts.
\item It requires a large number of fonts and as a result, lots of disk space.
\item It provides no access to slanted, bold, or other Japanese type-styles.
\item It cannot typeset vertically. 
\item It may take days to build the required fonts.
\end{itemize}

\subsubsection{\jemtex}

\ixx{\jemtex}{jemtex@Jem\TeX}\index{Japanese!jemtex@Jem\TeX} is a 
lot like \pmj; it uses fonts constructed from a collection of
24x24 dot bitmaps.  The \jemtex\ font maker includes a number of options for
tailoring the appearance of the characters.

A sample of Japanese typeset with \jemtex\ is shown in 
Figure~\ref{fig:jemtex}.

\epsbox[jgenji.eps]{fig:jemtex}{\protect\jemtex\ sample}

\jemtex\ takes a very different approach to processing Japanese text.
Instead of providing \TeX\ macros to interpret two-byte Japanese symbols
in the input file, \jemtex\ provides a preprocessor which translates
the Japanese input into equivalent \TeX\ input. The preprocessor understands
EUC\index{EUC encoding} and Shift-JIS input files.  

Using a preprocessor has
several advantages:

\begin{itemize}
  \item \TeX\ can process the files very quickly.  Because the input files
        are not edited by hand, they are designed to be processed quickly
        by \TeX\ rather than by human eyes.
  \item Only the fonts that are actually used must be loaded.  A system
        like \pmj\ must load all of the Japanese fonts because it does not
        know which ones will actually be used.  \jemtex\ knows exactly
        which fonts are required for each document.
  \item The preprocessor can handle subtle spacing issues automatically.
  \item The preprocessor can provide discretionary hyphens for \TeX, thereby
        allowing \TeX\ to hyphenate Japanese correctly.
\end{itemize}

\subsection{Chinese}

The general problems that apply to Japanese typesetting also apply to Chinese.

\subsubsection{Poor Man's Chinese \protect\TeX}

\ixx{Poor Man's Chinese \TeX\ (\pmc)}{Poor Man's Chinese tex (pmc)@Poor Man's Chinese \TeX\ (\pmc)}%
\index{Chinese}%
\index{Chinese!Poor Man's Chinese tex (pmc)@Poor Man's Chinese \TeX\ (\pmc)}%
\index{pmc (Poor Man's Chinese tex)@\pmc\ (Poor Man's Chinese \TeX)}
is closely related to \pmj.  The Chinese input
files should be encoded with 8-bit GB encoding (GB 2312-80).  If you use
another encoding, you will have to convert it into the 8-bit GB encoding
before you can use \pmc.

The \pmc\ package uses the same technique as \pmj\ to construct Chinese
fonts.  The relative advantages and disadvantages of \pmj\ apply equally
to \pmc.

Two sets of Chinese characters are available: traditional and simplified.
%% the sample text is apparently more than risque so I've removed it!
%A sample of Chinese created with \pmc\ using the traditional Chinese
%characters is shown in Figure~\ref{fig:pmct}.  A similar example, using
%the simplified characters, is shown in Figure~\ref{fig:pmcs}.
%
%\epsbox[tchinese.eps]{fig:pmct}{Poor Man's Chinese with traditional characters}
%
%\epsbox[schinese.eps]{fig:pmcs}{Poor Man's Chinese with simplified characters}

%\def\floatpagefraction{.3}
\subsection{Arabic}

Typesetting in Arabic\index{Arabic} can 
be accomplished with the \ixx{\arabTeX}{arabTeX@\arabTeX} package.
\arabTeX\ includes a complete set of fonts and macros for producing
documents in Persian, Arabic, and related scripts.  An example of
Arabic is shown in Figure~\ref{fig:flang:arab} on 
page~\pageref{fig:flang:arab}.

\epsbox[arab.eps]{fig:flang:arab}{Arabic text typeset with Arab\protect\TeX}

\newpage
\subsection{Hebrew}

Typesetting left-to-right Hebrew\index{Hebrew} (or occasional Hebrew words in an English
document) is relatively easy.  The required fonts and \TeX\ macros are
available from \path|noa.huji.ac.il| and on the CTAN archives in 
\ctandir{language/hebrew}.\footnote{At the time of this writing, the material
at \path|noa.huji.ac.il| is more up-to-date than the material in the CTAN
archives.}

Typesetting right-to-left Hebrew is more complicated.  First, you will
need an editor that handles right-to-left text entry, preferably one that
displays Hebrew text.  

After you have constructed a document that uses right-to-left Hebrew,
you will need a special version of \TeX, called \ixx{\XeT}{XeT@\XeT}, to process it.  
\XeT\ is a version of \TeX\ that understands right-to-left typesetting.

Early versions of \XeT, called \ixx{\TeX-\XeT}{TeX-XeT@\TeX-\XeT}, produced 
nonstandard \ext{DVI} files called \ixx{\ext{IVD} files}{IVD files@\ext{IVD} files}.  If you 
use \TeX-\XeT, a special program called
\ixx{\program{ivd2dvi}}{ivd2dvi} must be used to translate the \ext{IVD} files into
\ext{DVI} files before they can be printed.  More recently, \XeT\ has been
reimplemented to produce standard \ext{DVI} files.  The new version is called
\ixx{\TeX\hbox{-{}-}\XeT}{tex--XeT@\TeX\hbox{-{}-}\XeT}.\footnote{Yes, the only difference between the names really
is the number of hyphens!}  They are functionally identical.
You can get \Unix\ and PC versions of \XeT\ from \path|noa.huji.ac.il|.
An example of Hebrew is shown in Figure~\ref{fig:hebrew} on page~\pageref{fig:hebrew}.

\epsbox[hebrew-sample.eps]{fig:hebrew}{Hebrew}

\egroup % <<<<<<<<<<<<<<<<<<<<<<<<<<<<<<<<<<<<<<<<<<<<<<<<<<<<<<<<<<<<

\documentstyle{article}

\begin{document}

\input feynman

\begin{picture}(20000,20000) % Start at upper right.
\thicklines\drawline\photon[\SE\REG](18000,18000)[5]
\drawline\fermion[\S\REG](\photonfrontx,\photonfronty)[\boxlengthy]
\drawarrow[\N\ATBASE](\pmidx,\pmidy)
\drawline\fermion[\W\REG](\photonbackx,\photonbacky)[\photonlengthx]
\drawarrow[\W\ATTIP](\pmidx,\pmidy)
% Draw 2 small lines to connect the vector meson (parallel lines) at a corner:
\thinlines\drawline\fermion[\S\REG](\fermionbackx,\fermionbacky)[150]
\drawline\fermion[\W\REG](\fermionfrontx,\fermionfronty)[150]
\drawline\fermion[\SW\REG](\fermionbackx,\fermionbacky)[7000] % Upper || line
\drawline\fermion[\S\REG](\fermionbackx,\fermionbacky)[75] % Find the centre of
\drawline\fermion[\E\REG](\fermionbackx,\fermionbacky)[75] %the double fermion
\drawline\gluon[\SW\REG](\fermionbackx,\fermionbacky)[5]
\put(\gluonfrontx,\gluonfronty){\circle*{500}} % A `blob': the vector->gauge trans.
\drawline\fermion[\S\REG](\fermionbackx,\fermionbacky)[75] % Draw to position the
\drawline\fermion[\E\REG](\fermionbackx,\fermionbacky)[75] % Second parallel line
\drawline\fermion[\NE\REG](\fermionbackx,\fermionbacky)[7000] % Lower || line
\put(\gluonbackx,\gluonbacky){\circle*{500}}
\gaplength=300
\drawline\scalar[\NW\REG](\gluonbackx,\gluonbacky)[3] % Left half of scalar
\thicklines\drawarrow[\NW\ATBASE](\scalarbackx,\scalarbacky)
\gaplength=300 \thinlines
\drawline\scalar[\SE\REG](\gluonbackx,\gluonbacky)[3] % Right half of scalar
\put(\pfrontx,\scalarbacky){{\bf H}$^0$}
\put(\gluonfrontx,\gluonfronty){\ $\,{}_{\longleftarrow f_6(\omega,p_+\cdot q_-)}$}
\drawline\fermion[\NW\REG](\photonfrontx,\photonfronty)[2000]
\drawarrow[\NW\ATBASE](\pbackx,\pbacky)
\drawline\fermion[\SE\REG](\photonbackx,\photonbacky)[2000]
\drawarrow[\NW\ATTIP](\pmidx,\pmidy)
\put(\pbackx,\pbacky){$\,\,p+q$} % \, gives extra space in math mode.
\end{picture}

\end{document}

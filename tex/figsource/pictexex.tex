\nopagenumbers
%%%%%%%%%%%%%%%%%%%%%%%%%%%%%%%%%%%%%%%%%%%%%%%%%%%%%%%%%%%%%%%%%%%%%%%%%
\magnification=\magstephalf
{\obeyspaces\global\let =\ }
\input pictex

    \newdimen\unit   \unit=1.375in
    \newdimen\shadeunit
    \newif \ifFirstPass
    \FirstPasstrue
%    \FirstPassfalse  % uncomment this after the PiCture is the size you want

    \def\DF{{\cal A}}%

%The following figure appears in the preface to the \PiCTeX\ manual
$$   \beginpicture
      \ifFirstPass
        \savelinesandcurves on "pictex-Arcsine.tex"
      \else
        \replot "pictex-Arcsine.tex"
      \fi
%      \ninepoint  %   (See Appendix E of the TeXbook.)
      \normalgraphs
      % Density plot
      \setcoordinatesystem  units <\unit,.4\unit>  point at 0 0  
      \setplotarea x from 0 to 1, y from 0 to 2.5 
      \axis bottom invisible label {\lines {$t$\cr
        shaded area is $\DF(\beta) - \DF(\alpha)$\cr}} ticks
          numbered from 0.0 to 1.0 by 0.5   
          unlabeled short quantity 11
          length <0pt> withvalues $\alpha$ $\beta$ / at .65 .85 /  /
      \axis left invisible label {$a(t)$} ticks
        numbered from 0.0 to 2.0 by 1.0
        unlabeled short from 0.5 to 2.5 by 1.0  /
      \plotheading{\lineskiplimit=1pt \lines{%
         Density\cr
         $a(t) = 1\big/\bigl(\pi \sqrt{t(1-t)}\,\bigr)$\cr
         of the arc sine law\cr}}
      \grid 1 1
      \putrule from  .65 0.0  to  .65 .66736
      \putrule from  .85 0.0  to  .85 .89145
      \shadeunit=.2\unit  \divide\shadeunit by 12
      \setshadegrid  span <\shadeunit>  point at .75 0
      \setquadratic
      \vshade .65 0 .66736   <,,,1pt> .75 0 .73511   .85 0 .89145  /
      % Move origin to (.5,0)
      % Left half}
      \ifFirstPass
        \setcoordinatesystem  point at -.5 0
        \inboundscheckon
        \plot  -.48429 2.55990  -.47553 2.06015  -.46489 1.72936  /
        \inboundscheckoff
        \plot  -.46489 1.72936  -.43815 1.32146  -.40451 1.08308  
               -.36448  .92999  -.31871  .82623  -.26791  .75400 
               -.21289  .70358  -.12434  .65727   .0      .63662  /
        % Right half
        \inboundscheckon
        \plot   .48429 2.55990   .47553 2.06015   .46489 1.72936  /
        \inboundscheckoff
        \plot   .46489 1.72936   .43815 1.32146   .40451 1.08308  
                .36448  .92999   .31871  .82623   .26791  .75400  
                .21289  .70358   .12434  .65727   .0      .63662  /
      \fi
      % Distribution function 
      % Set origin of new coordinate system 1.7*1.375in=2.34in
      %   to the right of the original origin.
      \setcoordinatesystem  units <\unit,\unit>  point at -1.7 0
      \setplotarea x from 0 to 1, y from 0 to 1
      \axis bottom label {$x\vphantom{t}$} ticks
        numbered from 0.0 to 1.0 by 0.5  unlabeled short quantity 11 /
      \axis left label {$\DF(x)$} ticks
        numbered from 0.0 to 1.0 by 0.5  unlabeled short quantity 11 /
      \plotheading{\lines{%
        Distribution function\cr
        $\DF(x) = {2\over \raise1pt\hbox{\seveni ^^Y}} 
          \arcsin(\sqrt{x}\,)$\cr
        of the arc sine law\cr}}
      \linethickness=.25pt  \grid {20} {20}  
      \linethickness=.4pt   \grid 2 2
      % Left half
      % Now move origin of coordinate system up to (.5,.5)
      \ifFirstPass
        \setcoordinatesystem  point at -2.2 -.5
        \plot           -.50000 -.50   -.49901 -.48   -.49606 -.46 
          -.49104 -.44  -.48439 -.42   -.47553 -.40   -.46489 -.38 
          -.43815 -.34  -.40451 -.30   -.36448 -.26   -.31871 -.22 
          -.26791 -.18  -.21289 -.14   -.12434 -.08    .0      .0  /
        % Right half
        \plot            .50000  .50    .49901  .48    .49606  .46 
           .49104  .44   .48439  .42    .47553  .40    .46489  .38 
           .43815  .34   .40451  .30    .36448  .26    .31871  .22 
           .26791  .18   .21289  .14    .12434  .08    .0      .0  /
      \fi
    \endpicture
$$
\bye

\centerline{\it FIGURE 1}
\bigskip\noindent
You're welcome to use that. 

\bigskip
Here are a couple of passages from the preface to the \PiCTeX\ manual
that may be of some help to you (keep in mind that the manual is
copyrighted, so this material should be quoted if used verbatim):

\medskip
\bgroup \narrower
\PiCTeX\ offers these advantages:  (1)~Figures 
become an integral part of the typesetting process. You can avoid 
having to leave the proper amount of space in your document for material
that has to be created on some external device and later stripped into
the finished product. 
(2)~All of \TeX's formatting capabilities are available for annotating your
figures.  In addition, that annotation will be done (if you so desire)
in the same fonts as you're using in the rest of your document.
(3)~Just as \TeX\ is machine independent, so too is \PiCTeX. It doesn't
matter whether you're working on a PC or mainframe computer.
(4)~Since typeset figures are embedded in the
{\tt dvi} file along with the rest of your document, all the advantages
of \TeX's device independent output accrue to them.  In particular,
you can revise away to your heart's content on your local system until
things are just the way you want them, and then you can have the final
copy elegantly (but perhaps expensively) printed on a high resolution output 
device.  
(5)~\PiCTeX\ can easily be extended using \TeX's macro facilities.

On the other hand, \PiCTeX\ has several limitations: (1)~\PiCTeX\ was
expressly designed to facilitate the construction of pictures such as
Figure~1.  It simply is not the right tool for producing illustrations 
such as the lions that grace the title pages of the \TeX book.
(2) Within the realm of mathematical figures, ~\PiCTeX\ doesn't 
make 3D pictures or other complex things. Considering that \TeX\ provides
less arithmetic capabilities than the simplest pocket calculator, that
would be asking for too much.
(3)~\PiCTeX\ takes a while to draw a \PiC ture.  
Figure~1 initially took % $3{1\over 2}$~minutes on a Sun-2/120, 
%all but 30~seconds of the effort going into producing the two curves. 
%In subsequent drafts \PiCTeX\ replotted the curves in 40~seconds.
12 seconds on Spark~10. In subsequent drafts
\PiCTeX\ replotted the curves in less than 4 seconds.
(4)~\PiC ture s take up a sizeable amount of computer memory,
both within \TeX\ and within the software that processes the {\tt dvi} file.
For example, Figure~1 occupies about 15~Kilobytes. A larger \PiC ture, with
several more curves, could easily exceed the constraints of a small system.
\par
\egroup

\bigskip
Some additional information on \PiCTeX follows:
\medskip
\bgroup 
\def\par{\leavevmode\endgraf}
\obeylines \obeyspaces \tt %

Merchandizing of the PiCTeX manual is being handled by
TUG. Contact tug@tug.org or write to TUG at 

\quad  TeX Users Group
\quad  P.O. Box 869
\quad  Santa Barbara, CA 93102

I believe TUG is asking \$35.00 for the manual, with a \$5.00 
discount to members. The manual is in book form, about 100 
pages long.

There is a survey article on PiCTeX in TUGboat \#9 2 pp. 193-195 
(1988).
\par
\egroup
\bye

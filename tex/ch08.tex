\chapter{Printing}
\RCSID$Id: ch08.tex,v 1.1 2002/08/23 14:58:46 nwalsh Exp $
\label{chap:printing}

\ifincludechapter\else\endinput\fi

\index{printing}\index{tex@\TeX!printing}
The emphasis in this book so far has been on getting \TeX\ to process a 
document that
contains all the desired typographic elements without error.
The result of this effort is a \ext{DVI} file\index{DVI files}.

The next step is translating the \ext{DVI} file into a printed 
document\index{documents!printed}.
That is the focus of this chapter.  Usually, you want to preview a document
before you print it, but in many ways previewers are just a special kind
of printer. The two do differ significantly, however, so we will discuss
them separately.
(Chapter~\ref{chap:preview}, {\it\nameref{chap:preview}}, describes 
previewers.)

This chapter explores the issues related to printing a DVI file. 
There are sections concerning the printing of fonts, the printing of 
pictures and figures, and descriptions of several kinds of drivers 
you can use to print \TeX\ documents.

\section{Printing Fonts}
\label{sec:printing}
\index{printing!fonts}\index{fonts!printing}\nobreak

If you have 
a \ext{TFM} file for a font, you can use that font in \TeX.
In fact, you can use that font at an arbitrary magnification in \TeX,
which means that you can use Courier at 13.4pt as easily as Computer
Modern Roman at 10pt.
Unfortunately, this ease of use does not imply that the resulting \ext{DVI}
file will be easy to print.  It doesn't even imply that it will be possible
to print the document.  For example, if you request Courier at 13.4pt, but
it is available in your printer only at 10pt and 12pt (and you have no
other source for Courier),
there is no way to print your document without distortion.

This section discusses the issues involved in getting the desired fonts 
to print from your DVI file. Both bitmapped
and scalable fonts are considered, as well
as fonts built into the printer.
You will also gain a better understanding of why some documents do not
print, and learn alternatives that may enable you to print your
documents.

Every font can be classified in two broad, independent ways: internal
versus external, and scalable versus bitmapped.  Each class of fonts
has some advantages and some unique problems.  In general, there
are more restrictions on built-in fonts than on external fonts, and
more restrictions on bitmapped fonts than on scalable fonts.

\subsection{Built-in Fonts}

Built-in fonts\index{fonts!built-in!printing}, whether 
scalable or bitmapped, pose two problems.
First, you must obtain the 
appropriate \ext{TFM} files\index{TFM files!for printing}.  Usually, the
metric information has to be supplied by the vendor and then translated
into \TeX\ \ext{TFM} format using a conversion tool.  No vendor that I
am aware of distributes \TeX\ metric\index{metrics font} information 
directly.  
The section called ``\nameref{sec:gettingtfms}'' in Chapter~\ref{chap:fonts},
{\it\nameref{chap:fonts}},
describes several
options for obtaining metric information for built-in fonts.

The second problem is that the \dvidriver\ you are using must ``know''
somehow that the \ext{TFM} file corresponds to one of the printer's
internal fonts.  When the \dvidriver\ examines the \ext{DVI} file, the
only information available about each font is the name of the
\ext{TFM} file that describes its metrics.  
The most common way \dvidriver{}s handle this problem is with 
a font translation file or
a font substitution file.  

\subsubsection{Font substitution in \protect\emTeX}

\emTeX's \dvidriver\  reads 
a user-specified \ixx{font substitution file}{emTeX@\emTeX!font substitution file} before 
processing the \ext{DVI} file.  A line like the following in
the font substitution file informs the \ixx{\program{dvihplj}}{dvihplj} driver that
\filename{lpr1610u} is an internal font and describes the LaserJet
control sequence required to select it:

\begin{exindent}
  \verb|lpr1610u 300 => pcl: 10U s0P s16.66H s8.5V s0S s0B s0T|
\end{exindent}

Other substitutions are also possible.  For example, if you 
work with documents that come from other systems that    allow long
filenames, you can substitute the long names for shorter filenames
that are legal under MS-DOS.  The following line in a font substitution
file tells \emTeX\ that the font ``{\tt tmsrmn}'' should be used anywhere
that a \ext{DVI} file uses the font ``{\tt Times-Roman}'':

\begin{exindent}
  \verb|Times-Roman -> tmsrmn|
\end{exindent}

You can use even more sophisticated substitutions, including pattern matching, 
for example.  Consult {\it The {\emTeX} DVI Driver Manual}~\cite{em:dvidrv} 
for more information.

\subsubsection{Font substitution in \protect\dvips}

The \program{dvips}\index{dvips!font substitution file} program, 
which translates \ext{DVI} files into
PostScript, also reads a font substitution file before processing a
\ext{DVI} file.  In addition to identifying which fonts are built into the
printer, the substitution file can also instruct \program{dvips} to 
download PostScript fonts that are not resident in your printer 
if they are used in a document.

\program{dvips} has a system-wide font substitution file called
\filename{psfonts.map}\index{psfontsmap files@psfonts.map files}.  This file 
is distributed with \program{dvips}
and can be customized by your system administrator.  You can tell
\program{dvips} to load a personal font substitution file by
using the \textit{p} command in \program{dvips}'s initialization
file (called \filename{.dvipsrc} in your home directory on \Unix\ systems).
Consult the \program{dvips} documentation for more information
about setting up an initialization file.

The following lines in a font substitution file indicate that the font
\filename{grbk} (which is the name of the \ext{TFM} file) corresponds
to the built-in printer font Garamond-Book.
The \ext{TFM} file \filename{hlvcd} corresponds to the
printer font Helvetica-Condensed.  Helvetica-Condensed is
not built into the printer but is stored in the file
\filename{/home/walsh/fonts/helvcd.pfb}.  If \filename{hlvcd} is used in
a document, \program{dvips} will download \filename{helvcd.pfb} automatically
when it converts the document.

\begin{shortexample}
grbk   Garamond-Book
hlvcd  Helvetica-Condensed </home/walsh/fonts/helvcd.pfb
\end{shortexample}

\subsubsection{Font substitution in other drivers}

Other \dvidriver{}s, particularly those written for operating environments
like Microsoft Windows that
provide a standard interface to printers, may
leave the distinction between built-in and external fonts up to the
operating environment.  If this is the case, you must use the tools
provided by the operating system 
to support the printers and fonts that you require.

If your \dvidriver\ does use a font substitution file, 
make sure that the translations specified actually exist.  The
\dvidriver\ cannot practically determine if the font you have
specified as built into the printer really exists or not.  If it
doesn't, you won't get the output you expect, and the \dvidriver\ will
not be able to diagnose the problem.

The most significant disadvantage of using built-in fonts is that they
are not usually available for on-screen previewing of your document.
Some form of font substitution has to be employed to 
preview a document.  Sometimes this reduces the utility of on-screen
previewing.  Chapter~\ref{chap:preview}, {\it \nameref{chap:preview}},
discusses this issue in more detail.

\subsection{External Fonts}

External fonts\index{fonts!external}, those 
that aren't built into the printer, pose a
different set of problems.  First, they have to be available to the
\dvidriver, and they have to be represented in a manner that the
\dvidriver\ understands.  For the vast majority of \dvidriver{}s, this
means that the fonts must be stored in files on your hard disk in
\ext{PK} format.  The \ixx{\ext{PK}}{PK fonts@\ext{PK} fonts} files contain a compressed,
bitmapped rendering of the font at a particular resolution.  For more
information about \ext{PK} files and other bitmapped-font issues, see
``\nameref{sec:bitfonts}'' later in this chapter.

Most \dvidriver{}s locate fonts by searching in the directories contained
in an environment variable or in some system-dependent locations.  

Once the fonts are available, the \dvidriver\ has two options for
using them: it can download them to the printer, or it can send them as
bitmapped graphic images.  Most laser printers can accept downloadable
fonts, although they may require additional memory to accept large
numbers of fonts.  In either case, the printable files are generally
larger and print more slowly than documents that use the printer's
built-in fonts.  Of course, using external fonts does give you far
more flexibility than using only built-in fonts.

Unlike a missing internal font, if you attempt to use an external font
that is not available, your \dvidriver\ will be able to detect the
problem and may be able to substitute another font or compensate in
some other way.

\begin{sidebar}[Increasing Printing Speed]

How can you increase printing speed\index{printing!increasing speed} if 
you frequently print, on a laser printer,
documents that use a small number of external fonts
(for example, several sizes of Computer Modern)?  A handy trick is to 
download the
fonts manually and then tell the \dvidriver\ that the fonts are
built-in.  

Suppose you use the 7pt, 10pt, and 12pt sizes of
Computer Modern Roman, and the 12pt and 14pt sizes of Computer Modern
Bold Extended in most of your documents.  If you are printing on an HP
LaserJet printer with \emTeX's \program{dvihplj}, you could convert
the relevant \ext{PK} files  into HP LaserJet softfonts with \ixx{\program{PKtoSFP\/}}{PKtoSFP@PKtoSFP\/} (see the discussion of bitmapped fonts in
the next section to determine which \ext{PK} 
files are relevant).
Then download them to the printer every morning with a utility like
\ixx{\program{Sfload}}{Sfload}.  \program{Sfload} is part of the \ixx{\program{Sfware}}{Sfware}
package. (Disclaimer: I wrote \program{Sfware} so I'm partial to it.
There are many other programs that can download softfonts.)  In the
font-substitution file for \ixx{\program{dvihplj}}{dvihplj}, simply indicate that
those fonts are built-in,
and \program{dvihplj} will no longer download
them for each document.  This can significantly decrease the amount of
data that must be sent to the printer for each document.

Beware, however, that if you fail to download the fonts, you will
get very badly garbled output when you print a document.
\end{sidebar}

\subsection{Bitmapped Fonts}
\label{sec:bitfonts}

Each character in a bitmapped font is represented by a rectangular
array of dots.  Some of the dots in this array are ``on,'' and some are
``off.''  When the dots are printed very close together, they provide
the illusion of a solid character.  Bitmapped fonts are common in the
\TeX\ community because all of the fonts created by \MF\ are printed
in bitmapped form.

The notion of magnification discussed in the section called
``\nameref{sec:issueofsize}'' in Chapter~\ref{chap:fonts}
is interpreted as an issue of resolution
when dealing with bitmapped fonts.

Resolution\index{resolution of images} is used 
in a slightly counter-intuitive way by
\dvidriver{}s.  Generally, resolution is described as a feature of a
device that affects the {\em appearance\/} of images printed on that
device and not their {\em size}.  For example, in comparing two
drawings in which one was printed on 150dpi
dot-matrix printer and the other
was printed on a 300dpi laser printer, I might say, ``the laser printed page
looks better because it was printed at a higher resolution.''

That's true, but what isn't usually stated explicitly is that the
comparison is between a 150dpi drawing rendered at 150dpi and a
300dpi drawing rendered at 300dpi.  This is the situation shown
in Figure~\ref{fig:reschange} where a 4dpi image is printed at 4dpi
next to an 8dpi image printed at 8dpi.

\begin{figure}
  \begin{center}
    \input{ft-resch.fig}
    \caption{Resolution of the bitmap and the device changed simultaneously}
    \label{fig:reschange}
  \end{center}
\end{figure}

What if the resolution of the printer were held constant?  That's
the situation shown in Figure~\ref{fig:sizechange}.  The 4dpi image printed
at 4dpi is shown next to the 8dpi image printed at 4dpi.  The result
is that the size of the image is doubled.

\begin{figure}
  \begin{center}
    \input{ft-sizch.fig}
    \caption{Resolution of the bitmap changed while device held constant}
    \label{fig:sizechange}
  \end{center}
\end{figure}

This is the technique that \dvidriver{}s use to print a larger 
magnification of the same font; they print a version of the font
designed for a correspondingly higher-resolution device.  The same
technique is used to print at smaller magnifications.

Most \dvidriver{}s use bitmapped fonts stored in \ixx{\ext{PK} files}{PK files@\ext{PK} files}.  The
\ext{PK} format is a highly compressed binary format.\footnote{For a
complete, detailed description of the \ext{PK} and \ext{GF} formats,
consult {\it The GF to PK Processor}~\cite{mfware:gftopk}.}  
Two other bitmapped font formats
(sometimes accepted by \dvidriver{}s) are associated with \TeX:
\ext{GF}\index{GL format} and \ext{PXL} files\index{PXL format@\ext{PXL} format}.  
The \ext{GF} format is a very flat,
uncompressed bitmap format produced by several utility programs,
including \MF.  The \program{\idx{GFtoPK}} program converts \ext{GF} files
into \ext{PK} files.  The \ixx{\ext{PXL} format}{PXL format@\ext{PXL} format} is an uncompressed bitmap
format; it has been superseded by the \ext{PK} format (which achieves
better compression) and is completely obsolete.  If you still have
\ext{PXL} files, you should convert them to \ext{PK} format with the
\ixx{\program{PXtoPK}}{pxtopk} utility.  If you are using a \dvidriver\ that still
requires \ext{PXL} files, you should find out about an upgrade;
the program is obsolete.

The fact that \dvidriver{}s use different resolutions of the same
font file to obtain different magnifications introduces
a naming problem.  How can the \dvidriver{} distinguish between
\filename{cmr12.pk} at 300dpi and \filename{cmr12.pk} at 360dpi (or any
other resolution)?  

On \Unix\ systems, this problem is usually resolved by putting the
resolution in the filename in front of the extension \textit{pk}.
For example, \filename{cmr12.300pk} is \filename{cmr12} at 300dpi while
\filename{cmr12.360pk} is \filename{cmr12} at 360dpi.

On many other systems, where the operating system imposes limits on
the length of filenames, a solution is achieved by storing the
fonts in different subdirectories.  On MS-DOS, for example,
the 300dpi version of \filename{cmr12} might be stored in 
\textit{\bs texfonts\bs 300dpi\bs cmr12.pk} while the 360dpi version is
stored in \textit{\bs texfonts\bs 360dpi\bs cmr12.pk}.    

In either case, you should obey the conventions specified by your
\dvidriver\ to assure that the \dvidriver\ can find the
fonts.  One common solution to the problem of finding fonts is to
use an environment variable to list the directories where \ext{PK}
files occur.
For example, on a \Unix\ system, the environment variable \verb|TEXPKS|
might hold a list of directories separated by colons:\footnote{A backslash
is used here to escape the end of the first line.  This is a standard way
to continue a line in most \Unix\ shells.  Naturally, you can simply enter
it as one line in your editor, if you prefer.}

\begin{shortexample}
TEXPKS=/usr/local/lib/tex/fonts:/usr/local/lib/tex/fonts/pk:\
/usr/local/lib/mf/fonts:/usr/local/lib/mf/fonts/pk
\end{shortexample}

Under MS-DOS, the environment variable \verb|DVIDRVFONTS| might
hold a list of directories separated by semicolons:

\begin{shortexample}
DVIDRVFONTS=C:\TEXFONTS;C:\MYFONTS
\end{shortexample}

The format of the environment variable and how it is created or modified
is determined by the operating environment that you are using.  The name
of the environment variable differs according to the \dvidriver.  

To recap, \TeX\ works 
with {\em magnifications\/}\index{magnification!converting  to resolution}\index{magnification!of images} of abstract,
scalable measurements; \dvidriver{}s work with {\em resolutions\/} of
fixed, bitmapped images.

To convert a magnification into a resolution, multiply the resolution of
your output device by the magnification.  A font at a magnification of
120\% on a 300dpi laser printer has a resolution of 360dpi.  

To convert a resolution into a magnification, divide the font resolution by
the resolution of the printer.  A 420dpi font on a 300dpi laser printer
has a magnification of 140\%.

\subsection{Scalable Fonts}

Although \ixx{scalable fonts}{fonts!scalable} used to 
be very uncommon, the proliferation of 
PostScript
printers and products like Adobe Type Manager (not to mention built-in
support for TrueType fonts in Apple System 7.0 and Microsoft Windows 3.1)
have made these fonts very common.
It is important to remember that scalable fonts are ultimately
converted into bitmapped fonts.  All printers eventually treat the
page as a very large bitmap and either do or do not deposit a small
amount of ink in each position in the bitmap.

Most \dvidriver{}s leave the work of performing this rasterization to 
someone else.  If you are not using a PostScript printer,
you probably rely on some other piece of software to do the
work.  Adobe Type Manager and built-in support for TrueType are the most
common software solutions.
Some printers, like the LaserJet III, have built-in scalable fonts as
well, even though they are not PostScript printers.

\subsection{Font Printing Pitfalls}

Sometimes you need a font at a size that is not available.  Because scalable
fonts are available at any size, this is a problem only with
bitmapped fonts\index{metafont@\MF!fonts}\index{fonts!metafont@\MF}.  Remember 
that \MF\ creates bitmapped fonts, but they
can be scaled because the \ext{MF} source for the font is not
a bitmap.  You can't scale the \ext{PK} file, but you can generate a new
\ext{PK} file at the size you need.  In order to do this, you must
have \MF\ installed, and you must have the \ext{MF} source for the font.
Chapter~\ref{chap:mf}, {\it \nameref{chap:mf}}, describes how to create a 
\MF\ font at any size.

If the font isn't a \MF\ font, or you don't have the \ext{MF} source,
you may still be able to scale the font.  However, you should be aware
before you try that the result may be unacceptable.  Scaling bitmapped fonts
causes ugly, jagged edges if the font is scaled larger, and loss of
detail if it is scaled smaller.  Small changes in size are sometimes
manageable.

The \program{dvips} driver will scale bitmapped fonts if it cannot
find or build the requested font.  Alternatively, the \ixx{\program{sffx}}{sffx}
program can scale HP LaserJet softfonts.  I don't know of any other
scaling options.

Non-scalable fonts that are built into the printer cannot be
scaled at all.  If you need to have a built-in font at an unavailable
size, you will have to substitute another font in place of the built-in
one.

\section{Printing Pictures and Figures}

Pictures\index{printing!pictures}\index{pictures!printing} and 
figures\index{printing!figures}\index{figures!printing} are 
frequently the least portable elements of a document.
Chapter~\ref{chap:pictures}, {\it \nameref{chap:pictures}}, describes many of
the options that are available for including pictures and figures in a 
document.  

If the method used is supported by the \dvidriver\  and
printer that you use, pictures and figures are transparently printed.
They are even less difficult to print than fonts.
On the other hand, if you are attempting to print a document which
incorporates pictures and figures using a method not supported by your
\dvidriver\ or printer, it may be exceptionally difficult to print
the document.

\subsection{Unsolvable Problems}

\ext{DVI} files are not always ``complete'' with respect to pictures
and figures.  Many documents
use \cs{special} commands to access \dvidriver\ or printer-specific 
features in order to include pictures and figures.  These
\cs{special}s may have just the name of an external file that
contains the graphic image to be included.  For example, 
using the \emTeX\ drivers, I can include a bitmapped graphic image
with the command \cs{special}\verb|{em:graph spslogo}|.  When \emTeX's
\dvidriver{}s process this command, the graphic image in the file
\filename{spslogo.pcx} is included in the output.  If I transmit
this \ext{DVI} file to another computer but forget to transmit the
file \filename{spslogo.pcx}, there is no way to print the
\ext{DVI} file with the SPS logo.

At first, it may seem that a good solution to the problem described
above would be to include the actual data for the image in the
\cs{special} command.  But to do that, \TeX\ would have to
process the image data, defeating the purpose of the \cs{special}
mechanism.  The \cs{special} mechanism is better because it is open
ended---it can handle new types of graphics, for example, without 
changing \TeX.

The second, essentially unsolvable, problem involves
\ext{DVI} files containing \cs{special} commands that
have embedded data.  For example, one set of picture drawing \cs{special}
commands are the \verb|tpic| \cs{special}s.  If your \dvidriver\
does not understand these commands,
there is no practical way to extract them from the \ext{DVI} file
in order to convert them into a format your \dvidriver\ understands.
There may not even be a practical way of extracting them from the \TeX\
document to construct a printable image.
You simply can't print that document with the \verb|tpic| figures intact.

\subsection{Solvable Problems}

If you have a document that includes an external picture or figure
using a method that your \dvidriver\ or printer does not understand,
and you have the file that contains the graphic image and
the \TeX\ source for the document, you may be able to print it.

\newpage
The section called ``\nameref{sec:picconv}'' in Chapter~\ref{chap:pictures},
lists a wide variety of picture conversion
tools that may allow you to convert the image into a printable form.
For example, if the document in question uses a \cs{special}
command to include a \ext{PCX} graphic
image, but your \dvidriver\ only understands the Macintosh \ext{PICT} format,
you could use the \ixx{\program{PBMplus}}{PBMplus} utilities to convert the \ext{PCX}
image into \ext{PICT} format.

A conversion that might frequently be necessary if you work on both
PostScript and non-PostScript printers is  conversion from encapsulated
PostScript to a bitmap format.
This is a translation that \program{Ghostscript} or some other PostScript
interpreter can perform.

\subsection{Pictures Using Only \protect\TeX}

Device-independent pictures created using only \TeX\ pose no particular
printing problems.  These include the \LaTeX\ \verb|picture| environment,
as well as the \PiCTeX\ and \program{\XYPic} macro packages.

\subsection{\protect\MF\ Figures}

Pictures\index{metafont@\MF!pictures}\index{pictures!metafont@\MF} created 
with \MF\ (using \program{\MFPic}, \program{\FigMF}, or \MF\ directly) are
really \MF\ fonts.  As a result, they can be created for almost
any raster output device.\footnote{Because figures are frequently much
larger than individual characters, \MF\ may have difficulty with large
figures at very high resolutions.}
As long as \MF\ is available and the figures aren't too large, they pose no
problems. Very large images may break some \dvidriver{}s or may not be
printable on some output devices.

\subsection{Scalable Images}

Scalable formats\index{images!scalable (vector)}, 
primarily PostScript and HPGL, are difficult to convert
to other printing technologies.  However, the \ixx{\program{Ghostscript}}{Ghostscript}
program is one option for PostScript figures, and \ixx{\program{hp2xx}}{hp2xx} can
convert a subset of HPGL commands into other formats.

\subsection{Bitmap Images}

Bitmap images\index{images!bitmap} are generally the easiest to convert from one format
to another, but there is another issue that is more difficult to 
deal with---resolution.  Printing a bitmap image at 1200dpi if the
original is only 300dpi is not easy.

\section{Selected Drivers}

Table~\ref{tab:drivers} lists a number of 
common \dvidriver{}s\index{DVI drivers}\index{printing!DVI drivers} used
to print \TeX\ documents (as opposed to previewing them).  These drivers
are discussed in more detail in this section.
The list of drivers described here is nowhere near
complete.  The presence or absence of a particular driver from this
list is not intended as a reflection on the quality of the driver.
I tried to select a representative set of free and commercial drivers.
Drivers for other printers usually offer similar features.

\begin{xtable}{l|l|l|l}
  \caption{Common DVI Drivers\label{tab:drivers}}\\
  \bf \dvidriver         & \bf Supplier  &  \bf System      & \bf Printers \\[2pt]
  \hline
  \tstrut
  \program{dvihplj}      & Free (\emTeX) &              & 
    HP LaserJet compatible \\
  \program{dvidot}       & Free (\emTeX) & MS-DOS+OS/2  & 
    Most dot matrix \\
  \program{dvipcx}       & Free (\emTeX) &              & 
    \ext{PCX} graphic images\\[4pt]
  \program{dvilj2}       & Free ({\it dvi2xx}) & Most         & 
    HP LaserJet compatible \\[4pt]
  \program{dvips}        & Free ({\it dvips})  & Most         & 
    PostScript \\[4pt]
  \program{DVILASER/HP}  & ArborText     & MS-DOS, Unix & 
    HP LaserJet compatible\\
  \program{DVILASER/PS}  & ArborText     & MS-DOS, Unix & 
    PostScript \\[4pt]
  \program{PTI Laser/HP} & Personal \TeX & MS-DOS       & 
    HP LaserJet compatible \\
  \program{PTI Laser/PS} & Personal \TeX & MS-DOS       & 
    PostScript\\
  \program{PTI Jet}      & Personal \TeX & MS-DOS       & 
    HP DeskJet compatible\\[4pt]
  \program{dvipsone}     & \YY          & MS-DOS       & 
    PostScript\\[2pt]
  \hline
\end{xtable}

\subsection{\protect\emTeX\ Drivers}

The \emTeX\index{emTeX@\emTeX!drivers} distribution 
includes three \dvidriver{}s: \ixx{\program{dvihplj}}{dvihplj} for
printing on HP LaserJet, DeskJet, PaintJet, QuietJet printers 
(as well as the Kyocera laser printer), \ixx{\program{dvidot}}{dvidot} 
for printing on a wide variety of dot-matrix printers, and
\ixx{\program{dvipcx}}{dvipcx} for translating DVI files into
\ext{PCX} graphics images (for faxing, for example).
The printers that \program{dvidot} supports are listed below.  
If your dot-matrix
printer isn't listed, detailed instructions in the \emTeX\ documentation
will probably allow you to construct an appropriate parameter file.

{\LTleft=.25in
\begin{longtable}{ll}
  Apple Imagewriter          & IBM Proprinter 4202        \\
  C.ITOH 8510A               & IBM Proprinter 4207        \\ 
  Canon Bubble Jet BJ-10e    & IBM Proprinter 4208        \\
  EPSON FX and RX series     & NEC P6, Panasonic KX-P1124 \\
  EPSON LQ series            & NEC P7                     \\
  IBM Proprinter 4201        & Tandy DMP-130
\end{longtable}%
}
\vspace{-4pt}
All of the \emTeX\ drivers support several \cs{special} commands for
including bitmapped graphics and lines at any angle.

The Computer Modern fonts in \ext{PK} format occupy several directories
and considerable space on disk.  To minimize the impact of keeping several
magnifications of fonts around, the \emTeX\ drivers introduced the concept
of \textit{font libraries}.  \ixx{Font libraries}{font libraries} are 
single files that contain many,
many \ixx{\ext{PK} fonts}{PK fonts@\ext{PK} fonts}\index{fonts!PK}.  The fonts 
distributed with \emTeX\ are distributed
in font library format.  The \program{fontlib} program, distributed with
\emTeX, allows you to create and maintain font libraries of your own.

The \emTeX\ drivers support automatic font generation starting with 
version 1.4S.  This is accomplished by a second program, \ixx{\program{dvidrv}}{dvidrv},
that runs the \dvidriver\ and then \ixx{\program{MFjob}}{MFjob}, if required, to build
the fonts.  The relationship between these components is shown in
Figure~\ref{fig:emdvi}.

\epsimage{fig:emdvi}{Previewing and printing with \emTeX}

Starting with version 1.4t,\footnote{In alpha-testing at the time of this
writing.} \emTeX's \program{dvihplj} supports 600dpi fonts (for the
LaserJet 4 series) and built-in printer fonts.

The following list of features highlights some of the capabilities of the
\emTeX\ drivers:

\begin{itemize}
  \item Printing of a range of pages
  \item Printing of multiple copies (with or without collating)
  \item Reverse ordering of pages
  \item Selection of duplex printing
  \item Ability to scale \ext{PK} files to the requested size if an appropriate
        \ext{PK} file is unavailable.  This may result in poorer output
        quality due to distortion, but that's not \emTeX's
        fault.
  \item Extremely flexible support for printing booklets and ``n-up''
        arrangements of pages
  \item Selection of paper size and dimensions (margins, etc.)
  \item Transformations (rotation by a multiple of 90 degrees)
  \item Changes in magnification and resolution of output
  \item Support for font libraries
  \item Support for font compression in printers with that feature
  \item Ability to download each page as a large bitmap (to overcome font
        limitations in some printers)
  \item Support for automatic font generation
\end{itemize}

\subsection{dvilj2}

\ixx{\program{dvilj2}}{dvilj2} is a free driver for the HP LaserJet Series II printer.
It is part of the \program{dvi2xx} distribution, which also includes drivers
for the HP LaserJet Series IIP and III printers, as well as a driver for the
IBM 3812 page printer.

\subsection{dvips}

\ixx{\program{dvips}}{dvips} is one of the most popular PostScript \ext{DVI} drivers.
It is available for \Unix, MS-DOS, and OS/2 platforms.  It may also
be available on other platforms because the source code is freely
available.
\program{dvips} supports a wide range of \cs{special} commands for 
controlling the PostScript output, including pictures and figures and
raw PostScript code.

The following list of features highlights some of \program{dvips}'s 
capabilities:

\begin{itemize}
  \item Inclusion of printer-specific prologue files
  \item Support for compressed PostScript pictures and figures
  \item Automatic creation of pseudo-bold and pseudo-italic fonts
  \item Ability to run as a filter
  \item Automatic splitting of documents into sections to prevent
        out-of-memory errors on the printer
  \item Printing of crop marks
  \item Printing of a range of pages (by physical sheet or \TeX\ page number)
  \item Selection of manual-feed on the printer
  \item Limit on number of output pages
  \item Reverse order of pages
  \item Selection of paper type
  \item Changes in magnification of the output
  \item Printing of odd/even pages only
  \item Printing of multiple copies (with or without collating)
  \item Stripping of comments (to avoid printer/spooler bugs)
  \item Moving of printed image left/right or up/down page
  \item Compression of fonts before downloading
\end{itemize}

An option to the \program{dvips} driver can be used to indicate which
printer the output is destined for.  This allows many printer-specific
options (resolution, paper sizes, etc.) to be specified in a configuration
file, removing from the user the burden of remembering them.

Automatic font generation is supported by \program{dvips}, as described
above.  \ixx{\program{dvipsk}}{dvipsk}, a modified version of \program{dvips}, also 
supports a font-searching mechanism that 
greatly simplifies the task of specifying which directories contain
font files.  If any directory specification in the font path ends with
two slashes, \program{dvips} searches that directory and all of its
subdirectories for the font files.  This allows you to create a font
directory structure like the one shown here:

\begin{ttindent}
/usr/local/lib/tex/fonts/\textit{supplier}
/usr/local/lib/tex/fonts/\textit{supplier}/\textit{typeface}
/usr/local/lib/tex/fonts/\textit{supplier}/\textit{typeface}/src
/usr/local/lib/tex/fonts/\textit{supplier}/\textit{typeface}/tfm
/usr/local/lib/tex/fonts/\textit{supplier}/\textit{typeface}/vf
/usr/local/lib/tex/fonts/\textit{supplier}/\textit{typeface}/vpl
/usr/local/lib/tex/fonts/\textit{supplier}/\textit{typeface}/glyphs
/usr/local/lib/tex/fonts/\textit{supplier}/\textit{typeface}/glyphs/pk
\end{ttindent}

This arrangement is advantageous because it organizes your
fonts and simplifies maintenance of the directories
that contain them.  If \filename{/usr/local/lib/tex/fonts//} is 
in the font search path, \program{dvips} will search through
all of the font directories for the files that it needs.  For example,
it might find the \ext{TFM} file for {\tt cmr10} in
\filename{/usr/local/lib/tex/fonts/free/cm/tfm}. 
The directory \filename{/usr/local/lib/tex/fonts/adobe/garamond/glyphs/type1}
is the location for the PostScript source of the Garamond-Italic font.

The \program{dvips} distribution includes the \ixx{\program{afm2tfm}}{afm2tfm}
program for creating \TeX\ font metrics from Adobe \ixx{\ext{AFM} files}{AFM files@\ext{AFM} files}.
This version of \program{afm2tfm} can perform several useful tasks
such as automatically creating appropriate virtual fonts (for mapping
\TeX\ font encodings to PostScript encodings) or changing the encoding
of the PostScript font.

\subsection{DVILASER/HP}
\label{sec:pr:dvilaserhp}
\def\dvilaserhp{\program{DVILASER/HP}}

\idx{ArborText}'s \ixx{\dvilaserhp}{DVILASER/HP drivers} \dvidriver\ translates 
\TeX\ \ext{DVI} files into
a format that can be printed on HP LaserJet printers.  These drivers
are available for both MS-DOS and supported \Unix\ workstations.  The
following discussion is based on experiences with \dvilaserhp\
version 5.3.3, the MS-DOS implementation of ArborText's HP LaserJet driver.

\dvilaserhp\ functions in the way you would expect, generating
LaserJet printable documents from \TeX\ \ext{DVI} files.  It also
has many special features.  Some of the more interesting features
are summarized below.  \dvilaserhp\ can print documents
using \ext{PK} files, HP LaserJet softfonts, and built-in fonts
(including fonts from cartridges).

ArborText supplies a complete set of Computer Modern fonts in \ext{PK} format.
Recent releases of \dvilaserhp\ can use the virtual fonts introduced in \TeX\
version 3.0.

ArborText's \dvilaserhp\ \dvidriver\ provides the following features:

\begin{itemize}
  \item Selectable number of copies
  \item Optional reverse ordering of pages
  \item Selectable manual-feed
  \item Portrait or landscape orientation
  \item Selectable page size
  \item Ability to scale \ext{PK} files to the requested size if an appropriate
        \ext{PK} file is unavailable.  This may result in poorer output
        quality due to distortion, but that's not \dvilaserhp's 
        fault.\footnote{The scaling uses the next larger-sized \ext{PK} file,
        and ArborText claims that this results in good quality over a large 
        range of point sizes without providing \ext{PK} files for every 
        exact size.}
  \item Page movement and reordering options (for printing multiple pages
        on a single sheet of paper, for example)
  \item Configurable font substitution
  \item Support for LaserJet ``overlays.''  \dvilaserhp\ recognizes 
        \cs{special} commands for inserting raw HP LaserJet format documents
        and inserting HP LaserJet overlays\footnote{Consult your HP LaserJet 
        reference manual for more information about overlays.}
  \item An interactive mode (useful for printing only selected pages from
        a document without re-running \dvilaserhp\ many times or when many
        options are going to be used)
  \item Selectable paper tray
\end{itemize}

Many other utilities come with \dvilaserhp; they are summarized in
Table~\ref{tab:dvilaserhp:util}.\index{aftovt}\index{gltopk}\index{hpformat}\index{packpxl}%
\index{painthp}\index{pcltopk}\index{pktopx}\index{pxtopk}\index{spr}%
\index{tftovp}\index{unpkpxl}\index{VFtoVP}\index{VPtoVF}

{\def\x{${}^1$}%
\begin{xtable}{l|l}
  \caption{Other \protect\dvilaserhp\ Utilities\label{tab:dvilaserhp:util}}\\
  \bf Utility & \bf Description \\[2pt]
  \hline
\endfirsthead
  \caption[]{Other \protect\dvilaserhp\ Utilities (continued)}\\
  \bf Utility & \bf Description \\[2pt]
  \hline
\endhead  
  \tstrut
  \it aftovp   & Converts \ext{VPL} from \ext{AFM} file \\
  \it gftopk\,\x & Converts \ext{GF} files into \ext{PK} files \\
  \it hpformat & A print formatter for ASCII files \\
  \it packpxl  & Creates packed (byte-aligned) \ext{PXL} file \\
  \it painthp  & Converts MacPaint files into HP LaserJet format \\
  \it pcltopk\,\x& Converts HP LaserJet softfonts into 
                 \ext{PK}/\ext{TFM} file \\
  \it pktopx\,\x & Converts \ext{PK} files into \ext{PXL} files \\
  \tstrut \it pxtopk\,\x & Converts \ext{PXL} files into \ext{PK} files \\
  \it spr      & A serial-line print spooler\\
  \it tftovp   & Converts \ext{VPL} from \ext{TFM} file \\
  \it unpkpxl  & Creates standard, word-aligned \ext{PXL} file \\
  \it vftovp\,\x & Converts \ext{VF} files into \ext{VPL} files \\
  \it vptovf\,\,\x & Converts \ext{VPL} files into \ext{VF} files \\[2pt]
  \hline
  \multicolumn{2}{l}{%
    ${}^1$\vrule height11pt width0pt\tiny A standard, or otherwise 
    freely available, utility.}
\end{xtable}
}

ArborText recently released \program{\idx{DVILASER/HP3 drivers}} for the HP LaserJet III and IV
printers.  In addition to the features described above, \program{DVILASER/HP3}
supports more complex documents, compressed font downloading, 
600dpi and duplex printing, paper tray and output bin selection, 
and the ability to include TIFF images via a \cs{special} command.

\subsection{DVILASER/PS} 
\label{sec:pr:dvilaserps}
\def\dvilaserps{\program{DVILASER/PS}}

\idx{ArborText}'s \ixx{\dvilaserps}{DVILASER/PS drivers} \dvidriver\ translates \TeX\ \ext{DVI} files
into PostScript.  These drivers are available for both MS-DOS and
supported \Unix\ workstations.  The following discussion is based on
experiences with \dvilaserps\ version 6.3.5, the MS-DOS implementation
of ArborText's PostScript driver.

PostScript \dvidriver{}s are more flexible than many other
\dvidriver{}s because PostScript is a very powerful page description
language.  Some of \dvilaserps's more interesting features are
summarized below.  \dvilaserps\ can print documents using
\ext{PK} files, downloadable PostScript fonts, and built-in fonts.
ArborText supplies \ext{TFM} files for many built-in PostScript fonts.
Recent releases of \dvilaserps\ can use the virtual fonts introduced
in \TeX\ version 3.0.

ArborText's \dvilaserps\ \dvidriver\ provides the following features:

\begin{itemize}
  \item Options for loading custom PostScript prologue code
  \item Ability to be used as a filter (sends PostScript code to 
        standard output)
  \item Selectable number of copies
  \item Optional reverse ordering of pages
  \item Portrait or landscape orientation
  \item Selectable page size
  \item Ability to scale \ext{PK} files to the requested size if an appropriate
        \ext{PK} file is unavailable.  This may result in poorer output
        quality due to distortion, but that's not \dvilaserps's fault.
  \item Page movement and reordering options (for printing multiple pages
        on a single sheet of paper, for example)
  \item Selectable paper tray
  \item Configurable font substitution
  \item Support for LaserJet ``overlays''
  \item An interactive mode (useful for printing only selected pages from
        a document without re-running \dvilaserps\ many times or when many
        options are going to be used)
  \item Ability to download \ext{PK} files permanently, speeding printing of
        future documents
  \item Document, page, or encapsulated PostScript document structuring
        options
  \item Optional clipping of characters that fall outside the normally
        printable page area
\end{itemize}

Many other utilities come with \dvilaserps. They are summarized 
in\index{afmtopl}\index{afmtoplm}\index{aftovp}\index{gftopk}%
\index{packpxl}\index{pktopx}\index{PLtoTF}\index{psformat}\index{pxtopk}%
\index{pltot}\index{spr}\index{tftopl}\index{unpkpxl}%
\index{VFtoVP}\index{VPtoVF} Table~\ref{tab:dvilaserps:util}.

{\def\x{${}^1$}\def\y{${}^2$}
\begin{xtable}{l|l}
  \caption{Other \protect\dvilaserps\ Utilities\label{tab:dvilaserps:util}}\\
  \bf Utility & \bf Description \\[2pt]
  \hline
  \tstrut
  \it afmtopl  & Converts \ext{AFM} files into unmapped \ext{PL} files \\
  \it afmtoplm & Converts \ext{AFM} files into mapped \ext{PL} 
                 files\y\\
  \it aftovp   & Converts \ext{VPL} from \ext{AFM} file \\
  \it gftopk\,\x & Converts \ext{GF} files into \ext{PK} files \\
  \it packpxl  & Creates packed (byte-aligned) \ext{PXL} file \\
  \it pktopx\,\x & Converts \ext{PK} files into \ext{PXL} files \\
  \it pltotf\,\,\x & Converts \ext{PL} files into \ext{TFM} files\\
  \it psformat & A print formatter for ASCII files\\
  \it pxtopk\,\x & Converts \ext{PXL} files into \ext{PK} files \\
  \it spr      & A serial-line print spooler \\
  \it tftopl\,\x & Converts \ext{TFM} files into \ext{PL} files\\
  \it tftovp   & Converts \ext{VPL} from \ext{TFM} file \\
  \it unpkpxl  & Creates standard, word-aligned \ext{PXL} file \\
  \it vftovp\,\x & Converts \ext{VF} files into \ext{VPL} files \\
  \it vptovf\,\,\x & Converts \ext{VPL} files into \ext{VF} files \\[2pt]
  \hline
  \multicolumn{2}{l}{%
    ${}^1$\vrule height11pt width0pt\tiny A standard, or otherwise 
    freely available, utility.} \\
  \multicolumn{2}{l}{%
    ${}^2$\tiny Mapping changes the encoding vector of the font.}
\end{xtable}
}

\dvilaserps\ recognizes \cs{special} commands for inserting raw
PostScript files and commands, encapsulated PostScript files, and
automatic page overlays (which can be selectively enabled and
disabled).  Other \cs{special} commands allow you to set most
of the \dvilaserps\ command-line options directly in the document,
rotate any \TeX\ ``box'' to any angle, and print change
bars.  For \LaTeX\ users, ArborText includes a plug-in replacement
for \LaTeX's picture environments that uses PostScript instead of
special fonts to draw each figure.

\newpage
\subsection{PTI Laser/HP and PTI Jet}

\ixx{\program{PTI Laser/HP}}{PTI Laser/HP} and \ixx{\program{PTI Jet}}{PTI Jet} are 
distributed together
by \ixx{Personal \TeX, Inc.}{Personal tex, Inc.@Personal \TeX, Inc.}  \program{PTI Laser/HP} is an HP LaserJet driver.
\program{PTI Jet} is a DeskJet driver.  The \program{PTI Jet} driver works
with a standard DeskJet printer; no additional options or memory are
required.  The \program{PTI Laser/HP} driver is for HP LaserJet II and III
series printers.  A separate program, \program{PTI Laser/HP4}, is sold
for LaserJet 4 series printers (to support 600dpi fonts, for example).

The following features are available in these drivers:

\begin{itemize}
  \item Support for font substitution
  \item Support for built-in fonts
  \item Printing multiple copies of each page
  \item Setting page size
  \item Printing in landscape mode
  \item Selecting magnification
  \item Ability to print a range of pages 
        (by physical sheet or \TeX\ page number)
  \item Reversing order of pages
  \item Ability to move the page image left/right or up/down the page
  \item Support for 256 character fonts
\end{itemize}

\program{PTI Laser/HP} offers the following additional features,
which are not supported by \program{PTI Jet}:

\begin{itemize}
  \item Ability to reserve printer font ID numbers
  \item A utility program, \program{sftopk}, for converting
        HP LaserJet softfonts into \TeX\ \ext{PK} fonts
  \item Support for directly including HP LaserJet printer commands via
        the \cs{special} mechanism in \TeX
\end{itemize}

\subsection{PTI Laser/PS}

\ixx{\program{PTI Laser/PS}}{PTI Laser/PS} is Personal \TeX's PostScript \dvidriver.  
In addition to the \dvidriver, the \program{PTI Laser/PS} package
includes utilities for spooling output to a serially-connected printer
and converting \ext{AFM} files into \ext{TFM} files.

Some of \program{PTI Laser/PS}'s more interesting options are summarized
below.  

\begin{itemize}
  \item Ability to select different printer resolutions
  \item Selectable number of copies
  \item Configurable font substitution
  \item Selectable page size
  \item An interactive mode for selecting options and reacting to errors
        (characters that fall off the page, missing fonts, etc.)
  \item Portrait or landscape orientation
  \item Selectable \TeX\ magnification
  \item Option files for storing frequently used options.
  \item Selectable output filename
  \item Options to select individual pages or ranges of pages by \TeX\
        page number or physical sheet number
  \item Optional reverse ordering of pages
  \item Adjustable page offset (adjusts the position of the printed page
        on the physical page)
  \item Support for 256-character fonts
  \item Special support for the eccentricities of the Apple LaserWriter
        printer
\end{itemize}

\program{PTI Laser/PS} recognizes a \cs{special} command for inserting
raw PostScript files into the printed document.  The horizontal and
vertical size of the inserted image can be changed.

\subsection{dvipsone}

\ixx{\program{dvipsone}}{dvipsone} is a commercial PostScript driver for MS-DOS
distributed by \ixx{\YY\ Inc.}{YY Inc.@\YY\ Inc.} It
produces PostScript output from a \TeX\ \ext{DVI} file.
\program{dvipsone} is designed for a ``bitmap-free'' environment.
This makes \program{dvipsone} almost unique among \ext{DVI} drivers
because it {\em cannot\/} use standard \ext{PK} fonts.  To
use \program{dvipsone}, you must have PostScript fonts for
every font that you use (or the font must be built into the
printer).\footnote{The \program{dvipsone} distribution includes a utility
which can convert \ext{PK} fonts into bitmapped PostScript fonts.
This program translates a \ext{PK} file into an Adobe
Type~3 font.  Once in Type~3 format, \program{dvipsone} can
use the font.  Using Type~3 fonts created in this way 
adds resolution-dependence to your PostScript file.}
\YY\ sells a complete set of Computer Modern fonts
in Adobe Type~1 format.  

\program{dvipsone} has a rich set of features:

\begin{itemize}
  \item Printing multiple \ext{DVI} files with a single command
  \item Printing pages in reverse order
  \item Printing only odd or even pages
  \item Ability to force output to conform to EPS standards
  \item Assumption that all requested fonts are printer-resident
  \item Insertion of verbatim PostScript
  \item Printing a range of pages
  \item Resizing of the output
  \item Rotation of output by arbitrary angle
  \item Shifting output left/right or up/down
  \item Printing arbitrary number of copies
  \item Selection of paper type (letter, landscape, legal, etc.)
  \item Insertion of user-specified PostScript prologue
  \item Conservation of memory by downloading partial fonts
  \item Remapping of the font encoding on-the-fly
  \item Support for ten different styles of \cs{special} commands for
        including encapsulated PostScript images
\end{itemize}

Perhaps the most interesting feature of the \program{dvipsone} driver is its ability to
download partial PostScript fonts.  \YY\ claim that partial font downloading
is a feature unique to \program{dvipsone}.  A moderately complex \TeX\
document may use twenty or more fonts and each font is typically between 20K
and 30K.  This means that the \dvidriver\ downloads roughly 500K of font data
for the document.  In addition to requiring considerable printer memory, all
that font data increases the amount of time that it takes to send your
document to the printer.  The partial font downloading feature of
\program{dvipsone} means that only the characters that are actually used in
your document are sent to the printer.  The ability to download partial fonts
can result in a substantially smaller PostScript file.

Several other utilities are distributed with \program{dvipsone}.  They
are summarized in%
\index{download}\index{afmtotfm}\index{tfmtoafm}\index{pfatopfb}%
\index{pfbtopfa}\index{twoup}\index{pktops} Table~\ref{tab:dvipsone:util}.

{\def\x{${}^1$}%
\begin{xtable}{l|l}
  \caption{Other dvipsone Utilities\label{tab:dvipsone:util}}\\
  \textnormal{\bf Utility} & \bf Description \\[2pt]
  \hline
  \tstrut
  \filename{download} & Robust font-downloading program for PostScript fonts\\
  \filename{afmtotfm}\,\x & Converts \ext{AFM} files into \ext{TFM} files\\
  \filename{tfmtoafm} & Converts \ext{TFM} files into \ext{AFM} files\\
  \filename{pfatopfb}\,\x  & Convert \ext{PFA} files into \ext{PFB} files\\
  \filename{pfbtopfa}\,\x  & Convert \ext{PFB} files into \ext{PFA} files\\
  \filename{twoup}    & Reorders pages in a PostScript file\\
  \filename{pktops}   & Provides access to \ext{PK} fonts for 
                        \program{dvipsone}\\[2pt]
  \hline
  \multicolumn{2}{l}{%
    ${}^1$\vrule height11pt width0pt\tiny A standard, or otherwise 
    freely available, utility.}
\end{xtable}
}

